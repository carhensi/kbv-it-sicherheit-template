% Tabellen-Pakete
\usepackage{makecell}

% Layout und Typografie
\usepackage[main=ngerman]{babel}
% LuaLaTeX nutzt automatisch OpenType-Fonts ohne fontspec
\usepackage{microtype}

% Symbol-Fallback für Unicode-Zeichen (PDF/A-2u)
\usepackage{fontspec}
\newfontfamily\symbolsfont{DejaVu Sans}[Scale=MatchLowercase]
\usepackage{newunicodechar}
\newunicodechar{✓}{{\symbolsfont ✓}} % U+2713
\newunicodechar{✗}{{\symbolsfont ✗}} % U+2717
\newunicodechar{□}{{\symbolsfont □}} % U+25A1
\newunicodechar{☑}{{\symbolsfont ☑}} % U+2611

% Seitenlayout (print-optimiert)
\usepackage[a4paper,inner=28mm,outer=22mm,top=22mm,bottom=26mm]{geometry}
\usepackage{setspace}
\usepackage[absolute,overlay]{textpos}
\setstretch{1.15}

% PDF und Hyperlinks
\usepackage[
    bookmarks=true,
    bookmarksopen=true,
    bookmarksnumbered=true,
    hidelinks,
    colorlinks=true,
    linkcolor=black,
    urlcolor=blue,
    citecolor=black,
    pdfstartview=FitH,
    pdfpagelayout=OneColumn,
    pdfa
]{hyperref}
\usepackage{hyperxmp} % richer PDF metadata (load after hyperref)

% Fix für hyperref PDF-String Warnungen
\pdfstringdefDisableCommands{%
  \def\\{ }% Zeilenumbruch als Leerzeichen in PDF-Strings
  \let\tel\@firstofone% \tel{123} wird zu 123 in PDF-Strings
}
\usepackage{bookmark}

% Tabellen und Daten
\usepackage{tabularx}
\usepackage{longtable}
\usepackage{booktabs}
\usepackage{csvsimple}
\usepackage{array}
\usepackage{adjustbox}

% Tabellenlinien für sauberen Druck optimieren
\setlength{\arrayrulewidth}{0.5pt}

% Grafiken und Diagramme
\usepackage{graphicx}
\usepackage{tikz}
\usetikzlibrary{arrows.meta,positioning,shapes,calc}
\usepackage[rgb]{xcolor}  % RGB-Modus für PDF/A-2u
\usepackage{colortbl}

% Listen und Formatierung - Standard LaTeX für besseres Tagging
\ifaccessible
  % Accessible: Standard-Listen ohne enumitem für korrektes Tagging
  \setlength{\topsep}{0pt}
  \setlength{\partopsep}{0pt}
  \setlength{\itemsep}{0pt}
  \setlength{\parsep}{0pt}
\else
  % Standard: enumitem für kompakte Listen
  \usepackage{enumitem}
  \setlist{
    topsep=0pt,
    partopsep=0pt,
    itemsep=0pt,
    parsep=0pt,
    before={\setlength{\parskip}{0pt}},
    after={}
  }
\fi

% Code und Listings
\usepackage{listings}
\lstset{
  basicstyle=\ttfamily\small,
  frame=single,
  columns=fullflexible,
  breaklines=true
}

% Seitenlayout
\ifaccessible
  % Accessible: fancyhdr statt scrlayer-scrpage
  \usepackage{fancyhdr}
\else
  % Standard: KOMA scrlayer-scrpage
  \usepackage{scrlayer-scrpage}
\fi
\usepackage{refcount}
\usepackage{pdflscape}

% Deutsche Typografie
\usepackage[babel,german=quotes]{csquotes}
\usepackage{xspace}

% VVT Table Macros
\usepackage{ragged2e}   % \RaggedRight in Tabellen

% Hilfsmakro: farbige Kopfzeile ohne Weißsaum
\newcommand{\VVTHeader}[2]{%
  \noalign{\global\aboverulesep=0pt\global\belowrulesep=0pt}%
  \specialrule{\heavyrulewidth}{0pt}{0pt}% oben (entspricht \toprule)
  \rowcolor{#1}\multicolumn{2}{c}{\textbf{#2}}\\
  \specialrule{\lightrulewidth}{0pt}{0pt}% unten (entspricht \midrule)
  \noalign{\global\aboverulesep=0.4ex\global\belowrulesep=0.65ex}%
}

% Tabellenspalten: links 26%, rechts 66% (Summe < 100%)
\newcolumntype{L}{>{\RaggedRight\arraybackslash}p{0.26\textwidth}}
\newcolumntype{R}{>{\RaggedRight\arraybackslash}p{0.66\textwidth}}
