% ============================================================================
% ZENTRALE METADATEN-KONFIGURATION
% ============================================================================
% Alle praxisspezifischen Daten als Makros
% HINWEIS: Kopieren Sie diese Datei nach metadata.tex und passen Sie die Werte an.

% ============================================================================
% 1. PRAXIS-GRUNDDATEN
% ============================================================================
\newcommand{\PraxisName}{Musterpraxis für Psychotherapie Dr. Erika Mustermann}
\newcommand{\PraxisNameMitUmbruch}{Musterpraxis für Psychotherapie\\Dr. Erika Mustermann}
\newcommand{\PraxisInhaberin}{Dr. Erika Mustermann}
\newcommand{\PraxisInhaberBezeichnung}{Praxisinhaberin} % oder: Praxisinhaber
\newcommand{\PraxisGroesse}{Einzelpraxis ($\leq$ 5 Mitarbeitende), nur eine Betriebsstätte}
\newcommand{\StandortTyp}{Einzelpraxis}
\newcommand{\PraxisFachrichtung}{Psychotherapie (Verhaltenstherapie)}

% Adresse und Kontakt
\newcommand{\PraxisAdresse}{Musterstraße 1}
\newcommand{\PraxisOrt}{12345~Musterstadt}
\newcommand{\PraxisVollAdresse}{\PraxisAdresse, \PraxisOrt}
\newcommand{\PraxisTelefon}{\tel{151 12345678}}
\newcommand{\PraxisInhaberinPrivatTelefon}{\tel{170 1234567}}

% Online-Präsenz
\newcommand{\PraxisDomain}{musterpraxis-psychotherapie.de}
\newcommand{\PraxisEmail}{info@\PraxisDomain}
\newcommand{\PostmasterEmail}{dmarc-reports@\PraxisDomain}
\newcommand{\PraxisWebsite}{https://www.\PraxisDomain}
\newcommand{\PraxisWebsiteDatenschutz}{\PraxisWebsite/datenschutz/}

% ============================================================================
% 2. VERANTWORTLICHKEITEN UND ROLLEN
% ============================================================================
\newcommand{\Gesamtverantwortung}{\PraxisInhaberin{} (Praxisinhaberin)}
\newcommand{\ITSicherheitsbeauftragte}{\PraxisInhaberin}
\newcommand{\Datenschutzbeauftragte}{\PraxisInhaberin}

% IT-Support
\newcommand{\ITAnsprechpartner}{Max Mustermann}
\newcommand{\ITAnsprechpartnerAdresse}{Beispielweg~42, 54321 Beispielstadt}
\newcommand{\ITAnsprechpartnerTelefon}{\tel{172 9876543}}
\newcommand{\ITAnsprechpartnerMail}{it@example.com}

% Notfall-Kontakte
\newcommand{\NotfallVollmachtPerson}{Volker Vollmacht}

% ============================================================================
% 3. BEHÖRDEN UND RECHTLICHES
% ============================================================================
% Bundesland und Behörden (Beispiel: Muster-Bundesland – bitte anpassen!)
\newcommand{\Bundesland}{Muster-Bundesland}
\newcommand{\Gerichtsstand}{Musterstadt, Deutschland}

% Kassenärztliche Vereinigung
\newcommand{\KV}{Muster-KV}
\newcommand{\KVLang}{Kassenärztliche Vereinigung Muster-Bundesland}

% Kammer
\newcommand{\Kammer}{Muster-Kammer}
\newcommand{\KammerLang}{Psychotherapeutenkammer Muster-Bundesland}

% Datenschutzbehörde
\newcommand{\Datenschutzbehoerde}{LfD \Bundesland}
\newcommand{\DatenschutzbehoerdeAdresse}{Musterstraße 1, 12345 Musterstadt}
\newcommand{\DatenschutzbehoerdeURL}{lfd.muster-bundesland.de}

% ============================================================================
% 4. IT-INFRASTRUKTUR UND NETZWERK
% ============================================================================
\newcommand{\PraxisITInfrastruktur}{Lokales PVS (\PVSName{}), extern gehostetes TI-Gateway}

% Netzwerk-Grundkonfiguration
\newcommand{\PraxisSubnetz}{192.168.178.0/24}
\newcommand{\NetzwerkUplink}{Uplink via DSL-Modem}

% Router
\newcommand{\RouterName}{praxis-router}
\newcommand{\RouterKurz}{Router}
\newcommand{\RouterModell}{Muster-Router}

% WLAN
\newcommand{\WLANName}{MusterPraxis-WLAN}
\newcommand{\WLANGastName}{MusterPraxis-Gast}
\newcommand{\WLANAbschaltung}{20:00–06:00~Uhr}
\newcommand{\MACRandomisierung}{Konfiguriert gemäß Praxisanforderungen}

% ============================================================================
% 5. SOFTWARE UND TOOLS
% ============================================================================
% Praxisverwaltungssystem
\newcommand{\PVSName}{MusterPVS}
\newcommand{\PVSHersteller}{PVS-Anbieter GmbH}
\newcommand{\PVSSupportTel}{0800 123 4567}

% Sicherheits-Tools
\newcommand{\PasswortManager}{Passwort-Manager}
\newcommand{\MDMTool}{MusterMDM}
\newcommand{\RemoteTool}{Remote-Support-Tool}

% KI-Tools
\newcommand{\KIToolName}{MusterAI Docs}
\newcommand{\KIToolAnbieter}{MusterAI GmbH}
\newcommand{\KIToolSicherheitURL}{https://example.com/ki-sicherheit}

% Patientenkommunikation
\newcommand{\PatKomTool}{Patienten-App}
\newcommand{\PatKomAnbieter}{Patienten-App GmbH}

% Entwicklung und Deployment
\newcommand{\IaCTool}{IaC}
\newcommand{\IaCToolLang}{Infrastructure-as-Code (z.B. Terraform)}
\newcommand{\CodeRepo}{Code-Repository}

% ============================================================================
% 6. HOSTING UND EXTERNE DIENSTE
% ============================================================================
% E-Mail
\newcommand{\EmailAnbieter}{E-Mail-Anbieter}
\newcommand{\EmailAnbieterFirma}{E-Mail-Anbieter GmbH}

% Hosting
\newcommand{\HostingAnbieter}{Hosting-Anbieter}
\newcommand{\HostingRegion}{EU (Frankfurt)}

% CDN und DNS
\newcommand{\CDNService}{CDN}
\newcommand{\DNSService}{DNS-Service}

% Internet und Mobilfunk
\newcommand{\InternetAnbieter}{Internetanbieter}
\newcommand{\InternetSperrnummer}{0800 XXX XXXX}
\newcommand{\MobilfunkAnbieter}{Mobilfunkanbieter}
\newcommand{\SIMSperrnummer}{116 116}

% ============================================================================
% 7. SICHERHEIT UND ZERTIFIKATE
% ============================================================================
% Passwort-Richtlinien
\newcommand{\PasswortMindestlaenge}{14}
\newcommand{\PasswortSecurityScore}{90\%}

% S/MIME Zertifikate
\newcommand{\SMIMEIssuer}{Zertifikatsanbieter}
\newcommand{\SMIMEValidity}{2-Jahres-Zyklus}

% System-Accounts
\newcommand{\AdminUser}{praxisadmin}

% ============================================================================
% 8. BACKUP UND NOTFALL
% ============================================================================
\newcommand{\BackupStandort}{georedundanter externer Standort}
\newcommand{\NotfallkitVerwahrung}{Sicherer externer Aufbewahrungsort}

% ============================================================================
% 9. REPOSITORY UND ENTWICKLUNG
% ============================================================================
\newcommand{\GitHubRepoDoc}{https://github.com/carhensi/kbv-it-sicherheit-template}
\newcommand{\GitHubRepoWeb}{https://github.com/beispiel/praxis-web}

% ============================================================================
% 10. DOKUMENT-METADATEN
% ============================================================================
% Versionierung (wird von build.py überschrieben)
\newcommand{\DocumentVersion}{2025.12.28}
\newcommand{\DocumentDate}{26. Dezember 2025}
\newcommand{\ValidUntil}{25.12.2028}
\newcommand{\NextReview}{Dezember 2027}
\newcommand{\DokumentKlassifizierung}{Vorlage (nicht vertraulich)}
