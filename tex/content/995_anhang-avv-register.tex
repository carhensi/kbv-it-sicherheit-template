\section{Anhang – Auftragsverarbeitungsverträge (AVV-Register)}
\label{sec:avv-register}

\subsection*{Auftragsverarbeitungsverträge (AVV)}

\textbf{Stand:} \today

\vspace{1em}

\begin{longtable}{>{\raggedright\arraybackslash}p{3.8cm} >{\raggedright\arraybackslash}p{4.2cm} >{\raggedright\arraybackslash}p{3.2cm} >{\raggedright\arraybackslash}p{2.3cm}}
  \toprule
  \textbf{Dienstleister} & \textbf{Verarbeitungs\-zweck} & \Label{Status} &
  \Label{Datum} \\
  \midrule

  \PVSHersteller{} & PVS/Behandlung & \textcolor{StatusGreen}{vorhanden} & 01.09.2025 \\

  \HostingAnbieter{} & Website-Hosting & \textcolor{StatusGreen}{vorhanden, C5-Testat geprüft \DocumentDate} & 01.09.2025 \\

  \EmailAnbieterFirma{} & E-Mail (organisatorisch, keine Gesundheitsdaten) &
  \textcolor{StatusGreen}{vorhanden} & 30.11.2025 \\

  Apple Inc. (Business Manager) & MDM/Backup (verschlüsselte iCloud-Backups, keine Gesundheitsdaten) &
  \textcolor{StatusGreen}{vorhanden} & 01.09.2025 \\

  \MDMTool{} Software LLC (\MDMTool{}) & Mobile Device Management & \textcolor{StatusGreen}{vorhanden} & 01.09.2025 \\

  \PatKomAnbieter{} & Sichere~Patienten-Kommunikation & \textcolor{StatusGreen}{vorhanden} & 01.09.2025 \\

  \KIToolAnbieter{} & KI-gestützte Sitzungsdokumentation & \textcolor{StatusGreen}{vorhanden, C5-Testat geprüft \DocumentDate} & 01.09.2025 \\

  \bottomrule
\end{longtable}

\vspace{1em}

\Legende
\begin{itemize}
  \item \textcolor{StatusGreen}{vorhanden} = AVV vorhanden und aktuell
  \item \textcolor{StatusOrange}{geplant} = AVV geplant/in Bearbeitung
  \item \textcolor{StatusRed}{zu prüfen} = Status zu prüfen
  \item \textcolor{StatusGray}{nicht erforderlich} = Kein AVV erforderlich/möglich
  \item \textcolor{StatusGray}{nicht verfügbar} = AVV derzeit nicht verfügbar
\end{itemize}

\vspace{1em}

\textbf{Apple Inc. – Detaillierter Status und Monitoring:}

\textbf{Aktuelle Situation:} Apple Business Manager Customer Data Processing Addendum
ist verfügbar und wurde aktiviert. Apple Business Manager wird ausschließlich für
Mobile Device Management (MDM) und verschlüsselte iCloud-Backups genutzt.
Gesundheitsdaten werden grundsätzlich nicht über Apple-Dienste verarbeitet.

\textbf{Wichtig:} E-Mail-Kommunikation erfolgt ausschließlich über \EmailAnbieter{}
(siehe Kap.~\ref{sec:kommunikation}). Kalender wird über \EmailAnbieter{} oder lokal verwaltet.

\textbf{Halbjährliches Monitoring:}
\begin{itemize}
  \item \textbf{AVV-Compliance} - Regelmäßige Prüfung der Vertragsbedingungen
  \item \textbf{Strikte Datentrennung} - Keine Gesundheitsdaten in Apple-Diensten
  \item \textbf{Bewertung} bei neuen Apple Business-Optionen für Gesundheitswesen
  \item \textbf{Nächste Prüfungen:} März 2026, September 2026
\end{itemize}

\textbf{Rechtliche Einordnung:} Nutzung erfolgt auf Basis berechtigter Interessen
(Art. 6 Abs. 1 lit. f DSGVO) für organisatorische Zwecke ohne Gesundheitsdatenbezug.
Gesundheitskommunikation erfolgt ausschließlich über \PatKomTool{} (mit AVV).

\subsection*{Hinweise zur Pflege}

\textbf{Regelmäßige Prüfung:} Dieses Register wird quartalsweise auf Aktualität geprüft
und bei Änderungen der Dienstleister oder Verträge aktualisiert. \textbf{AVV-Verträge
werden jährlich überprüft} (siehe TOMs, Maßnahme zur Auftragsverarbeitung).

\Dokumentation Die vollständigen AVV-Verträge werden digital archiviert
(\PasswortManager{}/sicherer Ordner) und sind bei Prüfungen elektronisch oder als Ausdruck
vorzulegen.

\textbf{Verantwortlichkeit:} Die \PraxisInhaberBezeichnung{} ist für die Vollständigkeit und
Aktualität der AVVs verantwortlich.
