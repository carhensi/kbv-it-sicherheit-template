\section{Vorlage – Protokoll der Backup-Wiederherstellungstests}
\label{sec:vorlage-restore-tests}

Restore-Tests werden mindestens einmal jährlich sowie anlassbezogen mit repräsentativen 
Testdateien durchgeführt und dokumentiert. Halbjährliche Tests (Januar/Juli) sind für 
Einzelpraxen wünschenswert, aber nicht verpflichtend,
um die vollständige Wiederherstellbarkeit aller Backup-Daten zu gewährleisten:

\textbf{Hinweis:} Die nachfolgende Tabelle ist als Template vorgesehen und wird im
laufenden Betrieb handschriftlich geführt. Spätestens zum Zeitpunkt einer Prüfung liegen
ausgefüllte Blätter vor.

% Stellschrauben - verwende gleiche Checkboxen wie TOMS
\providecommand{\qsize}{1.05ex}
\renewcommand{\qsize}{1.05ex}
\providecommand{\q}{\fbox{\rule{0pt}{\qsize}\rule{\qsize}{0pt}}}
\renewcommand{\q}{\fbox{\rule{0pt}{\qsize}\rule{\qsize}{0pt}}}
\renewcommand{\arraystretch}{1.0}% Zeilenhöhe wie TOMS
\setlength{\tabcolsep}{6pt}       % Zell-Innenabstand

% Sexy Tabelle mit booktabs (ohne vertikale Linien)
\begin{center}
  {\small
  \begin{tabular}{p{0.15\textwidth} p{0.15\textwidth} p{0.12\textwidth} p{0.42\textwidth}}
    \toprule
    \Label{Datum} & \textbf{Time Machine (Lokal)} & \textbf{NAS (Offsite)} & \textbf{Bemerkungen} \\
    \midrule
    \rule{0pt}{2.6ex} & \q & \q & \\
    \midrule
    \rule{0pt}{2.6ex} & \q & \q & \\
    \midrule
    \rule{0pt}{2.6ex} & \q & \q & \\
    \midrule
    \rule{0pt}{2.6ex} & \q & \q & \\
    \midrule
    \rule{0pt}{2.6ex} & \q & \q & \\
    \midrule
    \rule{0pt}{2.6ex} & \q & \q & \\
    \midrule
    \rule{0pt}{2.6ex} & \q & \q & \\
    \midrule
    \rule{0pt}{2.6ex} & \q & \q & \\
    \bottomrule
  \end{tabular}
  } % Ende der small-Gruppe
\end{center}

\Legende \fbox{\cmark} = Erfolgreich bestanden, \fbox{\xmark} = Fehler aufgetreten

\textbf{Anleitung zum Ausfüllen:}
\begin{itemize}
  \item \textbf{Datum:} Testdatum eintragen (TT.MM.JJJJ).
  \item \textbf{Bemerkungen:} Probleme oder besondere Vorkommnisse kurz notieren.
  \item \textbf{Gerät:} MacBook Air (fest vorgegeben).
  \item \textbf{Häufigkeit:} Alle 6 Monate (empfohlen: Januar \& Juli).
\end{itemize}
