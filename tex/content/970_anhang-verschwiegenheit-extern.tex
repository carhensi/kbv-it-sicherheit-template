\section{Anhang – Vereinbarung über die Vertraulichkeit (extern)}
\label{sec:anhang-verschwiegenheit}

\subsection*{Vereinbarung über die Vertraulichkeit}

\textbf{1.} Zwischen

\ifaccessible
  % Accessible: Einfache Struktur ohne minipage
  \textbf{Auftraggeberin:}\\
  Praxis für Psychotherapie\\
  \PraxisInhaberin\\
  \PraxisAdresse\\
  \PraxisOrt

  \vspace{0.5cm}
  \textbf{Auftragnehmer:}\\
  \rule{5cm}{0.4pt}\\[0.2cm]
  \rule{5cm}{0.4pt}\\[0.2cm]
  \rule{5cm}{0.4pt}
\else
  % Standard: minipage-Struktur
  \begin{minipage}[t]{0.45\textwidth}
    \textbf{Auftraggeberin:}\\
    Praxis für Psychotherapie\\
    \PraxisInhaberin\\
    \PraxisAdresse\\
    \PraxisOrt
  \end{minipage}
  \hfill
  \begin{minipage}[t]{0.45\textwidth}
    \textbf{Auftragnehmer:}\\
    \rule{5cm}{0.4pt}\\[0.2cm]
    \rule{5cm}{0.4pt}\\[0.2cm]
    \rule{5cm}{0.4pt}
  \end{minipage}
\fi

wird folgende Vertraulichkeitsvereinbarung geschlossen:

\textbf{2.} Dem Auftragnehmer werden zum Zwecke der IT-Dienstleistungen vertrauliche
Informationen der Auftraggeberin bekannt.

\textbf{3.} Vertrauliche Informationen im Sinne dieser Vereinbarung sind sämtliche
Informationen (ob schriftlich, elektronisch, mündlich, digital verkörpert oder in anderer
Form), die von der Auftraggeberin an den Auftragnehmer zum vorgenannten Zweck offenbart
werden. Als vertrauliche Informationen gelten insbesondere Patientendaten,
Geschäftsgeheimnisse, geschäftliche Beziehungen, Geschäftsstrategien, Finanzplanung,
Personalangelegenheiten sowie das Bestehen dieser Vereinbarung und ihr Inhalt.

\textbf{4.} Keine vertrauliche Informationen sind solche Informationen, die der
Öffentlichkeit vor der Mitteilung oder Übergabe durch die Auftraggeberin bekannt oder
allgemein zugänglich waren oder dies zu einem späteren Zeitpunkt ohne Verstoß gegen eine
Geheimhaltungspflicht werden; die dem Auftragnehmer bereits vor der Offenlegung durch die
Auftraggeberin und ohne Verstoß gegen eine Geheimhaltungspflicht nachweislich bekannt
waren; die von dem Auftragnehmer ohne Nutzung oder Bezugnahme auf vertrauliche
Informationen selber gewonnen wurden; oder die der Auftragnehmer von einem berechtigten
Dritten ohne Verstoß gegen eine Geheimhaltungspflicht übergeben oder zugänglich gemacht werden.

\textbf{5.} Der Auftragnehmer verpflichtet sich, die vertraulichen Informationen streng
vertraulich zu behandeln und nur im Zusammenhang mit dem vorgenannten Zweck zu verwenden;
die vertraulichen Informationen nur gegenüber solchen Vertretern offen zu legen, die auf
die Kenntnis dieser Informationen angewiesen sind, vorausgesetzt, dass der Auftragnehmer
sicherstellt, dass seine Vertreter diese Vereinbarung einhalten, als wären sie selbst
durch diese Vereinbarung gebunden; die vertraulichen Informationen ebenfalls durch
angemessene Geheimhaltungsmaßnahmen gegen den unbefugten Zugriff durch Dritte zu sichern
und bei der Verarbeitung der vertraulichen Informationen die gesetzlichen und
vertraglichen Vorschriften zum Datenschutz einzuhalten. Dies beinhaltet auch dem
aktuellen Stand der Technik angepasste technische Sicherheitsmaßnahmen (Art.~32 DS-GVO)
und die Verpflichtung der Mitarbeiter auf die Vertraulichkeit und die Beachtung des
Datenschutzes (Art.~28 Abs.~3 lit.~b DS-GVO).

\textbf{6.} Sofern der Empfänger aufgrund geltender Rechtsvorschriften gerichtlicher oder
behördlicher Anordnungen verpflichtet ist, teilweise oder sämtliche vertraulichen
Informationen offenzulegen, die Auftraggeberin hierüber unverzüglich schriftlich zu
informieren und alle zumutbaren Anstrengungen zu unternehmen, um den Umfang der
Offenlegung auf ein Minimum zu beschränken und der Auftraggeberin erforderlichenfalls
jede zumutbare Unterstützung zukommen zu lassen.

\textbf{7.} Auf Aufforderung der Auftraggeberin sowie ohne Aufforderung spätestens nach
Erreichung oben genannten Zwecks ist der Auftragnehmer verpflichtet, sämtliche
vertraulichen Informationen einschließlich der Kopien hiervon innerhalb von zehn (10)
Arbeitstagen nach Zugang der Aufforderung bzw.\ nach Beendigung des Zwecks zurückzugeben
oder zu vernichten, sofern nicht mit der Auftraggeberin vereinbarte oder gesetzliche
Aufbewahrungspflichten dem entgegenstehen.

\textbf{8.} Auf Verlangen der Auftraggeberin hat der Auftragnehmer schriftlich zu
versichern, dass er sämtliche vertrauliche Informationen nach den Maßgaben der
vorstehenden Ziffer und den Weisungen der Auftraggeberin vollständig und unwiderruflich
gelöscht hat.

\textbf{9.} Verletzt der Auftragnehmer oder Mitarbeiter des Auftragnehmers oder sonstige
Personen, für die der Auftragnehmer gemäß §§~31, 278, 831 BGB einzustehen hat, die sich
aus dieser Vereinbarung ergebenden Pflichten, verpflichtet sich der Auftragnehmer für
jeden Fall der Zuwiderhandlung unter Ausschluss der Einrede des Fortsetzungszusammenhangs
eine Vertragsstrafe in Höhe von bis zu 25.000~EUR an die Auftraggeberin zu zahlen. Die
Höhe der Vertragsstrafe wird von der Auftraggeberin nach billigem Ermessen festgelegt.
Die Höhe der Vertragsstrafe ist vom zuständigen Gericht überprüfbar. Die sonstigen
Ansprüche, insbesondere etwaige Schadensersatzansprüche, auf die jedoch die
Vertragsstrafe angerechnet wird, bleiben unberührt.

\textbf{10.} Die Bestimmungen dieser Vereinbarung unterliegen in ihrer Durchführung und
Auslegung deutschem Recht unter Ausschluss des internationalen Privatrechts.
Ausschließlicher Gerichtsstand für Streitigkeiten aus oder im Zusammenhang mit der
Vereinbarung ist \Gerichtsstand.

\textbf{11.} Der Auftragnehmer hat ein Exemplar dieser Vereinbarung erhalten.

\vspace{1cm}

\ifaccessible
  % Accessible: Einfache Struktur ohne minipage
  \noindent
  Ort, Datum: \FormField[4cm] \hfill \PraxisInhaberin (\PraxisInhaberBezeichnung{}): \FormField[4cm]

  \vspace{2cm}
  \noindent
  Ort, Datum: \FormField[4cm] \hfill Auftragnehmer: \FormField[4cm]
\else
  % Standard: minipage-Struktur
  \begin{minipage}[t]{0.45\textwidth}
    \centering
    \rule{4cm}{0.3pt}\\[0.3cm]
    {\small Ort, Datum}
  \end{minipage}
  \hfill
  \begin{minipage}[t]{0.45\textwidth}
    \centering
    \rule{4cm}{0.3pt}\\[0.3cm]
    {\small \PraxisInhaberin \\
    (\PraxisInhaberBezeichnung{})}
  \end{minipage}

  \vspace{2cm}

  \begin{minipage}[t]{0.45\textwidth}
    \centering
    \rule{4cm}{0.3pt}\\[0.3cm]
    {\small Ort, Datum}
  \end{minipage}
  \hfill
  \begin{minipage}[t]{0.45\textwidth}
    \centering
    \rule{4cm}{0.3pt}\\[0.3cm]
    {\small Auftragnehmer}
  \end{minipage}
\fi

\vspace{0.5cm}

\textbf{Hinweis:} Diese Vereinbarung ist Bestandteil der IT-Sicherheitsdokumentation nach
§~390 SGB~V.
