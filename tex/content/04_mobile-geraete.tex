\section{Mobile Geräte}
\label{sec:mobile-geraete}

\subsection{Geltungsbereich}
Diese Richtlinie gilt für die \PraxisInhaberBezeichnung{} als Einzelpraxis ohne weitere Beschäftigte.
iPhone und iPad werden ausschließlich dienstlich genutzt, private Geräte sind strikt
getrennt (kein\ BYOD).

\subsection{Kernmaßnahmen}
Mobile Geräte sind mit biometrischer Authentifizierung (FaceID) und komplexen
Gerätesperrcodes geschützt. Technische Konfiguration und Sicherheitsrichtlinien
→ geregelt in \cref{sec:mdm}.

Details verbindlich in \cref{sec:anhang-mobile}.

\textbf{Genutzte Apps:} iOS-Standard-Apps (Mail, Kalender, Safari) und \PVSName{}-App.
Für Kurznachrichten ausschließlich \enquote{Nachrichten} (SMS).

\textbf{App-Richtlinie:} Es gilt das Allowlist-Prinzip – nur explizit freigegebene
Apps dürfen auf Praxisgeräten installiert werden. Explizit verboten sind insbesondere:
WhatsApp, Telegram, Signal (für berufliche Kommunikation), Social-Media-Apps
(Facebook, Instagram, TikTok) sowie private Cloud-Dienste (Dropbox, Google Drive).

\textbf{App-Berechtigungen:} → geregelt in \cref{sec:mdm}.

\textbf{App-Datenschutz:} → geregelt in \cref{sec:mdm}.

\textbf{Sprachassistent:} Siri ist auf allen Geräten deaktiviert, um unbeabsichtigte
Datenpreisgabe und Abhörrisiken zu vermeiden (Anlage~2 Anforderung).

\subsection{Smart Devices und Wearables}

\textbf{Smart Home Geräte:} Keine Smart Home Geräte (HomePod, Alexa, etc.) in den
Praxisräumen, um Abhörrisiken und unbeabsichtigte Datenpreisgabe zu vermeiden.

\textbf{Wearables:} Private Wearables (Apple Watch, etc.) werden während der Behandlung
verschlossen aufbewahrt oder haben deaktivierte Sprachassistenten. Aufzeichnungen nur
mit expliziter Patienteneinwilligung und über C5-zertifizierte, DSGVO-konforme Anwendungen.

\subsection{Datenverarbeitung}
Patientendaten werden nur verschlüsselt übertragen (TLS) und nicht in nicht-konformen
Cloud-Diensten gespeichert. Bei Verlust erfolgt unverzügliche Meldung und Fernlöschung
über \enquote{Mein iPhone suchen}.

\textbf{Kontaktdaten-Management:} Patientenkontakte werden ausschließlich im
Praxisverwaltungssystem (\PVSName{}) gepflegt. E-Mail-Adressen sind nur temporär in der
E-Mail-Inbox (\EmailAnbieter{}) für organisatorische Kommunikation vorhanden, werden aber
nicht als separate Kontakte gespeichert. Die Verarbeitung ist zweckgebunden und im
Verzeichnis von Verarbeitungstätigkeiten gemäß DSGVO Art.\ 30 dokumentiert.

\subsection{Dokumentation}
Die detaillierte Selbstverpflichtung zur Einhaltung der Mobile-Device-Richtlinie ist in
\cref{sec:anhang-mobile} dokumentiert und wurde von der \PraxisInhaberBezeichnung{}
unterzeichnet (siehe Mobile Geräte Selbstverpflichtung).

\textbf{Technische Umsetzung:} Die organisatorischen Richtlinien für mobile Geräte
werden durch das zentrale Mobile Device Management (MDM) technisch durchgesetzt und
überwacht, wie im folgenden Kapitel detailliert beschrieben.
