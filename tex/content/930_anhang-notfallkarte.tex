\section*{} % Keine sichtbare Überschrift
\stepcounter{section} % Zähler erhöhen für korrekte Nummerierung
\addcontentsline{toc}{section}{\protect\numberline{\thesection}Anhang – IT-Notfall-Checkkarte}
\label{sec:anhang-notfallkarte}

\ifaccessible
  % Accessible: Einfache Struktur ohne minipage/tikzpicture
  \textbf{\Large IT-Notfall-Checkkarte}\\
  \textbf{Praxis für Psychotherapie \PraxisInhaberin}

  \subsection*{STUFE 1: Sofortmaßnahmen}
  \begin{enumerate}
    \item Ruhe bewahren -- Keine Panik
    \item System isolieren -- Netzwerk trennen
    \item Nichts löschen -- Spuren erhalten
    \item IT-Ansprechpartner kontaktieren
  \end{enumerate}

  \subsection*{STUFE 2: Dokumentation}
  Was, Wann, Wie notieren\\
  → Siehe Incident-Management (\cref{sec:notfall})

  \subsection*{STUFE 3: DSGVO-Meldung (72h)}
  Nur bei Risiko für Betroffene:\\
  Patientendaten unbefugt zugänglich/verloren

  Aufsichtsbehörde LfD\\
  \DatenschutzbehoerdeAdresse\\
  \BoxEmpty{} Art der Verletzung beschreiben\\
  \BoxEmpty{} Betroffene Daten/Personen angeben\\
  \BoxEmpty{} Bei hohem Risiko: Betroffene informieren

  \subsection*{Notfallkontakte}
  SOFORT ANRUFEN:\\
  IT: \ITAnsprechpartner\\
  Mobil: \ITAnsprechpartnerTelefon\\
  E-Mail: \href{mailto:\ITAnsprechpartnerMail}{\ITAnsprechpartnerMail}

  Praxis: \PraxisInhaberin\\
  Tel: \PraxisTelefon

  \subsection*{Weitere Kontakte}
  \PVSName{} Support: \PVSSupportTel\\
  SMC-B/eHBA: Herstellerportal

  Diese Karte ausdrucken und griffbereit halten!
\else
  % Standard: Original mit minipage/tikzpicture
  \begin{center}
    \textbf{\Large IT-Notfall-Checkkarte}\\
    \textbf{Praxis für Psychotherapie \PraxisInhaberin}
  \end{center}

  % Linke Spalte: Stufen-Ablauf
  \begin{minipage}[t]{0.48\textwidth}

    \subsection*{STUFE 1: Sofortmaßnahmen}
    \begin{enumerate}
      \item \textbf{Ruhe bewahren} – Keine Panik
      \item \textbf{System isolieren} – Netzwerk trennen
      \item \textbf{Nichts löschen} – Spuren erhalten
      \item \textbf{IT-Ansprechpartner kontaktieren}
    \end{enumerate}

    \subsection*{STUFE 2: Dokumentation}
    \textbf{Was, Wann, Wie notieren}\\
    → Siehe Incident-Management (\cref{sec:notfall})

    \subsection*{STUFE 3: DSGVO-Meldung (72h)}
    \textbf{Nur bei Risiko für Betroffene:}\\
    Patientendaten unbefugt zugänglich/verloren\\[0.2cm]
    \textbf{Aufsichtsbehörde LfD}\\
    \DatenschutzbehoerdeAdresse\\
    \BoxEmpty{} Art der Verletzung beschreiben\\
    \BoxEmpty{} Betroffene Daten/Personen angeben\\
    \BoxEmpty{} Bei hohem Risiko: Betroffene informieren

  \end{minipage}
  \hfill
  % Rechte Spalte: Kontakte & Support
  \begin{minipage}[t]{0.48\textwidth}

    \subsection*{Notfallkontakte}
    \begin{tikzpicture}
      \node[rectangle, draw=EmergencyBoxBorder, fill=EmergencyBoxFill, rounded corners=3pt,
      text width=6cm, align=left, inner sep=0.4cm] {
        \textbf{SOFORT ANRUFEN:}\\[0.1cm]
        \textbf{IT:} \ITAnsprechpartner\\
        \textbf{Mobil:} \ITAnsprechpartnerTelefon\\
        \textbf{E-Mail:} \href{mailto:\ITAnsprechpartnerMail}{\ITAnsprechpartnerMail}\\[0.2cm]
        \textbf{Praxis:} \PraxisInhaberin\\
        \textbf{Tel:} \PraxisTelefon\\[0.2cm]
      };
    \end{tikzpicture}

    \subsection*{Weitere Kontakte}
    \textbf{\PVSName{} Support:} \PVSSupport\\
    \textbf{SMC-B/eHBA:} Herstellerportal

  \end{minipage}

  \begin{center}
  \begin{tikzpicture}
    \node[rectangle, draw=EmergencyBoxBorder, fill=EmergencyBoxFill, rounded corners=5pt,
    text width=8cm, align=center, inner sep=0.5cm] {
      \textbf{Diese Karte ausdrucken und griffbereit halten!}
    };
  \end{tikzpicture}
  \end{center}
\fi
