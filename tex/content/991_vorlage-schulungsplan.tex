\section{Vorlage – IT-Sicherheits-Schulungsplan}
\label{sec:vorlage-schulungsplan}

\Hinweis Template zum handschriftlichen Ausfüllen. IT-Sicherheitsschulungen
gemäß Art. 32 DSGVO jährlich durchführen (quartalsweise angestrebt) und 10 Jahre Aufbewahrung.

\subsection*{Jahresübersicht \FormField[2cm] \hfill Verantwortlich: \PraxisInhaberBezeichnung{}}

\ifaccessible
% Accessible: Einfache Textdarstellung ohne farbige Boxen
\subsubsection*{Q1 (Januar) -- Passwort-Sicherheit}
Dauer: 20 Min | Lernziel: Sichere Passwörter \& 2FA einrichten\\
Inhalt: \PasswortManager{}-Audit, \PVSName{}-Passwort prüfen, 2FA aktivieren\\
Checkliste: \q~\PasswortManager{}-Audit \q~\PVSName{}-Passwort \q~2FA aktiviert\\
Test: Mindestlänge sicherer Passwörter? \FormField[2cm]\\
Durchgeführt am: \FormField[3cm] Unterschrift: \FormField[4cm]

\subsubsection*{Q2 (April) -- Phishing-Erkennung}
Dauer: 25 Min | Lernziel: Phishing-Mails erkennen \& richtig reagieren\\
Inhalt: 5 Merkmale lernen, verdächtige E-Mail identifizieren, S/MIME-Zertifikat prüfen\\
Checkliste: \q~5 Merkmale \q~Verdächtige Mail identifiziert \q~S/MIME geprüft\\
Test: 3 Phishing-Merkmale nennen: \FormField[5cm]\\
Durchgeführt am: \FormField[3cm] Unterschrift: \FormField[4cm]

\subsubsection*{Q3 (Juli) -- Abmelden \& Bildschirmsperre}
Dauer: 15 Min | Lernziel: Arbeitsplatz sicher verlassen\\
Inhalt: Tastenkombinationen üben, automatische Sperre testen, Unterschiede verstehen\\
Checkliste: \q~Tastenkombinationen geübt \q~Auto-Sperre getestet \q~Unterschied verstanden\\
Test: Wann aktiv abmelden? \FormField[5cm]\\
Durchgeführt am: \FormField[3cm] Unterschrift: \FormField[4cm]

\subsubsection*{Q4 (Oktober) -- DSGVO \& Incident Response}
Dauer: 25 Min | Lernziel: Datenschutzvorfälle erkennen \& melden\\
Inhalt: 72h-Meldepflicht verstehen, LfD-Meldeformular ausfüllen, AVV-Register prüfen\\
Checkliste: \q~72h-Regel verstanden \q~Meldeformular ausgefüllt \q~AVV-Register geprüft\\
Test: Zuständige Behörde in \Bundesland? \FormField[4cm]\\
Durchgeführt am: \FormField[3cm] Unterschrift: \FormField[4cm]

\subsubsection*{Jahresbewertung}
Alle Module durchgeführt: \q~Ja \q~Nein (Begründung: \FormField[4cm])\\
Nachschulungen erforderlich: \q~Keine \q~Anzahl: \FormField[1cm]\\
Besondere Vorfälle: \q~Keine \q~Beschreibung: \FormField[4cm]\\
Sicherheitsniveau: \q~Hoch \q~Mittel \q~Verbesserung nötig\\
Datum: \FormField[3cm] Unterschrift \PraxisInhaberBezeichnung{}: \FormField[4cm]

\else
% Standard: Farbige Boxen
% Q1 Card - Blau
\noindent\fcolorbox{blue!50}{blue!20}{\parbox{\dimexpr\textwidth-2\fboxsep-2\fboxrule}{%
\textbf{Q1 (Januar) - Passwort-Sicherheit}\\[0.2cm]
\textbf{Dauer:} 20 Min \textbf{|} \textbf{Lernziel:} Sichere Passwörter \& 2FA einrichten\\
\textbf{Inhalt:} \PasswortManager{}-Audit, \PVSName{}-Passwort prüfen, 2FA aktivieren (Apple-ID)\\
\textbf{Checkliste:} \q~\PasswortManager{}-Audit \q~\PVSName{}-Passwort \q~2FA aktiviert\\
\textbf{Test:} Mindestlänge sicherer Passwörter? \FormField[2cm]\\
\textbf{Durchgeführt am:} \FormField[3cm] \textbf{Unterschrift:} \FormField[4cm]
}}

\vspace{0.3cm}

% Q2 Card - Grün
\noindent\fcolorbox{green!50}{green!20}{\parbox{\dimexpr\textwidth-2\fboxsep-2\fboxrule}{%
\textbf{Q2 (April) - Phishing-Erkennung}\\[0.2cm]
\textbf{Dauer:} 25 Min \textbf{|} \textbf{Lernziel:} Phishing-Mails erkennen \& richtig reagieren\\
\textbf{Inhalt:} 5 Merkmale lernen, verdächtige E-Mail identifizieren, S/MIME-Zertifikat prüfen\\
\textbf{Checkliste:} \q~5 Merkmale \q~Verdächtige Mail identifiziert \q~S/MIME geprüft\\
\textbf{Test:} 3 Phishing-Merkmale nennen: \FormField[5cm]\\
\textbf{Durchgeführt am:} \FormField[3cm] \textbf{Unterschrift:} \FormField[4cm]
}}

\vspace{0.3cm}

% Q3 Card - Orange
\noindent\fcolorbox{orange!50}{orange!20}{\parbox{\dimexpr\textwidth-2\fboxsep-2\fboxrule}{%
\textbf{Q3 (Juli) - Abmelden \& Bildschirmsperre}\\[0.2cm]
\textbf{Dauer:} 15 Min \textbf{|} \textbf{Lernziel:} Arbeitsplatz sicher verlassen\\
\textbf{Inhalt:} Tastenkombinationen üben, automatische Sperre testen, Unterschiede verstehen\\
\textbf{Checkliste:} \q~Tastenkombinationen geübt \q~Auto-Sperre getestet \q~Unterschied verstanden\\
\textbf{Test:} Wann aktiv abmelden? \FormField[5cm]\\
\textbf{Durchgeführt am:} \FormField[3cm] \textbf{Unterschrift:} \FormField[4cm]
}}

\vspace{0.3cm}

% Q4 Card - Rot
\noindent\fcolorbox{red!50}{red!20}{\parbox{\dimexpr\textwidth-2\fboxsep-2\fboxrule}{%
\textbf{Q4 (Oktober) - DSGVO \& Incident Response}\\[0.2cm]
\textbf{Dauer:} 25 Min \textbf{|} \textbf{Lernziel:} Datenschutzvorfälle erkennen \& melden\\
\textbf{Inhalt:} 72h-Meldepflicht verstehen, LfD-Meldeformular ausfüllen, AVV-Register prüfen\\
\textbf{Checkliste:} \q~72h-Regel verstanden \q~Meldeformular ausgefüllt \q~AVV-Register geprüft\\
\textbf{Test:} Zuständige Behörde in \Bundesland? \FormField[4cm]\\
\textbf{Durchgeführt am:} \FormField[3cm] \textbf{Unterschrift:} \FormField[4cm]
}}

\vspace{0.3cm}

\noindent\fcolorbox{gray!50}{gray!20}{\parbox{\dimexpr\textwidth-2\fboxsep-2\fboxrule}{%
\textbf{Jahresbewertung}\\[0.2cm]
\textbf{Alle Module durchgeführt:} \q~Ja \q~Nein (Begründung: \FormField[4cm])\\
\textbf{Nachschulungen erforderlich:} \q~Keine \q~Anzahl: \FormField[1cm]\\
\textbf{Besondere Vorfälle:} \q~Keine \q~Beschreibung: \FormField[4cm]\\
\textbf{Sicherheitsniveau:} \q~Hoch \q~Mittel \q~Verbesserung nötig\\[0.3cm]
\textbf{Datum:} \FormField[3cm] \textbf{Unterschrift \PraxisInhaberBezeichnung{}:} \FormField[4cm]
}}
\fi

\newpage

\subsection*{Grundlagen \& Verantwortlichkeiten}

\textbf{Rechtliche Grundlage:} TOMs Art. 32 DSGVO -- Schulung der Mitarbeitenden\\
\textbf{Durchführung:} \PraxisInhaberBezeichnung{} (Selbstschulung + zukünftige Mitarbeitende)\\
\textbf{Teilnahmepflicht:} \PraxisInhaberBezeichnung{} (Selbstschulung), bei Neueinstellungen alle Mitarbeitenden vor Systemzugang\\
\Dokumentation Teilnahmeblatt + Unterschrift + \enquote{Verstanden}-Checkbox\\
\textbf{Nachschulung:} Bei Verständnisproblemen binnen 2 Wochen\\
\textbf{Review:} Jährlich + bei aktuellen Bedrohungen

\subsection*{Detaillierte Schulungsinhalte}

\ifaccessible\else\begin{samepage}\fi
\ifaccessible
  \EinfacheSchulungsbox
    {Q1: Passwort-Sicherheit}
    {20 Min | Praxisbüro, PC mit Internetzugang | Max. 3 Teilnehmer}
    {Sichere Passwörter gemäß BSI-Empfehlungen (min. 12--16 Zeichen, Einzigartigkeit, Passwortmanager) erstellen und für alle relevanten Dienste (\PVSName{}, Apple-ID, Microsoft 365) Zwei-Faktor-Authentifizierung einrichten}
    {DSGVO Art. 32 (Integrität \& Vertraulichkeit)}
    {BSI \enquote{Sichere Passwörter} (bsi.bund.de) + eigene \PasswortManager{}-Doku}
    {\q~\PasswortManager{}-Audit durchgeführt \q~\PVSName{}: min. \PasswortMindestlaenge{} Zeichen, Bildschirmsperre 5min geprüft \q~2FA bei Apple-ID, Microsoft 365, \PVSName{} aktiviert}
    {\enquote{Wie lang muss ein sicheres Passwort mindestens sein?} Antwort: \FormField[3cm]}
\else
  % Standard: Tabelle
  \ifaccessible\begin{tabular}{p{3.5cm} p{11.0cm}}\else\begin{longtable}{p{3.5cm} p{11.0cm}}\fi
  \toprule
  \AccessibleTableHeader{Q1: Passwort-Sicherheit}{}\\
  \midrule
  \textbf{Dauer \& Ort} & 20 Min | Praxisbüro, PC mit Internetzugang | Max. 3 Teilnehmer \\
  \textbf{Lernziel} & Sichere Passwörter gemäß BSI-Empfehlungen (min. 12--16 Zeichen, Einzigartigkeit, Passwortmanager) erstellen und für alle relevanten Dienste (\PVSName{}, Apple-ID, Microsoft 365) Zwei-Faktor-Authentifizierung einrichten \\
  \textbf{Rechtsbezug} & DSGVO Art. 32 (Integrität \& Vertraulichkeit) \\
  \textbf{Quellen} & BSI \enquote{Sichere Passwörter} (bsi.bund.de) + eigene \PasswortManager{}-Doku \\
  \textbf{Checkliste} & \q~\PasswortManager{}-Audit durchgeführt \q~\PVSName{}: min. \PasswortMindestlaenge{} Zeichen, Bildschirmsperre 5min geprüft \q~2FA bei Apple-ID, Microsoft 365, \PVSName{} aktiviert \\
  \textbf{Mini-Test} & \enquote{Wie lang muss ein sicheres Passwort mindestens sein?} Antwort: \FormField[3cm] \\
  \Label{Dokumentation} & Teilnahmeblatt \\
  \bottomrule
  \ifaccessible\end{tabular}\else\end{longtable}\fi
\fi
\ifaccessible\else\end{samepage}\fi

\ifaccessible\else\begin{samepage}\fi
\ifaccessible\begin{tabular}{p{3.5cm} p{11.0cm}}\else\begin{longtable}{p{3.5cm} p{11.0cm}}\fi

  \toprule
  \AccessibleTableHeader{Q2: Phishing-Erkennung}{}\\
  \midrule

  \textbf{Dauer \& Ort} & 25 Min | Praxisbüro, PC mit E-Mail-Zugang | Max. 3 Teilnehmer \\
  \textbf{Lernziel} & Phishing-Mails anhand von mindestens fünf typischen Merkmalen (z.\,B. Absender, Links, Dringlichkeit, Sprache, Anhänge) zuverlässig erkennen und im Verdachtsfall korrekt reagieren (Nicht-Öffnen, Melden, Löschen) \\
  \textbf{Rechtsbezug} & BSI-Empfehlungen \enquote{Phishing erkennen} + TOMs Awareness-Maßnahmen \\
  \textbf{Quellen} & BSI Newsletter + \KV-Material + Top-3-Bedrohungen: Fake-KV-Mails, Fake-\PVSName{}-Updates, Fake-Patientenanfragen \\
  \textbf{Checkliste} & \q~5 Erkennungsmerkmale Phishing gelernt \q~Verdächtige E-Mail im Postfach identifiziert \q~S/MIME-Zertifikat-Gültigkeit geprüft \\
  \textbf{Mini-Test} & \enquote{Woran erkenne ich eine Phishing-Mail?} (3 Merkmale nennen) Antwort: \FormField[8cm] \\
  \Label{Dokumentation} & Teilnahmeblatt \\
  \bottomrule
\ifaccessible\end{tabular}\else\end{longtable}\fi
\ifaccessible\else\end{samepage}\fi

\ifaccessible\else\begin{samepage}\fi
\ifaccessible\begin{tabular}{p{3.5cm} p{11.0cm}}\else\begin{longtable}{p{3.5cm} p{11.0cm}}\fi

  \toprule
  \AccessibleTableHeader{Q3: Abmelden und Bildschirmsperre}{}\\
  \midrule

  \textbf{Dauer \& Ort} & 15 Min | Arbeitsplatz, alle verwendeten Geräte | Max. 3 Teilnehmer \\
  \textbf{Lernziel} & Nach jeder Arbeitsunterbrechung aktiv abmelden oder Bildschirm sperren und Tastenkombinationen für alle verwendeten Geräte kennen (macOS: Control+Command+Q, iOS: Power-Taste) \\
  \textbf{Rechtsbezug} & KBV Anlage 1 Nr. 19 -- Abmelden nach Aufgabenerfüllung \\
  \textbf{Quellen} & BSI \enquote{Arbeitsplatz absichern} + eigene Gerätedokumentation \\
  \textbf{Checkliste} & \q~Abmelde-Tastenkombinationen an allen Geräten geübt \q~Automatische Bildschirmsperre (5 Min) aktiviert und getestet \q~Unterschied aktives Abmelden vs. Bildschirmsperre verstanden \\
  \textbf{Mini-Test} & \enquote{Wann müssen Sie sich aktiv abmelden?} (2 Situationen nennen) Antwort: \FormField[8cm] \\
  \Label{Dokumentation} & Teilnahmeblatt \\
  \bottomrule
\ifaccessible\end{tabular}\else\end{longtable}\fi
\ifaccessible\else\end{samepage}\fi

\ifaccessible\else\begin{samepage}\fi
\ifaccessible\begin{tabular}{p{3.5cm} p{11.0cm}}\else\begin{longtable}{p{3.5cm} p{11.0cm}}\fi

  \toprule
  \AccessibleTableHeader{Q4: DSGVO \& Incident Response}{}\\
  \midrule

  \textbf{Dauer \& Ort} & 25 Min | Praxisbüro, PC mit Internetzugang | Max. 3 Teilnehmer \\
  \textbf{Lernziel} & Datenschutzvorfälle erkennen, innerhalb von 72 Stunden an die zuständige Aufsichtsbehörde melden und das AVV-Register sowie das VVT auf Aktualität prüfen \\
  \textbf{Rechtsbezug} & DSGVO Art. 33 (Meldung von Verletzungen), Art. 30 (VVT) \\
  \textbf{Quellen} & \Datenschutzbehoerde{} + eigene Incident-Dokumentation \\
  \textbf{Checkliste} & \q~72h-Meldepflicht-Kriterien verstanden \q~LfD-Meldeformular ausgefüllt (Testszenario) \q~AVV-Register auf Vollständigkeit geprüft \\
  \textbf{Mini-Test} & \enquote{Welche Behörde ist für DSGVO-Meldungen in \Bundesland{} zuständig?} Antwort: \FormField[6cm] \\
  \Label{Dokumentation} & Teilnahmeblatt \\
  \bottomrule
\ifaccessible\end{tabular}\else\end{longtable}\fi
\ifaccessible\else\end{samepage}\fi

\subsection*{Zusätzliche Informationen}

\textbf{Quellen regelmäßiger Sicherheitsinformationen:}
\begin{itemize}
  \item BSI Newsletter \enquote{Einfach • Cybersicher} (monatlich, praxisnah)
  \item BSI Technische Sicherheitshinweise (RSS-Feed, fortgeschritten)
  \item \KVLang{} + \KammerLang{} (Rundmails)
\end{itemize}

\textbf{Anlassbezogene Schulungen:}
\begin{itemize}
\item \textbf{Bei Neueinstellung:} Alle 4 Module in ersten 4 Wochen
\item \textbf{Aktuelle Bedrohungen:} BSI-Newsletter → zeitnahe Kurz-Schulung (innerhalb 1 Woche)
\item \textbf{Nach Vorfällen:} Incident-Analyse + verstärkte Schulung betroffener Bereiche
\end{itemize}

\textbf{Dokumentation \& Aufbewahrung:}
Diese Schulungsdokumentation ist Bestandteil der IT-Sicherheitsdokumentation nach § 390 SGB V und wird 10 Jahre aufbewahrt.
