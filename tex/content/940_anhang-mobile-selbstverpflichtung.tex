\section{Anhang – Selbstverpflichtung Mobile Geräte}
\label{sec:anhang-mobile}

\textbf{Geltungsbereich:} Diese Selbstverpflichtung gilt für den Umgang mit allen mobilen
Geräten (iPhone, iPad), die zur Verarbeitung von Praxisdaten eingesetzt werden.

\subsection*{Verpflichtungen}

\textbf{Erstkonfiguration und Gerätesicherheit:}
\begin{itemize}
  \item Ich führe vor Erstnutzung alle verfügbaren Sicherheitsupdates durch
  \item Ich nutze komplexe Entsperrcodes (min. 8-stellig) und biometrische Authentifizierung
  \item Ich aktiviere die automatische Gerätesperre nach max. 5 Minuten Inaktivität
  \item Ich melde mich nach Beendigung der Arbeit aktiv von allen Apps und Diensten ab
  \item Ich sperre das Gerät bei jeder Unterbrechung der Nutzung manuell
  \item Ich aktiviere automatische Betriebssystem- und App-Updates
  \item Ich schütze alle SIM-Karten durch eine PIN und verwahre Super-PIN/PUK sicher
  \item Ich verwende keine Jailbreak- oder Rooting-Verfahren
  \item Ich halte Siri deaktiviert (Datenschutz/Abhörschutz)
\end{itemize}

\textbf{App-Management und Datenübertragung:}
\begin{itemize}
  \item Ich installiere Apps ausschließlich aus dem offiziellen App Store
  \item Ich übertrage Patientendaten nur TLS-verschlüsselt (HTTPS/S-MIME)
  \item Erlaubte Datenübertragung: E-Mails, Kalendertermine (anonymisiert), Kontaktdaten
  \item Ich speichere keine Gesundheitsdaten in nicht-konformen Cloud-Diensten
  \item Ich trenne dienstliche und private Nutzung strikt
\end{itemize}

\textbf{Notfallverfahren und Entsorgung:}
\begin{itemize}
  \item Bei Verlust melde ich den Vorfall unverzüglich und sperre SIM-Karte (\MobilfunkAnbieter{}:
    \tel{\SIMSperrnummer{}})
  \item Ich aktiviere sofort Fernsperre und Fernlöschung
  \item Vor Entsorgung führe ich sicheren Werksreset mit Datenlöschung durch
  \item Ich dokumentiere alle Vorfälle vollständig
\end{itemize}

\praxissignatur

\BestandteilJaehrlich
