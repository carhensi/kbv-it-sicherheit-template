\begin{landscape}
  \enlargethispage{2cm}

  % Alles zusammen etwas nach oben ziehen (Bild + Überschrift + Text)
  \vspace*{-2.5cm}   % vorher -2.0cm → jetzt 0.5cm höher

  % Phantom Section für TOC-Eintrag mit Nummer
  \phantomsection
  \refstepcounter{section}
  \addcontentsline{toc}{section}{\protect\numberline{\thesection}Anhang – Netzplan (Praxisnetz)}
  \label{sec:anhang-netzplan}

  % Bild + Überschrift im TikZ-Rahmen
  \begin{center}
    \begin{tikzpicture}[alt={Überschrift-Overlay für Netzplan-Diagramm}]
      % Bild-Knoten: seitenfüllend, linksbündig
      \node[inner sep=0, anchor=north west] (img) at (0,0) {%
        \includegraphics[
          width=\paperwidth,     % praktisch ganze Seite im Querformat
          height=0.84\textheight,
          keepaspectratio,
          alt={Netzplan der Praxis mit Router, Arbeitsplätzen und Netzwerk-Segmentierung. Zeigt die Trennung zwischen Praxisnetz und Standortnetz über einen dedizierten Router.}
        ]{assets/netzplan.png}%
      };

      % Überschrift oben links auf dem Bild mit Nummer
      \node[
        anchor=north west,
        xshift=-35mm,   % näher an die linke Kante (0mm = direkt am Bildrand)
        yshift=-4mm   % leicht ins Bild nach unten
      ] at (img.north west) {%
        \ifaccessible
          % Accessible: Keine TikZ-Überschrift
        \else
          % Standard: TikZ-Überschrift
          \usekomafont{sectioning}\Large \thesection{} Anhang – Netzplan (Praxisnetz)%
        \fi
      };
    \end{tikzpicture}
  \end{center}

  % Textblock unter dem Bild 0.6cm näher ans Bild heranziehen
  \vspace*{-0.6cm}   % vorher -0.5cm → jetzt 0.6cm näher am Bild

  % Info-Boxen nebeneinander
  \ifaccessible
    % Accessible: Einfache vertikale Struktur ohne minipage
    \textbf{\large \RouterKurz{} Konfiguration}\\
    \begin{small}
      \textbf{Modell:} \RouterModell{} (\RouterName)\\
      \textbf{LAN:} \PraxisSubnetz{}, Router: .1, DHCP: .20–.250\\
      \textbf{WLAN:} \WLANName{} (5GHz), WPA2/3, 23–07h aus\\
      \textbf{Gäste-WLAN:} \WLANGastName, WPA2/3, Mo–Fr 8–18h, Gastgeräte isoliert\\
      \textbf{Sicherheit:} Keine Portfreigaben, UPnP aus, Remote-Admin aus\\
      \textbf{Updates:} Firmware Auto-Update aktiviert
    \end{small}

    \vspace{0.5cm}
    \textbf{\large Netzwerk-Segmentierung}\\
    \begin{small}
      \textbf{Trennung:} \RouterModell{} isoliert Praxisnetz von \StandortTyp{}\\
      \textbf{Verkabelung:} \NetzwerkUplink{}\\
      \textbf{Netz:} Separates Subnetz \PraxisSubnetz{} mit eigenem DHCP\\
      \textbf{Firewall:} NAT/Firewall-Schutz gegen externe Zugriffe\\
      \textbf{Kontrolle:} Vollständige Verwaltungshoheit über Praxisgeräte\\
      \textbf{Physische Sicherheit:} \RouterKurz{} in abgeschlossenem Praxisraum
    \end{small}
  \else
    % Standard: minipage nebeneinander
    \begin{minipage}[t]{0.5\textwidth}
      \textbf{\large \RouterKurz{} Konfiguration}\\
      \begin{small}
        \textbf{Modell:} \RouterModell{} (\RouterName)\\
        \textbf{LAN:} \PraxisSubnetz{}, Router: .1, DHCP: .20–.250\\
        \textbf{WLAN:} \WLANName{} (5GHz), WPA2/3, 23–07h aus\\
        \textbf{Gäste-WLAN:} \WLANGastName, WPA2/3, Mo–Fr 8–18h, Gastgeräte isoliert\\
        \textbf{Sicherheit:} Keine Portfreigaben, UPnP aus, Remote-Admin aus\\
        \textbf{Updates:} Firmware Auto-Update aktiviert
      \end{small}
    \end{minipage}
    \hfill
    \begin{minipage}[t]{0.49\textwidth}
      \textbf{\large Netzwerk-Segmentierung}\\
      \begin{small}
        \textbf{Trennung:} \RouterModell{} isoliert Praxisnetz von \StandortTyp{}\\
        \textbf{Verkabelung:} \NetzwerkUplink{}\\
        \textbf{Netz:} Separates Subnetz \PraxisSubnetz{} mit eigenem DHCP\\
        \textbf{Firewall:} NAT/Firewall-Schutz gegen externe Zugriffe\\
        \textbf{Kontrolle:} Vollständige Verwaltungshoheit über Praxisgeräte\\
        \textbf{Physische Sicherheit:} \RouterKurz{} in abgeschlossenem Praxisraum
      \end{small}
    \end{minipage}
  \fi

\end{landscape}