\section{Telematikinfrastruktur}
\label{sec:ti}

\subsection{TI-Komponenten}

Die Praxis nutzt das von \PVSName{} gehostete TI-Gateway (Managed Service).
Die Verbindung erfolgt verschlüsselt über VPN-Softclient gemäß \PVSName{}-Vorgaben.

\textbf{Vorteile des gehosteten TI-Gateway:}
\begin{itemize}
  \item Keine lokale Hardware erforderlich
  \item Automatische Updates durch \PVSName{}
  \item 24/7-Monitoring durch \PVSName{}
  \item Verschlüsselte VPN-Verbindung
\end{itemize}

Zusätzlich wird ein Kartenterminal am Praxisstandort für Kartenzugriffe genutzt
(Details siehe Geräteliste \cref{sec:geraeteliste}).

\subsection{Sicherheitsmaßnahmen}

Alle Standard-Passwörter der TI-Komponenten wurden geändert. Updates werden zeitnah nach
Verfügbarkeit eingespielt. Die Administrationsdaten werden sicher aufbewahrt, wobei der
Praxis-Zugriff auf alle TI-Komponenten gewährleistet bleibt.

Die SMC-B-Karte (Praxisausweis) bleibt dauerhaft im Terminal gesteckt und wird versiegelt.
Der eHBA (Therapeutenausweis) wird nur bei Bedarf eingesteckt und anschließend sicher verwahrt.

\subsection{KIM – Kommunikation im Medizinwesen}
\label{sec:kim}

Für die sichere Kommunikation mit anderen Leistungserbringern wird KIM (Kommunikation
im Medizinwesen) genutzt. KIM ist der offizielle TI-Dienst für verschlüsselte
Nachrichten zwischen Praxen, Krankenhäusern und anderen TI-Teilnehmern.

\textbf{Nutzung:} KIM wird direkt über das PVS (\PVSName{}) verwendet – nicht als
separater E-Mail-Client. Arztbriefe, Befunde und andere medizinische Dokumente
werden ausschließlich über KIM versendet, nicht per unverschlüsselter E-Mail oder Fax.

\textbf{Sicherheitsmerkmale:}
\begin{itemize}
  \item Ende-zu-Ende-Verschlüsselung
  \item Authentifizierung über TI-Zertifikate
  \item Zustellung nur an verifizierte TI-Teilnehmer
\end{itemize}

\subsection{Außerbetriebnahme und Entsorgung}
\label{sec:ti-ausserbetriebnahme}

Bei Außerbetriebnahme von TI-Komponenten (Kartenterminal, Konnektor):
\begin{enumerate}
  \item SMC-B und gSMC-KT aus Terminal entfernen
  \item Karten über Herstellerportal (D-Trust, medisign oder T-Systems) sperren
  \item Terminal/Konnektor auf Werkseinstellungen zurücksetzen
  \item Geräte an Lieferanten zurückgeben oder zertifiziert entsorgen
  \item PIN/PUK-Dokumentation datenschutzkonform vernichten
\end{enumerate}

\textbf{Bei Verlust oder Diebstahl von SMC-B/eHBA:}
Sofortige Sperrung über Herstellerportal, ersatzweise über \KV{} (Arztregister).
Anschließend Incident-Response-Prozess gemäß \cref{sec:notfall} einleiten.
