\section{IT-Sicherheitskonzept}
\label{sec:it-sicherheit}

\subsection{Praxisdaten und Verantwortlichkeiten}

\begin{tabular}{p{5cm} p{7cm}}
  \toprule
  \textbf{Praxis} & {\def\\{\newline}\PraxisNameMitUmbruch} \\
  & \PraxisAdresse \\
  & \PraxisOrt \\
  & Tel.: \PraxisTelefon \\[0.5em]

  \textbf{\PraxisInhaberBezeichnung{}} & \PraxisInhaberin \\
  & \ITAnsprechpartnerAdresse \\
  & Tel.: \PraxisInhaberinPrivatTelefon \\[0.5em]

  \textbf{IT-Ansprechpartner} & \ITAnsprechpartner \\
  & Tel.: \ITAnsprechpartnerTelefon \\[0.5em]

  \textbf{Passwortmanagement} & \PasswortManager{} (Notfallzugang geregelt) \\
  \bottomrule
\end{tabular}

\textbf{\PasswortManager{} Notfallzugang:} Emergency Kit sicher verwahrt: \NotfallkitVerwahrung.
Physischer Zugriff ist sowohl für die \PraxisInhaberBezeichnung{} als auch den IT-Ansprechpartner
jederzeit eigenständig möglich.
\NotfallVollmachtPerson{} ist als Emergency Contact hinterlegt und
kann nach einer Wartezeit von einer Woche Zugriff auf den Praxis-Vault anfordern.

\subsection{Personal}
\label{sec:personal}

\textbf{Mitarbeitende:} In der Praxis sind keine Mitarbeitenden beschäftigt. Die
\PraxisInhaberBezeichnung{} führt alle Tätigkeiten eigenverantwortlich durch.

\textbf{Einstellungsverfahren:} Bei zukünftigen Einstellungen wird besonders auf
Vertrauenswürdigkeit geachtet, einschließlich der Prüfung vorliegender Arbeitszeugnisse
und Referenzen. Alle Angaben werden auf Glaubhaftigkeit kontrolliert.

\textbf{Einarbeitung:} Bei zukünftigen Einstellungen erfolgt systematische Einarbeitung
neuer Mitarbeitender nach strukturierter Checkliste (siehe \cref{sec:vorlage-einarbeitung}).
Dies umfasst IT-Sicherheitsschulung, Kontenerstellung, Arbeitsplatzeinweisung und
Verständnisprüfung aller Sicherheitsrichtlinien.

\textbf{Weggang von Mitarbeitenden:} Strukturiertes Offboarding nach Checkliste
(siehe \cref{sec:vorlage-offboarding}) mit vollständiger Rückgabe aller Unterlagen,
Deaktivierung von Zugängen und Betonung fortdauernder Verschwiegenheitspflichten.

\textbf{Externe Dienstleister:} IT-Support erfolgt durch \PVSHersteller{} (PVS-Hersteller) und IT-Ansprechpartner \ITAnsprechpartner. Beide sind über Verschwiegenheitserklärungen für externe Dienstleister verpflichtet (siehe \cref{sec:anhang-verschwiegenheit}). Zugangsberechtigungen werden restriktiv vergeben und bei Beendigung der Zusammenarbeit entzogen.

\textbf{Vertrauenswürdigkeit:} Die Auswahl externer Dienstleister erfolgt nach
Vertrauenswürdigkeit und fachlicher Qualifikation. Alle externen Zugriffe erfolgen nur
nach vorheriger Terminabsprache und unter Aufsicht.

\textbf{Beaufsichtigung:} Externes Personal (z.\,B.\ IT-Techniker, Wartungspersonal) wird
in sicherheitsrelevanten Bereichen kontinuierlich beaufsichtigt. Zugangsberechtigungen
werden so restriktiv wie möglich gehalten und nach Arbeitsende sofort entzogen.

\subsection{Sensibilisierung und Schulung}
\label{sec:sensibilisierung}

\textbf{Strukturierte Selbstschulung:} Die \PraxisInhaberBezeichnung{} hält sich über IT-Sicherheit
durch dokumentierte Weiterbildung auf dem Laufenden:
\begin{itemize}
  \item KV-Rundschreiben und -Newsletter
  \item \PVSName{}-Newsletter
  \item BSI für Bürger: IT-Sicherheit im Homeoffice und Praxis
  \item BSI-Grundschutz-Kompendium: Bausteine für kleine Unternehmen
  \item KBV-Praxisnachrichten: Aktuelle IT-Sicherheitswarnungen und -empfehlungen
  \item Fachspezifische Fortbildungen zu IT-Sicherheit und Datenschutz
\end{itemize}

\Dokumentation Selbstschulungsmaßnahmen werden im Kalender mit Datum, Quelle und
Thema dokumentiert und sind dadurch nachvollziehbar. Jährlich wird mindestens eine
strukturierte Schulung zu IT-Sicherheit durchgeführt (mindestens 4 Stunden gemäß §390 SGB V);
ergänzend werden quartalsweise Kurzmodule angestrebt (siehe \cref{sec:vorlage-schulungsplan}).

\textbf{KBV-Ressourcen und Fortbildungen:}
\begin{itemize}
  \item \textbf{IT-Sicherheit Übersicht:} \\
    \url{https://www.kbv.de/praxis/digitalisierung/it-sicherheit}
  \item \textbf{KBV-Hub (Musterdokumente, FAQs):} \\
    \url{https://hub.kbv.de/display/itsrl}
  \item \textbf{Fortbildungsportal:} \\
    \url{https://www.kbv.de/praxis/tools-und-services/fortbildungsportal}
  \item Basis-Schulung für MFA (PDF im KBV-Hub)
  \item Phishing-Schulung für MFA (PDF im KBV-Hub)
  \item IT-Sicherheit in der Praxis für Niedergelassene (PDF im KBV-Hub)
  \item Zertifizierte Online-Fortbildungen (bis zu 6 Fortbildungspunkte, kostenfrei)
\end{itemize}

\textbf{Sensibilisierung:} Regelmäßige Überprüfung der Sicherheitsmaßnahmen durch
Eigenprüfung (siehe \cref{sec:vorlage-eigenpruefung}). Bei Sicherheitsvorfällen
erfolgt unverzügliche Meldung an den IT-Ansprechpartner.

\subsection{Netzwerksicherheit}
\label{sec:netzwerk}

\textbf{Router und Firewall:} \RouterModell{} mit sicherer Grundkonfiguration (Details
siehe Netzplan~\ref{sec:anhang-netzplan}). Automatische Firmware-Updates, geänderte
Standard-Passwörter, keine Portfreigaben.

\textbf{Firewall-Begründung (Anlage~1 Nr.~11):} Die \RouterModell{} erfüllt als
Stateful-Packet-Inspection-Firewall die KBV-Anforderungen zum Netzübergangsschutz.

Die \RouterModell{} bietet:
\begin{itemize}
  \item Stateful Packet Inspection (SPI)
  \item NAT als zusätzliche Schutzschicht
  \item Automatische Firmware-Updates
  \item Keine Portfreigaben konfiguriert
  \item UPnP deaktiviert
\end{itemize}

\textbf{Firewall-Architektur (Anlage 1 Nr. 11):} Die Praxis implementiert eine
risikobasierte Firewall-Strategie. Der \RouterModell{} mit Stateful Packet Inspection
erfüllt die KBV-Anforderungen vollständig:

\begin{itemize}
  \item Stateful Packet Inspection blockiert unerwünschte Verbindungen
  \item NAT bietet zusätzliche Schutzschicht durch IP-Maskierung
  \item Extern gehostetes TI-Gateway eliminiert lokale Angriffsfläche (siehe \cref{sec:ti})
  \item Automatische Firmware-Updates gewährleisten aktuellen Schutzstand
\end{itemize}

\textbf{Compliance-Bewertung:} Die \KV-FAQ empfiehlt Hardware-Firewall oder
``Einsatz des Konnektors im Reihenbetrieb''. Da das TI-Gateway extern gehostet
ist, entfällt die Reihenschaltung. Die gewählte Architektur entspricht dem
Bedrohungsmodell und übertrifft die Mindestanforderungen für Einzelpraxen
ohne exponierte Dienste. Jährliche Neubewertung erfolgt im Rahmen der
IT-Sicherheitsevaluation.

\textbf{Netzmanagement-Authentisierung:} Der Management-Zugriff auf Netz\-kom\-po\-nen\-ten (\RouterKurz{}) erfolgt ausschließlich über sichere Authentisierung mit starken, individuellen Passwörtern. Standard-Zugangsdaten wurden geändert und werden sicher in \PasswortManager{} verwaltet.

\textbf{WLAN-Sicherheit:} WPA2\slash WPA3-Ver\-schlüs\-se\-lung, automatische Abschaltung nachts
(\WLANAbschaltung).

\textbf{Gäste-WLAN:} Separater WLAN-Gastzugang \enquote{\WLANGastName} für Patientinnen mit
WPA2\slash WPA3-Ver\-schlüs\-se\-lung und starkem Passwort. Vollständige Netz\-werk\-tren\-nung vom
Praxisnetz, Gastgeräte sind isoliert voneinander. Automatische Abschaltung nachts
(\WLANAbschaltung).

\textbf{Logging und Überwachung (freiwillig, Anlage~2 Nr.~1+2):} Systemprotokoll
aktiviert. Protokolle (\RouterKurz{}, macOS, \PVSName{}) werden anlassbezogen (z.\,B.\ bei
Sicherheitsvorfällen) ausgewertet. Im Rahmen der jährlichen Eigenprüfung erfolgt
zusätzlich eine stichprobenartige Sichtung zur Verifizierung.

\textbf{Monitoring-Strategie (Anlage 3):} Die Praxis implementiert eine
praxisgerechte Überwachungsarchitektur statt zentralem SIEM:

\begin{itemize}
  \item Dezentrale Logs (Router, macOS, PVS) für gezielte Analyse
  \item Honeypots (Canary Tokens) für Anomalie-Erkennung (siehe \cref{sec:honeypots})
  \item Quartalsweise Stichproben-Auswertung in TOMs-Prüfung
  \item Ereignisbasierte Tiefenanalyse bei Sicherheitsvorfällen
\end{itemize}

\textbf{Architektur-Vorteil:} Diese Lösung bietet effektive Überwachung ohne
SIEM-Overhead und entspricht der Einzelpraxis-Infrastruktur optimal.
SIEM-Systeme erfordern kontinuierliche Analyse-Kapazität und spezialisierte
Expertise, die in Einzelpraxen nicht wirtschaftlich darstellbar ist.
Bei Wachstum wird diese Strategie neu bewertet.

\subsection{Patch- und Änderungsmanagement}
\label{sec:patch}

\textbf{Verantwortlichkeiten:} Die \PraxisInhaberBezeichnung{} ist für die zeitnahe Installation von
Updates verantwortlich. Der IT-Ansprechpartner unterstützt bei kritischen Updates und Problemen.

\textbf{Update-Verfahren:}
\begin{itemize}
  \item \textbf{macOS:} Automatische kritische Updates und Minor-Updates aktiviert, Major-Updates nur in Absprache mit \PVSName{} und \PraxisInhaberBezeichnung{}
  \item \textbf{\PVSName{} PVS:} Automatische Updates außerhalb der Praxiszeiten konfiguriert, manuelle Updates nach Herstellerfreigabe binnen 14 Tagen. Automatische Backup-Erstellung vor jedem Update durch Server-Tools
  \item \textbf{Microsoft 365:} Automatische Updates aktiviert
  \item \textbf{\RouterKurz{}:} Automatische Firmware-Updates aktiviert
  \item \textbf{iOS/iPadOS:} Automatische Sicherheitsupdates aktiviert
\end{itemize}

\textbf{Identifizierung ausbleibender Updates:}
\begin{itemize}
  \item \textbf{macOS/iOS:} Systemeinstellungen → Softwareupdate (wöchentliche Kontrolle)
  \item \textbf{\MDMTool{}:} MDM-Dashboard zeigt Update-Status aller verwalteten Geräte
  \item \textbf{\PVSName{}:} Automatische Update-Benachrichtigungen im System
  \item \textbf{\RouterKurz{}:} Admin-Interface zeigt verfügbare Firmware-Updates
\end{itemize}

\textbf{End-of-Life-Strategie:} Geräte ohne verfügbare Sicherheitsupdates (EOL) werden
rechtzeitig durch aktuelle Modelle ersetzt oder durch zusätzliche Sicherheitsmaßnahmen
kompensiert. Austausch spätestens bei Hersteller-EOL oder ausbleibenden
Sicherheitsupdates; planmäßige Bewertung und ggf. Ersatz bis zur nächsten
Dokumentüberprüfung (\NextReview). Die Bewertung erfolgt im Rahmen der jährlichen
IT-Sicherheitsevaluation (siehe \cref{sec:vorlage-eigenpruefung}).

\Dokumentation Bei Problemen erfolgt eine Dokumentation und Wiederherstellung
über Backup-Restore.

\subsection{Endgeräte-Sicherheit}
\label{sec:endgeraete}

\textbf{Grundschutz:} Alle Endgeräte sind mit aktuellen Betriebssystemen und Virenschutz
(macOS XProtect) ausgestattet. Automatische Bildschirmsperre nach 5 Minuten Inaktivität.

\textbf{Integrierte Sicherheitsarchitektur (Anlage 1 Nr. 20):} Die Praxis nutzt
eine Defense-in-Depth-Strategie basierend auf dem macOS-Sicherheitsstack:

\begin{itemize}
  \item XProtect: Tägliche Malware-Signaturen von Apple
  \item Gatekeeper: Automatische Code-Signatur-Validierung
  \item SIP: Hardware-basierter Systemschutz
  \item Sandboxing: Anwendungsisolierung auf Kernel-Ebene
  \item Secure Boot: Firmware-Integritätsprüfung
\end{itemize}

\textbf{Technische Überlegenheit:} Diese integrierte Lösung übertrifft herkömmliche
Virenschutz-Software durch tiefe Systemintegration ohne Performance-Einbußen oder
Kompatibilitätskonflikte. Apple dokumentiert im Platform Security Guide die
Ausreichend dieser Architektur für Enterprise-Umgebungen. Dritt-Anbieter-Software
würde Kernel-Zugriff erfordern und potenzielle Angriffsfläche schaffen.

\textbf{Compliance-Nachweis:} Der Apple Platform Security Guide bestätigt, dass
\enquote{das System so konzipiert ist, dass zusätzliche Antivirus-Software nicht
erforderlich ist} (Quelle: \url{https://support.apple.com/guide/security/}).
Diese Architektur erfüllt die KBV-Anforderungen vollständig.

\textbf{macOS Security-Baseline (CIS Level~1):} Die Konfiguration orientiert sich am
CIS Apple macOS Benchmark (Level~1). Umgesetzte Maßnahmen: Festplattenverschlüsselung
(FileVault) mit sicher hinterlegtem Recovery-Key, macOS-Firewall aktiviert, Gatekeeper
und XProtect für Malware-Schutz, System Integrity Protection (SIP) aktiviert, Secure Boot
auf \enquote{Full Security} (Apple Silicon), USB-Zubehör erfordert explizite Freigabe,
alle Sharing-Dienste deaktiviert, Gastaccount deaktiviert, separater Admin-Account für
Systemverwaltung, automatisches Login deaktiviert, Passwort sofort nach Bildschirmschoner
erforderlich, Safari Auto-Open Downloads deaktiviert, Remote Management deaktiviert,
Standortdienste nur für \enquote{Wo ist?} (Fernlöschung bei Verlust).
Halbjährliche Überprüfung der Baseline-Konfiguration in den TOMs-Prüfungen.

\textbf{Zugangsschutz:} Starke Passwörter/Biometrie für alle Benutzerkonten. Keine
Gastkonten oder geteilte Accounts. Separater lokaler Admin-Account (\texttt{\AdminUser{}}) für
Systemverwaltung, tägliche Arbeit mit Standard-Benutzerrechten.

\textbf{Sperr-/Abmeldepflicht:} Nach Ende der Nutzung sofort Bildschirm sperren/abmelden
(macOS: Ctrl+Cmd+Q). Auto-Sperre bleibt Sicherheitsnetz (5 Min).

\textbf{Gruppenberechtigungen:} Datei- und Freigabeberechtigungen sind pro Personengruppe
(Praxisinhaber, externe Dienstleister) und pro Person individuell geregelt. Zugriffe
folgen dem Need-to-know-Prinzip.

\textbf{Verschlüsselung:} FileVault-Festplattenverschlüsselung und TLS-Datenübertragung
aktiviert. Details zu Wechseldatenträgern siehe \cref{sec:wechseldatentraeger}.

\textbf{Software-Beschränkung:} Installation nur aus vertrauenswürdigen Quellen (Mac App
Store, Hersteller-Websites). Keine P2P-Software oder unsichere Downloads.
Sicherheitsfeatures siehe macOS Security-Baseline oben.

\textbf{Mikrofon/Kamera-Schutz (Anlage~1 Nr.~18):} Mikrofon und Kamera sind in den
macOS-Systemeinstellungen grundsätzlich für alle Anwendungen deaktiviert. Aktivierung nur
bei konkretem Bedarf (z.\,B.\ Videosprechstunde) und anschließende Deaktivierung. Keine
permanenten Berechtigungen für nicht-medizinische Apps.

\textbf{Drucker-Sicherheit:} Brother DCP-L2660DW Laserdrucker mit folgenden Sicherheitsmaßnahmen:
\begin{itemize}
  \item \textbf{Standortschutz}: Drucker steht direkt neben Arbeitsplatz (physische Kontrolle)
  \item \textbf{Cloud-Print deaktiviert}: Keine Verbindung zu externen Print-Diensten
  \item \textbf{WLAN-Sicherheit}: WPA2\slash WPA3-ver\-schlüs\-sel\-te Verbindung zum Praxis-WLAN
  \item \textbf{Admin-PIN}: Gerätezugriff durch PIN geschützt
  \item \textbf{Kein PIN-Druck}: Nicht erforderlich (Drucker direkt am Arbeitsplatz $<$2m,
    Einzelpraxis, Patienten nicht allein im Raum, sofortige Entnahme, Druck nur bei
    Anwesenheit im Praxis-WLAN möglich)
  \item \textbf{Speicher}: Druckaufträge werden nicht dauerhaft im Gerät gespeichert
\end{itemize}

\subsection{Windows-Endgeräte}
\label{sec:windows}

\textbf{Nicht anwendbar:} In der Praxis werden ausschließlich Apple-Geräte eingesetzt.
Keine Windows-Systeme vorhanden (auch nicht virtualisiert).

\subsection{Internet- und Cloud-Anwendungen}
\label{sec:internet}

\textbf{Cloud-Nutzung:}
\begin{itemize}
  \item \textbf{iCloud Backup:} Verschlüsselte Backups für iOS/macOS-Geräte über Apple Business Manager (MDM-verwaltet, AVV vorhanden). Keine Gesundheitsdaten in iCloud Drive oder iCloud Mail.
  \item \textbf{Microsoft 365:} Nur lokale Office-Apps (Word, Excel), keine Cloud-Speicherung
  \item \textbf{Keine weiteren Cloud-Dienste} für Patientendaten
\end{itemize}

\textbf{Authentisierung:} Passwort-Richtlinie:
\begin{itemize}
  \item Mindestlänge: \PasswortMindestlaenge{} Zeichen (\PasswortManager{}-generiert)
  \item Keine Wiederverwendung
  \item 2FA wo möglich (iCloud, \PasswortManager{}, \HostingAnbieter{})
  \item Passwort-Rotation: Nur bei Verdacht auf Kompromittierung
  \item \PasswortManager{} Sicherheits-Dashboard Score: Ziel $\geq$\PasswortSecurityScore{} (quartalsweise Prüfung in TOMs, Abweichungen werden dokumentiert und bewertet)
\end{itemize}

\textbf{Datenschutz:} Patientendaten werden nicht in öffentlichen Cloud-Diensten gespeichert.
Gesundheits- und Sozialdaten werden nicht in Cloud-Speichern (iCloud Drive, OneDrive, etc.) abgelegt.

Kalender wird über \EmailAnbieter{} oder lokal verwaltet. Bei Kalender-Synchronisation werden
ausschließlich anonymisierte Patientenkürzel für die Terminorganisation verwendet.
Dabei werden keine Namen, Diagnosen oder medizinischen Inhalte übertragen.
Die Verarbeitung ist zweckgebunden und DSGVO-konform dokumentiert.

\textbf{Cloud-Compliance:} Soweit Sozial- oder Gesundheitsdaten im Wege des
Cloud-Computing verarbeitet werden, verfügt der Anbieter über ein aktuelles C5-Testat
entsprechend § 393 SGB V in Verbindung mit § 384 SGB V (\KIToolName{}, \PVSName{} TI-Gateway).
Alle Cloud-Dienste werden im Verzeichnis von Verarbeitungstätigkeiten (VVT, \cref{sec:vvt}) dokumentiert.

\subsection{Datensicherung}
\label{sec:datensicherung}

\textbf{\PVSName{}-Backup-Konzept:} \PVSName{} läuft lokal als Server und Client auf dem MacBook Air
(siehe Geräteliste \cref{sec:geraeteliste}). \PVSName{} erstellt automatisch primäre Backups aller
Patientendaten und Systemkonfigurationen in einem lokalen Backup-Verzeichnis auf der
System-SSD (FileVault-verschlüsselt).
Diese primären Backups werden anschließend über Time Machine auf verschlüsselte externe
SSD und verschlüsseltes NAS-Backup gesichert. Bei MacBook-Ausfall erfolgt Wiederherstellung
aus Backup gemäß RTO/RPO (siehe unten).

Das Backup-Konzept umfasst lokale Time Machine-Sicherung auf externe Festplatte sowie
Offsite-Backup auf NAS. Wöchentliche Vollsicherung, tägliche inkrementelle Sicherung.
Restore-Tests werden mindestens einmal jährlich, angestrebt halbjährlich (Januar/Juli)
sowie anlassbezogen durchgeführt und dokumentiert.

\textbf{Wiederanlaufziele:}
\begin{itemize}
  \item \textbf{RTO (Recovery Time):} $\leq$4h ab Bereitstellung der Ersatzhardware
    (Neugerät ggf.\ kurzfristig im Handel erhältlich; Gesamtwiederherstellung
    inkl.\ Hardwarebeschaffung im Worst Case bis zu 24h)
  \item \textbf{RPO (Recovery Point):} $\leq$24h (konservativ gerechnet)
\end{itemize}

\textbf{Verantwortlichkeit:} Die \PraxisInhaberBezeichnung{} ist für die ordnungsgemäße Durchführung
der Datensicherung, tägliche Backup-Kontrolle, Restore-Tests und Meldung bei Problemen verantwortlich.
Der IT-Ansprechpartner ist für die technische Konfiguration zuständig und
steht bei Fragen zur Verfügung.

\textbf{Tägliche Backup-Kontrolle:} Die \PraxisInhaberBezeichnung{} prüft täglich den Time Machine-Status
(Letztes Backup < 24h). \PVSName{} meldet automatisch bei nicht erfolgreichem Backup.
Bei Fehlern wird der IT-Ansprechpartner unverzüglich informiert.

Das Offsite-Backup wird verschlüsselt auf einem NAS am \BackupStandort{}
gespeichert. Durch die Verschlüsselung ist ein Zugriff auf Praxisdaten durch andere
Dritte ausgeschlossen. Auch NAS-Administratoren haben keinen Zugriff auf
die verschlüsselten Time Machine-Container ohne die Backup-Schlüssel. Die
Datenübertragung erfolgt lokal oder verschlüsselt per IPSec-VPN (\RouterKurz{} zu NAS).
VPN-Konfiguration und -Betrieb erfolgen durch den IT-Ansprechpartner.

Die Backup-Schlüssel für die verschlüsselten Time Machine-Container werden
ausschließlich von der \PraxisInhaberBezeichnung{} verwaltet und sicher in \PasswortManager{}
gespeichert. Das NAS steht gesichert am \BackupStandort{}.

\textbf{Backup-Aufbewahrung und Ausdünnung:} Time Machine dünnt Backups automatisch aus:
Stündliche Backups der letzten 24 Stunden, tägliche des letzten Monats, danach
wöchentliche. Bei Platzmangel werden die ältesten Backups automatisch gelöscht.
Gelöschte Patientendaten können daher in verschlüsselten Backups verbleiben, bis diese
turnusmäßig überschrieben werden. Dies ist datenschutzrechtlich zulässig, da die Backups
verschlüsselt, nicht aktiv zugänglich und ausschließlich für die Notfallwiederherstellung
bestimmt sind.

\subsection{Sichere Datenträgerentsorgung}
\label{sec:entsorgung}

Vor Entsorgung, Verkauf oder Weitergabe von IT-Geräten erfolgt sichere Löschung aller
Daten durch Überschreibung oder physische Zerstörung der Datenträger. Bei SSDs wird die
Herstellersoftware für Secure Erase verwendet. Festplatten werden mindestens dreifach
überschrieben oder mechanisch zerstört.

\Dokumentation Jede Entsorgung wird mit Gerät, Datum, Löschmethode und
verantwortlicher Person in der Geräteliste (siehe \cref{sec:geraeteliste})
dokumentiert.

\subsection{Aufbewahrungsplan für nicht-medizinische Daten}
\label{sec:aufbewahrungsplan}

Ergänzend zu den medizinischen Aufbewahrungspflichten (§ 630f BGB - 10 Jahre)
gelten für organisatorische und technische Daten folgende Aufbewahrungsfristen:

\begin{center}
\begin{tabular}{p{5cm}p{2.5cm}p{6cm}}
\toprule
\textbf{Datenart} & \textbf{Aufbewahrung} & \textbf{Rechtsgrundlage/Zweck} \\
\midrule
System-/Sicherheitslogs & 90 Tage & IT-Betrieb, Incident-Analyse \\
Organisatorische E-Mails & 12 Monate & Geschäftskorrespondenz \\
Patientenbezogene E-Mails & Sofort in PVS & Siehe \cref{sec:kommunikation} \\
Support-/Änderungsprotokolle & 24 Monate & Nachvollziehbarkeit, Wartung \\
Schulungsnachweise & 36 Monate & DSGVO-Rechenschaftspflicht \\
AVV/Vertragsunterlagen & Laufzeit + 3 Jahre & Verjährungsfristen \\
Website-Logs (\HostingAnbieter{}) & 30 Tage & Automatische Löschung \\
Backup-Daten & Entsprechend Ursprungsdaten & Zweckbindung \\
\bottomrule
\end{tabular}
\end{center}

\textbf{Löschprozess:} Automatische Löschung wo technisch möglich (\HostingAnbieter{}-Logs),
manuelle Löschung nach Fristablauf mit Dokumentation. Ausnahmen nur bei
rechtlichen Aufbewahrungspflichten oder laufenden Verfahren.

\textbf{Verweis:} Detaillierte Aufbewahrungsfristen siehe VVT (\cref{sec:vvt}).

\subsection{Patientenaufnahme und Datenschutz-Information}
\label{sec:patientenaufnahme}

\subsubsection{Datenschutz-Information}

Die Datenschutz-Information nach Art.\ 13 DSGVO ist in der Praxis ausgehängt und online
unter \url{\PraxisWebsiteDatenschutz} abrufbar. Änderungen werden
versioniert dokumentiert und Patienten bei wesentlichen Änderungen informiert.

\subsubsection{Digitaler Aufnahmeprozess}

Die Patientenaufnahme erfolgt vollständig digital im PVS mit minimalen, kompakten Formularen:
\begin{enumerate}
  \item Datenschutz-Information (Aushändigung/Online-Verweis)
  \item KI-Einwilligung (bei gewünschter KI-Dokumentation)
  \item Behandlungsvertrag und Anamnesebogen
\end{enumerate}

Alle Einwilligungen werden direkt im PVS dokumentiert und versioniert gespeichert.
Der Widerruf von Einwilligungen ist jederzeit mündlich oder schriftlich möglich
und wird sofort wirksam. Widerrufe werden im PVS dokumentiert.

\subsubsection{KI-gestützte Dokumentation}

\textbf{\KIToolName{} Integration:} Die Praxis nutzt \KIToolName{} für die
KI-gestützte Erstellung von Sitzungsnotizen und psychologischen Berichten. \KIToolName{}
ist C5-zertifiziert und DSGVO-konform (Details: \url{\KIToolSicherheitURL}).

\textbf{Patienteneinwilligung:} Vor der ersten KI-gestützten Sitzung wird eine explizite
schriftliche Einwilligung nach Art.\ 9 Abs.\ 2 lit.\ a DSGVO eingeholt. Die Einwilligung
erfolgt über die in \KIToolName{} integrierte, anpassbare Vorlage und wird im PVS dokumentiert.

\textbf{Datenverarbeitung:} Audioaufzeichnung und Transkription erfolgen temporär zur
Erstellung der Sitzungsnotizen. \KIToolName{} speichert keine Patientendaten dauerhaft.
Alle Daten werden nach der Verarbeitung automatisch gelöscht. Die generierten Protokolle
werden ausschließlich im lokalen PVS (\PVSName{}) gespeichert.

\textbf{Widerruf:} Patienten können die Einwilligung jederzeit ohne Angabe von Gründen
widerrufen. Bei Widerruf erfolgt die Dokumentation wieder manuell.

\subsubsection{Schweigepflichtentbindungen}

Für Berichte an andere Behandler oder Institutionen werden bei Bedarf spezifische
Schweigepflichtentbindungen eingeholt. Diese werden zweckgebunden, zeitlich begrenzt
und dokumentiert im PVS verwaltet. Auskünfte an Krankenkassen erfolgen nur im
gesetzlich vorgesehenen Rahmen (Antragsverfahren, PTV-Formulare).

\subsubsection{Aufbewahrung und Löschung}

\textbf{Aufbewahrungsfristen:} Patientendaten werden gemäß § 630f BGB für 10 Jahre nach
Behandlungsende aufbewahrt. KI-generierte Sitzungsnotizen unterliegen denselben Fristen
wie manuell erstellte Dokumentation.

\textbf{Löschkonzept:} Nach Ablauf der Aufbewahrungsfristen erfolgt sichere Löschung
aller Patientendaten aus dem PVS. In verschlüsselten Backups können gelöschte Daten
länger verbleiben (siehe \cref{sec:datensicherung}). Bei einer Wiederherstellung aus
Backup wird geprüft, ob zwischenzeitlich gelöschte Daten erneut zu löschen sind.

\subsubsection{Betroffenenrechte}

Patienten haben das Recht auf Auskunft, Berichtigung, Löschung und Datenübertragbarkeit
ihrer Daten. Anfragen werden binnen 30 Tagen bearbeitet. Der Prozess ist in der
Vorlage Betroffenenrechte (\cref{sec:vorlage-betroffenenrechte}) dokumentiert.

\textbf{Zusammenfassung:} Das IT-Sicherheitskonzept etabliert die technischen und
organisatorischen Grundlagen für den sicheren Praxisbetrieb. Aufbauend auf dieser
Infrastruktur behandelt das folgende Kapitel die sichere Kommunikation mit Patienten
und externen Partnern.
