\section{Vorlage – SOP Remote-Support (unter Aufsicht)}
\label{sec:vorlage-remote-support}

\Hinweis Template zur sicheren Durchführung von Fernwartung durch externe
Dienstleister. 10 Jahre Aufbewahrung.

%% Präambel - definiere nur wenn nicht schon vorhanden
\providecommand{\qsize}{1.05ex}
\renewcommand{\qsize}{1.05ex}
\providecommand{\q}{\fbox{\rule{0pt}{\qsize}\rule{\qsize}{0pt}}}
\renewcommand{\q}{\fbox{\rule{0pt}{\qsize}\rule{\qsize}{0pt}}}

\textbf{Zweck \& Geltung:} Diese SOP regelt Fernwartung durch externe Dienstleister für Praxis-IT (z.\,B. PVS,
Netzwerk, Endgeräte). Ziel ist eine nachvollziehbare, sichere Durchführung unter
Aufsicht der Praxisinhaberin bei minimalen Rechten und lückenloser Protokollierung.

\textbf{Voraussetzungen:}
\begin{itemize}
  \item Zugelassen sind nur benannte Dienstleister lt. AVV/Vertrag; Identität wird vor jeder Sitzung geprüft.
  \item Kein dauerhafter Fernzugang. Sitzungen sind \emph{terminiert}, \emph{zeitlich begrenzt} und \emph{einmalig freizugeben}.
  \item Nutzung eines \emph{temporären, eingeschränkten} Kontos; \emph{kein} Einsatz persönlicher Konten der Dienstleister.
  \item Bildschirmfreigabe \emph{unter Aufsicht}; Dateiübertragungen nur nach expliziter Zustimmung.
  \item Aktive Schutzfunktionen (AV, Firewall, FileVault) bleiben eingeschaltet; Protokollierung ist aktiviert.
\end{itemize}

\textbf{Ablauf:}

\textbf{Vorher:}
\begin{itemize}
  \item[\q] Ticket/Anlass definieren (Fehlerbild, Ziel, betroffene Systeme)
  \item[\q] Temporäres Konto anlegen, \emph{Least-Privilege} (nur benötigte Rechte)
  \item[\q] Zeitfenster vereinbaren; Nutzer informiert; Backups aktuell
\end{itemize}

\textbf{Während:}
\begin{itemize}
  \item[\q] Sitzung unter Aufsicht der Praxisinhaberin/IT-Ansprechpartner:in
  \item[\q] Änderungsliste mitschreiben (Komponenten, Befehle, Versionen)
  \item[\q] Keine Persistenz hinterlassen (keine dauerhaften Dienste/Tunnel)
\end{itemize}

\textbf{Danach:}
\begin{itemize}
  \item[\q] Sitzung beenden; temporäre Konten/Token \emph{sofort} sperren/löschen
  \item[\q] Funktionstest (Smoke-Test); ggf. Rollback aus Backup
  \item[\q] Ticket schließen mit Freigabe durch Praxisinhaberin
\end{itemize}

\textbf{Abbruchkriterien:} Unklare Identität, ungeplante Dateiübertragungen oder Deaktivierung von Schutzmechanismen.

{%
\renewcommand{\arraystretch}{1.0}
\setlength{\tabcolsep}{3pt}
\footnotesize

\begin{longtable}{p{0.30\textwidth} p{0.66\textwidth}}
\toprule
\AccessibleTableHeader{Remote-Support Protokoll}{}
\midrule
\textbf{Dienstleister / Techniker} & \FormField[7cm] \\[0.15cm]
\textbf{Ticket / Anlass / Ziel} & \FormField[7cm] \\
& \FormField[7cm] \\[0.15cm]
\textbf{Datum / Zeit / Dauer} & \FormField[7cm] \\[0.15cm]
\textbf{Betroffene Systeme} & (Host, OS, Inventar-ID) \FormField[5cm] \\[0.15cm]
\textbf{Zugänge} & (temporäres Konto, Rechte, 2FA) \FormField[5cm] \\[0.15cm]
\textbf{Durchgeführte Änderungen} & (Schritte, Versionen, Konfigs) \\
& \FormField[7cm] \\
& \FormField[7cm] \\[0.15cm]
\textbf{Dateiübertragungen} & (Quelle/Ziel, Zweck) \FormField[5cm] \\[0.15cm]
\textbf{Ergebnis / Tests} & (Smoke-Test ok/Fehler) \FormField[5cm] \\[0.15cm]
\textbf{Nacharbeiten} & (Passwortrotation, Cleanup) \FormField[5cm] \\[0.15cm]
\textbf{Freigabe Praxisinhaberin} & Name: \FormField[3cm] Datum: \FormField[2cm] \\
& Unterschrift: \FormField[4cm] \\
\bottomrule
\end{longtable}

}% Ende der footnotesize-Gruppe

\BestandteilZehnJahre

\Legende \fbox{\cmark} = Erledigt, \fbox{\xmark} = Fehler aufgetreten, \q = Noch nicht durchgeführt
