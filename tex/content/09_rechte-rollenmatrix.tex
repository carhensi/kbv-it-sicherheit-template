\section{Rechte- und Rollenmatrix}
\label{sec:rechte-rollenmatrix}

\subsection{Übersicht}

Die Rechte- und Rollenmatrix definiert systematisch die Zugriffsberechtigung auf alle
IT-Systeme der Praxis. Sie gewährleistet das Prinzip der minimalen Berechtigung und
unterstützt die Nachvollziehbarkeit von Systemzugriffen gemäß §~390 SGB~V.

\subsection{Rollendefinitionen}

\begin{tabular}{>{\raggedright\arraybackslash}p{3cm}>{\raggedright\arraybackslash}p{4cm}>{\raggedright\arraybackslash}p{7cm}}
\toprule
\textbf{Rolle} & \textbf{Beschreibung} & \textbf{Verantwortlichkeiten} \\
\midrule
Praxisinhaberin & \PraxisInhaberBezeichnung{} & Gesamtverantwortung IT-Sicherheit, Patientendaten, Compliance \\
IT-Support & Externer IT-Ansprechpartner & Technische Betreuung, Wartung, Updates (unter Aufsicht) \\
\PVSName{}-Support & PVS-Hersteller (extern) & \PVSName{}-System-Support, \RemoteTool{}-Fernwartung (unter Aufsicht) \\
Therapeut/in & Angestellte Therapeuten & Patientenbehandlung, Dokumentation (zukünftig) \\
Verwaltung & Verwaltungsangestellte & Termine, Abrechnung, Organisation (zukünftig) \\
Vertretung & Praxisvertretung & Notfall-Patientenversorgung (zukünftig) \\
\bottomrule
\end{tabular}

\subsection{Aktuelle Rollenzuordnung}

\begin{tabular}{ll}
\toprule
\textbf{Person} & \textbf{Rolle} \\
\midrule
\PraxisInhaberin & Praxisinhaberin \\
\ITAnsprechpartner{} (extern) & IT-Support \\
\PVSName{} GmbH (extern) & \PVSName{}-Support \\
\textit{(noch nicht besetzt)} & \textit{Therapeut/in (inkl. Vertretung)} \\
\textit{(noch nicht besetzt)} & \textit{Verwaltung (zukünftig)} \\
\bottomrule
\end{tabular}

\subsection{Systemberechtigungen nach Rollen}

\subsubsection{Praxisverwaltungssystem (\PVSName{})}

\begin{tabular}{p{4cm}p{2.5cm}p{2.5cm}p{2.5cm}p{2.5cm}}
\toprule
\textbf{Funktion} & \textbf{Praxisinhaberin} & \textbf{IT-Support} & \textbf{Therapeut/in*} & \textbf{Verwaltung*} \\
\midrule
Admin-Account & \cmark & \xmark & \xmark & \xmark \\
Standard-Benutzer & \cmark & \xmark & \cmark & \cmark \\
Patientendaten & Vollzugriff & \xmark & Eigene Patienten & Lesezugriff \\
Termine & Vollzugriff & \xmark & Eigene Termine & Vollzugriff \\
Abrechnung & Vollzugriff & \xmark & \xmark & Vollzugriff \\
Systemeinstellungen & \cmark & \xmark & \xmark & \xmark \\
\bottomrule
\end{tabular}

\textit{* = Zukünftige Rollen, noch nicht besetzt}

\textbf{Hinweis:} \PVSName{}-Support hat Admin-Zugriff auf alle PVS-Funktionen, jedoch nur unter Aufsicht der Praxisinhaberin via \RemoteTool{}-Session. \RemoteTool{} wird nur bei Bedarf gestartet und ist standardmäßig nicht aktiv. Alle Remote-Support-Sitzungen werden nach der SOP in \cref{sec:vorlage-remote-support} durchgeführt und dokumentiert.

\subsubsection{Sicherheit und Passwörter}

\begin{tabular}{p{4cm}p{2.5cm}p{2.5cm}p{2.5cm}p{2.5cm}}
\toprule
\textbf{System} & \textbf{Praxisinhaberin} & \textbf{IT-Support} & \textbf{Therapeut/in} & \textbf{Verwaltung} \\
\midrule
\PasswortManager{} Praxis-Vault & Vollzugriff & Emergency & \xmark & \xmark \\
\PasswortManager{} Shared Vault & \cmark & \cmark & \xmark & \xmark \\
TLS-Zertifikate & \cmark & \cmark & \xmark & \xmark \\
\bottomrule
\end{tabular}

\subsubsection*{Netzwerk-Infrastruktur}

\begin{tabular}{p{4cm}p{2.5cm}p{2.5cm}p{2.5cm}p{2.5cm}}
\toprule
\textbf{Komponente} & \textbf{Praxisinhaberin} & \textbf{IT-Support} & \textbf{Therapeut/in} & \textbf{Verwaltung} \\
\midrule
\RouterKurz{} Admin & \cmark & \cmark & \xmark & \xmark \\
WLAN-Verwaltung & \cmark & \cmark & \xmark & \xmark \\
Gastnetz-Aktivierung & \cmark & \cmark & \xmark & \xmark \\
Netzwerk-Monitoring & \cmark & \cmark & \xmark & \xmark \\
\bottomrule
\end{tabular}

\tabellenheader{Hardware und Geräte}
\begin{tabular}{p{4cm}p{2.5cm}p{2.5cm}p{2.5cm}p{2.5cm}}
\toprule
\textbf{Gerät} & \textbf{Praxisinhaberin} & \textbf{IT-Support} & \textbf{Therapeut/in} & \textbf{Verwaltung} \\
\midrule
MacBook (Benutzer) & Persönlicher User & Aufsicht & Persönlicher User & Persönlicher User \\
MacBook (Admin) & \texttt{\AdminUser{}}-Account & Aufsicht & \xmark & \xmark \\
Drucker (Nutzung) & \cmark & \cmark & \cmark & \cmark \\
Drucker (Admin) & \cmark & \cmark & \xmark & \xmark \\
TI-Terminal & \cmark & Updates & \xmark & \xmark \\
Mobile Geräte & \xmark & \xmark & \xmark & \xmark \\
\bottomrule
\end{tabular}

\tabellenheader{TI-Infrastruktur}
\begin{tabular}{p{4cm}p{2.5cm}p{2.5cm}p{2.5cm}p{2.5cm}}
\toprule
\textbf{Komponente} & \textbf{Praxisinhaberin} & \textbf{IT-Support} & \textbf{Therapeut/in} & \textbf{Verwaltung} \\
\midrule
SMC-B Karte & Geteilte Praxiskarte & \xmark & Geteilte Praxiskarte & \xmark \\
KIM/ePA/eAU & \cmark & \xmark & \cmark & \xmark \\
TI-Gateway & Monitoring & Updates & \xmark & \xmark \\
\bottomrule
\end{tabular}

\tabellenheader{Backup und Wiederherstellung}
\begin{tabular}{p{4cm}p{2.5cm}p{2.5cm}p{2.5cm}p{2.5cm}}
\toprule
\textbf{System} & \textbf{Praxisinhaberin} & \textbf{IT-Support} & \textbf{Therapeut/in} & \textbf{Verwaltung} \\
\midrule
NAS-Backup & Vollzugriff & \xmark & \xmark & \xmark \\
USB-Backup & Vollzugriff & \xmark & \xmark & \xmark \\
Restore-Durchführung & \cmark & Unterstützung & \xmark & \xmark \\
\bottomrule
\end{tabular}

\tabellenheader{Web-Services und Cloud}
\begin{tabular}{p{4cm}p{2.5cm}p{2.5cm}p{2.5cm}p{2.5cm}}
\toprule
\textbf{Service} & \textbf{Praxisinhaberin} & \textbf{IT-Support} & \textbf{Therapeut/in} & \textbf{Verwaltung} \\
\midrule
Website (\HostingAnbieter{}/\IaCTool{}) & \xmark & Vollzugriff & \xmark & \xmark \\
\HostingAnbieter{}/\CodeRepo{} & \xmark & Vollzugriff & \xmark & \xmark \\
E-Mail-Postfach & Persönlich + info@ & \xmark & \xmark & \xmark \\
iCloud-Zugriff & \cmark & \xmark & \xmark & \xmark \\
\MDMTool{} MDM & Vollzugriff & Vollzugriff & \xmark & \xmark \\
\KIToolName{} KI-Dokumentation & \cmark & \xmark & \cmark & \xmark \\
\bottomrule
\end{tabular}

\subsection{Berechtigungsprinzipien}

\textbf{Minimale Berechtigung:} Jede Rolle erhält nur die für ihre Aufgaben erforderlichen
Mindestrechte.

\textbf{Funktionstrennung:} Administrative und operative Tätigkeiten sind klar getrennt.

\textbf{Datenschutz:} Therapeuten haben nur Zugriff auf ihre eigenen Patientendaten.

\textbf{Vier-Augen-Prinzip:} Kritische Systemänderungen erfolgen in Abstimmung zwischen
Praxisinhaberin und IT-Support.

\textbf{Aufsichtspflicht:} Externe IT-Support-Tätigkeiten erfolgen grundsätzlich unter
Aufsicht der Praxisinhaberin.

\subsection{Änderungsmanagement}

\textbf{Neue Mitarbeitende:} Erhalten standardmäßig die Mindestrechte ihrer Rolle.
Erweiterte Berechtigungen werden individuell geprüft und genehmigt.

\textbf{Rollenwechsel:} Bei Funktionsänderungen werden Berechtigungen entsprechend
angepasst und alte Zugänge deaktiviert.

\textbf{Ausscheiden:} Alle Zugänge werden unverzüglich deaktiviert (spätestens am nächsten Werktag), 
geteilte Passwörter geändert,
Hardware zurückgegeben.

\textbf{Regelmäßige Überprüfung:} Die Rechte-Matrix wird jährlich im Rahmen der
TOMs-Überprüfung auf Aktualität und Angemessenheit geprüft.

\Dokumentation Alle Änderungen werden im Changelog der IT-Sicherheitsdokumentation
erfasst.
