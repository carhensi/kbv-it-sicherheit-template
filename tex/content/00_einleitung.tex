\section{Einleitung und Zweck}

\subsection{Zweck der Dokumentation}

Diese IT-Sicherheitsdokumentation wurde gemäß §~390 SGB~V erstellt und dient dem Nachweis
der ordnungsgemäßen Umsetzung der gesetzlich vorgeschriebenen
IT-Sicher\-heits\-maß\-nah\-men in der \PraxisName.

\textbf{Praxisprofil:}
\begin{itemize}
  \item \textbf{Praxisgröße:} \PraxisGroesse
  \item \textbf{Fachrichtung:} \PraxisFachrichtung
  \item \textbf{IT-Infrastruktur:} \PraxisITInfrastruktur
\end{itemize}

\textbf{Rechtsgrundlage:} KBV-Richtlinie nach §~390 SGB~V, Version 1.1 (ab
01.10.2025). Pflicht zur jährlichen Evaluation gem. Abschnitt A.IV letzter Abs.\ (siehe
\cref{sec:vorlage-eigenpruefung}).

Die Dokumentation richtet sich an Prüfer der \KVLang{}
(\KV) und andere befugte Kontrollinstanzen zur Überprüfung der Einhaltung der
IT-Sicherheitsrichtlinie.

\subsection{Rechtliche Grundlagen}

\begin{itemize}
  \item \S~390 SGB~V – IT-Sicherheit in der vertragsärztlichen Versorgung und
    –psychotherapeutischen Versorgung
  \item Anlage~1 zur IT-Sicherheitsrichtlinie (Richtlinie über die Anforderungen an die
    IT-Sicherheit)
  \item Art.~32 DSGVO – Sicherheit der Verarbeitung
  \item \S\S~203, 204 StGB – Verletzung von Privatgeheimnissen
  \item Berufsordnung der \KammerLang
\end{itemize}

\subsection{Dokumentenerstellung}

Diese Dokumentation wird regelmäßig aktualisiert und versioniert. Änderungen werden
nachvollziehbar dokumentiert und sicher archiviert. Das private Repository ist verfügbar
unter: \url{\GitHubRepoDoc} (Zugang nur für autorisierte Personen).

\textbf{Hinweis zur Druckversion:} Bei geringfügigen Änderungen wird die Druckversion
nicht neu erstellt. Die jeweils aktuelle digitale Fassung ist maßgeblich und im
Repository hinterlegt.

Sicherheitsmerkmale: SHA256-Checksummen für Integrität, versionierte
Änderungsnachverfolgung, sichere digitale Archivierung.

\subsection{Geltungsbereich}

Diese Dokumentation umfasst alle IT-Systeme, Netzwerke und Prozesse der
\PraxisName, die zur Verarbeitung von Patientendaten oder anderen
schützenswerten Informationen eingesetzt werden.

\textbf{Räumlicher Geltungsbereich:} Praxisräume \PraxisVollAdresse

\textbf{Zeitlicher Geltungsbereich:} Gültig vom \DocumentDate{} bis \ValidUntil

\subsection{Aufbau der Dokumentation}

Die Dokumentation ist thematisch strukturiert und umfasst folgende Hauptbereiche:

\begin{itemize}
  \item \textbf{IT-Sicherheitskonzept} (\cref{sec:it-sicherheit}) -- Allgemeine
    Sicherheitsprinzipien, Personal, Netzwerk, Updates und Endgeräte-Sicherheit
  \item \textbf{E-Mail und Kommunikation} (\cref{sec:kommunikation}) -- Sichere
    E-Mail-Konfiguration, Spam-Behandlung und Website-Sicherheit
  \item \textbf{Telematikinfrastruktur} (\cref{sec:ti}) -- TI-Komponenten und
    sichere Anbindung
  \item \textbf{Mobile Geräte} (\cref{sec:mobile-geraete}) -- Umfassende
    Richtlinien für iPhone/iPad inkl. Apps und Datenverarbeitung
  \item \textbf{Mobile Device Management} (\cref{sec:mdm}) -- Zentrale
    Geräteverwaltung, Sicherheitsrichtlinien und Compliance
  \item \textbf{Wechseldatenträger} (\cref{sec:wechseldatentraeger}) --
    Verschlüsselung, Mitnahme-Regelungen und sichere Entsorgung
  \item \textbf{Verschwiegenheit und externe Dienstleister}
    (\cref{sec:verschwiegenheit}) -- AVV und Vertraulichkeitsvereinbarungen
  \item \textbf{Notfall- und Incident-Management} (\cref{sec:notfall}) --
    Backup-Konzept und Wiederherstellungsverfahren
  \item \textbf{Technische und Organisatorische Maßnahmen} (\cref{sec:toms}) --
    DSGVO Art.\ 32 Compliance
  \item \textbf{Rechte- und Rollenmatrix} (\cref{sec:rechte-rollenmatrix}) --
    Systematische Zugriffsberechtigung auf alle IT-Systeme
  \item \textbf{Referenzdokumente} -- Netzplan, Geräteliste, AVV-Register,
    Verzeichnis von Verarbeitungstätigkeiten (VVT)
  \item \textbf{Anhänge} -- IT-Notfall-Checkliste, Selbstverpflichtungen,
    Verschwiegenheitserklärungen, KBV-Compliance-Mapping, Changelog
\end{itemize}

\textbf{Arbeitsmappe (getrennt von der Dokumentation):} Die Arbeitsmappe enthält
regelmäßig zu füllende Arbeitsblätter: Restore-Tests (mindestens jährlich, angestrebt halbjährlich),
TOMs-Prüfung (jährlich/quartalsweise) und IT-Sicherheitsevaluation (jährlich).
Diese werden kontinuierlich gepflegt, während die Hauptdokumentation als
Regelwerk weitgehend stabil bleibt.

Jedes Kapitel behandelt die jeweiligen Anforderungen der IT-Sicherheitsrichtlinie
vollständig und praxisnah. Querverweise ermöglichen eine themenübergreifende Navigation.

\textbf{Hinweis für Prüfer:} Diese Dokumentation muss nicht linear gelesen werden.
Das \textbf{KBV-Compliance-Mapping} (\cref{sec:kbv-compliance}) ermöglicht die gezielte
Prüfung einzelner Anforderungen: Jede Anforderung der Anlage~1, 2 und 5 ist dort
mit Erfüllungsstatus und Kapitelreferenz aufgeführt. So kann z.\,B. direkt
\enquote{Anlage~1 Nr.~21 -- Datensicherung} nachgeschlagen und im referenzierten Kapitel
vertieft werden.

\textbf{Arbeitsmappe:} Ausgefüllte Vorlagen liegen in der separaten Arbeitsmappe
(Arbeitsvorlagen, On-/Offboarding-Vorlagen, Protokolle, regelmäßige Checks).

\subsection{Verantwortlichkeiten}

\textbf{Gesamtverantwortung:} \Gesamtverantwortung

\textbf{IT-Sicherheitsbeauftragte:} \ITSicherheitsbeauftragte

\textbf{IT-Ansprechpartner:} \ITAnsprechpartner{} (externer IT-Ansprechpartner)

Die \PraxisInhaberBezeichnung{} trägt die Gesamtverantwortung für die Umsetzung und Einhaltung aller
IT-Sicher\-heits\-maß\-nah\-men gemäß §~390 SGB~V.

\subsection{Datenschutz-Information für Patienten}

Die Datenschutz-Information nach Art. 13 DSGVO ist in der Praxis ausgehängt und
online unter \url{https://\PraxisDomain/#datenschutz} abrufbar. Änderungen werden
versioniert dokumentiert und Patienten bei wesentlichen Änderungen informiert.
