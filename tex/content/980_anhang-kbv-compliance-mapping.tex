\section{Anhang – KBV-Compliance-Mapping}
\label{sec:kbv-compliance}

\subsection*{Zweck und Aufbau}

Dieses Kapitel dokumentiert die Umsetzung und den Erfüllungsstatus der Anforderungen der
IT-Sicherheitsrichtlinie nach §390 SGB V. Es dient als Nachweis gegenüber KBV-Prüfern
und ermöglicht die schnelle Lokalisierung relevanter Dokumentationsstellen.

\textbf{Aufbau:} Jede KBV-Anforderung wird mit Status und Verweis auf das entsprechende
Kapitel dieser Dokumentation verknüpft. Dadurch ist bei Prüfungen sofort ersichtlich,
wo die Umsetzung dokumentiert ist.

\subsection*{Übersicht Anlage 1 (IT-Sicherheitsrichtlinie §390 SGB V)}

\textit{Basis: KBV-Richtlinie v1.1 (gültig ab Oktober 2025)}

\small
\begin{longtable}{>{\raggedright\arraybackslash}p{0.7cm}
    >{\raggedright\arraybackslash}p{6.8cm} >{\centering\arraybackslash}p{1.2cm}
  >{\raggedright\arraybackslash}p{4.3cm}}
  \toprule
  \textbf{Nr.} & \textbf{Anforderung} & \Label{Status} &
  \textbf{Referenz/Kapitel} \\
  \midrule  \ifaccessible\else\endfirsthead
  \toprule
  \textbf{Nr.} & \textbf{Anforderung} & \Label{Status} &
  \textbf{Referenz/Kapitel} \\
  \midrule  \endhead\fi

  1 & Geregelte Einarbeitung neuer Mitarbeitender & + &~\ref{sec:personal} Personal
  (dokumentiert für zukünftige Einstellungen) \\
  2 & Geregelte Verfahrensweise beim Weggang von Mitarbeitenden & + &~\ref{sec:personal}
  Personal (dokumentiert für zukünftige Einstellungen) \\
  3 & Festlegung von Regelungen für den Einsatz von Fremdpersonal & +
  &~\ref{sec:personal} Personal \\
  4 & Vertraulichkeitsvereinbarungen für den Einsatz von Fremdpersonal & +
  &~\ref{sec:anhang-verschwiegenheit} Verschwiegenheit extern \\
  5 & Aufgaben und Zuständigkeiten von Mitarbeitenden & + &~\ref{sec:personal} Personal \\
  6 & Qualifikation des Personals & + &~\ref{sec:personal} Personal \\
  7 & Überprüfung der Vertrauenswürdigkeit von Mitarbeitenden & + &~\ref{sec:personal}
  Personal (Einstellungsverfahren dokumentiert) \\
  8 & Sensibilisierung der Praxisinhaberin für Informationssicherheit & +
  &~\ref{sec:sensibilisierung} Sensibilisierung \\
  9 & Einweisung des Personals in den sicheren Umgang mit IT & +
  &~\ref{sec:sensibilisierung} Sensibilisierung \\
  10 & Durchführung von Sensibilisierungen und Schulungen zur Informationssicherheit & +
  &~\ref{sec:sensibilisierung} Sensibilisierung und Schulung \\
  11 & Absicherung der Netzübergangspunkte & + &~\ref{sec:netzwerk} Netzwerksicherheit \\
  12 & Dokumentation des Netzes & + &~\ref{sec:anhang-netzplan} Netzplan \\
  13 & Grundlegende Authentisierung für den Netzmanagement-Zugriff & +
  &~\ref{sec:netzwerk} Netzwerksicherheit \\
  14 & Installation von Updates & + &~\ref{sec:patch} Patch-Management \\
  15 & Verantwortlichkeit für Updates & + &~\ref{sec:patch} Patch-Management \\
  16 & Identifizierung ausbleibender Updates & + &~\ref{sec:patch} Patch-Management \\
  17 & Ausmusterung oder Separierung bei ausbleibenden Updates & + &~\ref{sec:patch}
  Patch-Management \\
  18 & Verhinderung der unautorisierten Nutzung von Rechner-Mikrofonen und Kameras & +
  &~\ref{sec:endgeraete} Endgeräte-Sicherheit \\
  19 & Abmelden nach Aufgabenerfüllung & + &~\ref{sec:endgeraete} Endgeräte-Sicherheit \\
  20 & Einsatz von Viren-Schutzprogrammen & + &~\ref{sec:endgeraete} Endgeräte-Sicherheit \\
  21 & Regelmäßige Datensicherung & + &~\ref{sec:datensicherung} Datensicherung \\
  22 & Schutz der Datensicherung & + &~\ref{sec:datensicherung} Datensicherung \\
  23 & Art der Datensicherung & + &~\ref{sec:datensicherung} Datensicherung \\
  24 & Verantwortliche der Datensicherung & + &~\ref{sec:datensicherung} Datensicherung \\
  25 & Test der Datensicherung & + &~\ref{sec:vorlage-restore-tests} Restore-Tests \\
  26 & Der Zugriff auf Geräte und Software muss abgesichert werden & +
  &~\ref{sec:endgeraete} Endgeräte-Sicherheit \\
  27 & Konfiguration von Synchronisationsmechanismen (Windows) & -- &~\ref{sec:windows}
  Windows-Endgeräte \\
  28 & Datei- und Freigabeberechtigungen (Windows) & -- &~\ref{sec:windows} Windows-Endgeräte \\
  29 & Datensparsamkeit (Windows) & -- &~\ref{sec:windows} Windows-Endgeräte \\
  30 & Verwendung der SIM-Karten-PIN & + &~\ref{sec:mobile-geraete} Mobile Geräte \\
  31 & Sichere Grundkonfiguration für mobile Geräte & + &~\ref{sec:mobile-geraete} Mobile Geräte \\
  32 & Verwendung eines Zugriffschutzes & + &~\ref{sec:mobile-geraete} Mobile Geräte \\
  33 & Datenschutz-Einstellungen & + &~\ref{sec:mobile-geraete} Mobile Geräte \\
  34 & Sperrmaßnahmen bei Verlust eines Mobiltelefons & + &~\ref{sec:mobile-geraete}
  Mobile Geräte \\
  35 & Nutzung der Sicherheitsmechanismen von Mobiltelefonen & +
  &~\ref{sec:mobile-geraete} Mobile Geräte \\
  36 & Schutz vor Schadsoftware (Wechseldatenträger) & + &~\ref{sec:wechseldatentraeger}
  Wechseldatenträger \\
  37 & Angemessene Kennzeichnung der Datenträger beim Versand & n.a.
  &~\ref{sec:wechseldatentraeger} Kein Versand, nur Online-Übertragung \\
  38 & Sichere Versandart und Verpackung & n.a. &~\ref{sec:wechseldatentraeger} Wechseldatenträger \\
  39 & Sicheres Löschen der Datenträger vor und nach der Verwendung & +
  &~\ref{sec:wechseldatentraeger} Wechseldatenträger \\
  40 & Sichere Konfiguration der E-Mail-Clients & + &~\ref{sec:kommunikation} E-Mail-Sicherheit \\
  41 & Umgang mit Spam durch Benutzende & + &~\ref{sec:kommunikation} E-Mail-Sicherheit \\
  42 & Sichere Apps nutzen & + &~\ref{sec:mobile-geraete} Mobile Apps \\
  43 & Sichere Speicherung lokaler App-Daten & + &~\ref{sec:mobile-geraete} Mobile Apps \\
  44 & Verhinderung von Datenabfluss & + &~\ref{sec:mobile-geraete} Mobile Apps \\
  45 & Authentisierung bei Webanwendungen (Anbieter) & n.a. &~\ref{sec:kommunikation}
  Website-Sicherheit (statische Seite) \\
  46 & Schutz vertraulicher Daten (Anbieter) & n.a. &~\ref{sec:kommunikation} Website-Sicherheit \\
  47 & Einsatz von Web Application Firewalls (Anbieter) & n.a. &~\ref{sec:kommunikation}
  Website-Sicherheit \\
  48 & Schutz vor unerlaubter automatisierter Nutzung (Anbieter) & n.a.
  &~\ref{sec:kommunikation} Website-Sicherheit \\
  49 & Kryptografische Sicherung vertraulicher Daten (Anwender) & + &~\ref{sec:internet}
  Internet/Cloud \\
  50 & Sicherheit von Cloud-Dienstleistern & + &~\ref{sec:internet} Internet/Cloud \\
  \bottomrule
\end{longtable}

\subsection*{Übersicht Anlage 2 (Zusätzliche Anforderungen für mittlere Praxen) -
freiwillig (nicht verpflichtend für Praxen $\leq$5)}

\small
\begin{longtable}{>{\raggedright\arraybackslash}p{0.7cm}
    >{\raggedright\arraybackslash}p{6.8cm} >{\centering\arraybackslash}p{1.2cm}
  >{\raggedright\arraybackslash}p{4.3cm}}
  \toprule
  \textbf{Nr.} & \textbf{Anforderung} & \Label{Status} &
  \textbf{Referenz/Kapitel} \\
  \midrule  \ifaccessible\else\endfirsthead
  \toprule
  \textbf{Nr.} & \textbf{Anforderung} & \Label{Status} &
  \textbf{Referenz/Kapitel} \\
  \midrule  \endhead\fi

  1 & Alarmierung und Logging: Wichtige Ereignisse auf Netzkomponenten automatisch an zentrales Management-System übermitteln & ± &~\ref{sec:netzwerk} \RouterKurz{}-Logging, macOS-Systemlogs, kein zentrales SIEM (Einzelpraxis-angemessen) \\
  2 & Nutzung verschlüsselter Verbindungen (TLS): Kryptografische Algorithmen nach Stand der Technik verwenden & + &~\ref{sec:kommunikation} TLS 1.2+, HTTPS, S/MIME dokumentiert \\
  3 & Restriktive Rechtevergabe: Rechte nach Need-to-know-Prinzip, regelmäßige Überprüfung & + &~\ref{sec:rechte-rollenmatrix} Rollenmatrix, Einzelpraxis-Berechtigungen dokumentiert \\
  4 & Sichere zentrale Authentisierung in Windows-Netzen (Kerberos/SSO): Ausschließlich Kerberos für SSO & -- & Nicht anwendbar (macOS-Umgebung, kein Windows-Netz) \\
  5 & Richtlinie für mobile Geräte: Verbindliche Richtlinie für Mitarbeitende zur Benutzung mobiler Geräte & + &~\ref{sec:anhang-mobile} Mobile-Geräte-Richtlinie mit Unterschrift \\
  6 & Verwendung von Sprachassistenten: Nur bei zwingender Notwendigkeit, sonst deaktivieren & + &~\ref{sec:mobile-geraete} Siri deaktiviert, Hey Siri deaktiviert \\
  7 & Sicherheitsrichtlinien für Mobiltelefon-Nutzung: Nutzungs- und Sicherheitsrichtlinie erstellen & + &~\ref{sec:mobile-geraete} Mobiltelefon-Richtlinie dokumentiert \\
  8 & Sichere Datenübertragung über Mobiltelefone: Regelung welche Daten übertragen werden dürfen, Verschlüsselung & + &~\ref{sec:mobile-geraete} Nur organisatorische Daten, S/MIME-Verschlüsselung \\
  9 & Regelung zur Mitnahme von Wechseldatenträgern: Schriftliche Regeln für Transport außer Haus & + &~\ref{sec:wechseldatentraeger} Wechseldatenträger-Richtlinie mit Unterschrift \\
  10 & Minimierung und Kontrolle von App-Berechtigungen: App-Berechtigungen auf notwendiges Minimum einschränken & + &~\ref{sec:mdm} App-Berechtigungen restriktiv konfiguriert \\
  \bottomrule
\end{longtable}

\subsection*{Übersicht Anlage 5 (TI-Anforderungen)}

\small
\begin{longtable}{>{\raggedright\arraybackslash}p{0.7cm}
    >{\raggedright\arraybackslash}p{6.8cm} >{\centering\arraybackslash}p{1.2cm}
  >{\raggedright\arraybackslash}p{4.3cm}}
  \toprule
  \textbf{Nr.} & \textbf{Anforderung} & \Label{Status} &
  \textbf{Referenz/Kapitel} \\
  \midrule
  \ifaccessible\else\endfirsthead
  \toprule
  \textbf{Nr.} & \textbf{Anforderung} & \Label{Status} &
  \textbf{Referenz/Kapitel} \\
  \midrule
  \endhead\fi

  1 & Planung und Durchführung der Installation nach gematik-Vorgaben & + &~\ref{sec:ti}
  TI-Komponenten, \PVSName{}-Gateway nach Herstellervorgaben \\
  2 & Betrieb der TI-Komponenten & + &~\ref{sec:ti} TI-Betrieb dokumentiert \\
  3 & Schutz vor unberechtigtem physischem Zugriff & + &~\ref{sec:ti} SMC-B sicher
  aufbewahrt, Terminal am Praxisstandort \\
  4 & Internet-Verbindung parallel zur TI-Anbindung absichern & + &~\ref{sec:netzwerk}
  \RouterKurz{} Firewall, Netzwerksicherheit \\
  5 & Verbindung zu gehostetes TI-Gateway absichern (VPN) & + &~\ref{sec:ti}
  \PVSName{}-Gateway mit sicherer Verbindung \\
  6 & Beachtung der Vorgaben des TI-Gateway-Anbieters & + &~\ref{sec:ti} \PVSName{}-Vorgaben befolgt \\
  7 & Geschützte Kommunikation mit TI-Gateway (Zertifikate) & + &~\ref{sec:ti}
  Authentisierung über \PVSName{}-System \\
  8 & Zeitnahes Installieren verfügbarer Aktualisierungen & + &~\ref{sec:ti} Updates
  zeitnah eingespielt \\
  9 & Sicheres Aufbewahren von Administrationsdaten & + &~\ref{sec:ti}
  Administrationsdaten sicher aufbewahrt \\
  \bottomrule
\end{longtable}

\subsection*{Compliance-Status}

\Legende + = Erfüllt, ± = Teilweise erfüllt, × = Nicht erfüllt, n.a. = Nicht anwendbar

\textbf{Anlage 1:} 41 von 41 anwendbaren Anforderungen erfüllt (\textbf{100\% Compliance})
\begin{itemize}
  \item 41 Anforderungen: + Vollständig erfüllt
  \item 9 Anforderungen: n.a. Nicht anwendbar (Kein Versand: Nr. 37,38; Windows: Nr.
    27–29; Webdienst-Anbieter: Nr. 45–48)
\end{itemize}

\textbf{Anlage 2:} 7 von 8 anwendbaren Anforderungen vollständig erfüllt (\textbf{88\%
Compliance}), 1 teilweise erfüllt (\textbf{12\%})
\begin{itemize}
  \item 7 Anforderungen: + Vollständig erfüllt (Nr. 2,3,5,6,7,8,9,10)
  \item 1 Anforderung: ± Teilweise erfüllt (Nr. 1 - Logging vorhanden, kein zentrales SIEM)
  \item 1 Anforderung: -- Nicht anwendbar (Nr. 4 - macOS-Umgebung, kein Windows-Netz)
\end{itemize}

\textit{Hinweis: Für Praxen mit weniger als 6 Personen sind gemäß KBV die Anforderungen
  aus Anlage 1 und Anlage 5 umzusetzen (A.IV.1); Anlage 2 ist nicht verpflichtend, kann
aber freiwillig dokumentiert werden.}

\textbf{Anlage 5 (TI):} 9 von 9 Anforderungen erfüllt (\textbf{100\% Compliance})
\begin{itemize}
  \item 9 Anforderungen: + Vollständig erfüllt (Installation, Betrieb, Schutz, Updates,
    Administrationsdaten)
\end{itemize}

\textbf{Gesamtstatus:} Alle verpflichtenden Anforderungen gemäß KBV-Richtlinie v1.1
erfüllt (Anlage 1: 41 + Anlage 5: 9 = 50). Zusätzlich wurden Anlage-2-Anforderungen
(freiwillig) 7/8 vollständig und 1/8 teilweise umgesetzt. (Rechtsgrundlage: A.IV.1; neue
Pflichten ab 01.10.2025).

\textbf{KI-Software Compliance:} \KIToolName{} erfüllt die KBV-Anforderungen für KI-Software
gemäß KBV-Leitfaden \enquote{Praxiswissen KI} (S. 5): C5-Zertifizierung durch das BSI liegt vor.
Nachweis: \url{\KIToolSicherheitURL}

\normalsize
\vspace{1cm}
% \textit{Stand: \today}
