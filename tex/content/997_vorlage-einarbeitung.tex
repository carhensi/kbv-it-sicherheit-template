\section{Vorlage – Einarbeitung neuer Mitarbeitende}
\label{sec:vorlage-einarbeitung}

\textbf{Geltungsbereich:} Diese Checkliste gilt für die systematische Einarbeitung aller
neuen Mitarbeitenden vor Aufnahme der Tätigkeit.

\subsection*{Einarbeitung-Checkliste}

\textbf{Vor dem ersten Arbeitstag:}
\begin{itemize}
  \item[\q] Arbeitszeugnis und Referenzen geprüft
  \item[\q] Verschwiegenheitserklärung unterschrieben (Anhang)
  \item[\q] IT-Sicherheitsschulung durchgeführt (Schulungsplan)
  \item[\q] Benutzerkonten eingerichtet (\PVSName{}, E-Mail, etc.)
  \item[\q] Passwort-Manager-Zugang erstellt (\PasswortManager{})
  \item[\q] Mobile-Device-Richtlinie erklärt und unterschrieben
  \item[\q] Arbeitsplatz und Sicherheitsmaßnahmen gezeigt
  \item[\q] Notfallkontakte und -verfahren erklärt
\end{itemize}

\textbf{Zusätzliche Einarbeitung:}
\begin{itemize}
  \item[\q] IT-Sicherheitsdokumentation ausgehändigt und erklärt
  \item[\q] DSGVO-Schulung durchgeführt (TOMs Art. 32)
  \item[\q] Praxisspezifische Arbeitsabläufe erklärt
  \item[\q] Backup- und Notfallverfahren demonstriert
  \item[\q] Erste Woche: Tägliche Rücksprache über Sicherheitsfragen
\end{itemize}

\textbf{Verantwortlich:} \PraxisInhaberBezeichnung{} | \textbf{Frist:} Vor erstem Arbeitstag

\vspace{1cm}

\textbf{Bestätigung \PraxisInhaberBezeichnung{}:}\\
Ich bestätige, dass die Einarbeitung vollständig nach obiger Checkliste durchgeführt wurde.

\praxissignatur

\textbf{Bestätigung Fachkraft:}\\
Ich bestätige, dass ich die Einarbeitung erhalten und alle Sicherheitsrichtlinien verstanden habe.

\vspace{1.5cm}
\noindent\makebox[0.4\textwidth]{\hrulefill} \hfill \makebox[0.4\textwidth]{\hrulefill}\\
\makebox[0.4\textwidth][c]{Datum} \hfill \makebox[0.4\textwidth][c]{Unterschrift Fachkraft}

\BestandteilZehnJahre
