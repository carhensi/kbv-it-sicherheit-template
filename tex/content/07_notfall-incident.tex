\section{Notfall- und Incident-Management}
\label{sec:notfall}

\subsection{Notfallkontakte}

% Tagging: Keine Header-Zeile (Key-Value Tabelle)
\begin{tabular}{p{0.35\textwidth} p{0.55\textwidth}}
  \textbf{\PraxisInhaberBezeichnung{}:} & \PraxisInhaberin \\
  & Praxis: \PraxisTelefon, Privat: \PraxisInhaberinPrivatTelefon \\
  & E-Mail: info@\PraxisDomain \\[0.5em]
  \textbf{IT-Ansprechpartner:} & \ITAnsprechpartner \\
  & Mobil: \ITAnsprechpartnerTelefon \\
  & Adresse: \ITAnsprechpartnerAdresse \\[0.5em]
  \textbf{Externe Dienstleister:} & \PVSName{}, \InternetAnbieter{} (Internet) \\
  \textbf{Behörden:} & \KV{} (IT-Sicherheit), \Kammer{} (Kammer) \\
  & \Datenschutzbehoerde{} (Datenschutz-Aufsichtsbehörde): \DatenschutzbehoerdeAdresse \\
\end{tabular}

\subsection{Verfügbarkeitsziele}
\label{sec:verfuegbarkeit}

\textbf{Kritische Systeme (Praxisbetrieb gefährdet):}
\begin{itemize}
  \item \textbf{\PVSName{} PVS:} Max. 4h Ausfall, Ziel 99,5\% Verfügbarkeit
  \item \textbf{Internet/E-Mail:} Max. 2h Ausfall, Ziel 99,8\% Verfügbarkeit
  \item \textbf{Telefon:} Max. 1h Ausfall, Ziel 99,9\% Verfügbarkeit
\end{itemize}

\textbf{Wichtige Systeme (Betrieb eingeschränkt):}
\begin{itemize}
  \item \textbf{Drucker:} Max. 72h Ausfall, Ziel 95\% Verfügbarkeit
  \item \textbf{Mobile Geräte:} Max. 8h Ausfall, Ziel 99\% Verfügbarkeit
\end{itemize}

\textbf{Unkritische Systeme:}
\begin{itemize}
  \item \textbf{Website:} Max. 72h Ausfall, Ziel 95\% Verfügbarkeit
\end{itemize}

\subsection{Datenschutzvorfall-Management}

Bei Datenschutzverletzungen erfolgen binnen 24 Stunden die Dokumentation des Vorfalls
(Was, Wann, Wie, betroffene Daten), Schadensbegrenzung, Information des
IT-Ansprechpartners und eine Risikobewertung (siehe \cref{sec:vorlage-breach-register}).

Die Meldepflicht an die Aufsichtsbehörde LfD erfolgt binnen 72 Stunden mit Angabe der
Art der Verletzung, Kategorien betroffener Daten, ungefähre Anzahl Betroffener, Folgen
und ergriffene Maßnahmen. Bei hohem Risiko für Rechte und Freiheiten werden Betroffene
unverzüglich in verständlicher Sprache informiert.

\subsection{IT-Security-Incident-Response}

IT-Sicherheitsvorfälle werden nach Kategorien behandelt: Malware/Virus-Befall, unbefugter
Zugriff, Datenverlust/Hardware-Ausfall und Netzwerk-/Internet-Ausfall.

\textbf{Incident-Response-Timeline:}

\textbf{Sofortmaßnahmen (0–15 Min):}
\begin{itemize}
  \item Isolation: Betroffenes Gerät sofort vom Netz trennen (WLAN aus, Kabel ziehen)
  \item Schadensbegrenzung: Weitere Systeme prüfen und ggf. isolieren
  \item Erste Dokumentation: Was, Wann, Wo in Stichpunkten notieren
\end{itemize}

\textbf{Meldung und Analyse (15 Min - 2h):}
\begin{itemize}
  \item IT-Ansprechpartner informieren: Binnen 30 Min nach Abschluss der laufenden Patientenbehandlung (außer nachts)
  \item Incident-Log erstellen: Binnen 1h strukturiert dokumentieren
  \item Erste Ursachenanalyse: Binnen 2h Verdacht eingrenzen
\end{itemize}

\textbf{Wiederherstellung (2h - 24h):}
\begin{itemize}
  \item Bereinigung:
    \begin{itemize}
      \item Malware-Befall: 2–4h (Neuinstallation aus Backup)
      \item Ransomware: 4–8h (vollständige Systemwiederherstellung)
      \item Hardware-Ausfall: 24h (Ersatzbeschaffung)
    \end{itemize}
  \item System-Wiederherstellung: \PVSName{} binnen 4h, andere Systeme binnen 24h
  \item Vollständige Dokumentation: Binnen 24h abgeschlossen
\end{itemize}

\textbf{Nachbereitung (24h - 72h):}
\begin{itemize}
  \item Ursachenanalyse final: Binnen 48h
  \item Lessons Learned: Binnen 72h dokumentiert
  \item Maßnahmen-Update: Binnen 1 Woche umgesetzt
\end{itemize}

\textbf{Eskalation:} Nach 2h ohne Lösung → Externe Hilfe (\PVSName{}). Nach 4h → Praxisbetrieb gefährdet → Notfallmodus.

\subsection{Frühwarnsystem (Honeypots)}
\label{sec:honeypots}

Zur frühzeitigen Erkennung unbefugter Zugriffe sind digitale Köder (Canary Tokens)
im Praxisnetzwerk platziert. Diese inaktiven Dateien simulieren vertrauliche Inhalte
und lösen bei Zugriff automatisch Alarm aus.

\textbf{Implementierung:}
\begin{itemize}
  \item Mehrere Köderdateien verschiedener Typen an strategischen Speicherorten
  \item Dateinamen suggerieren sensiblen Inhalt (z.\,B. Patientendaten, Abrechnungen)
  \item Bei Zugriff: Automatische E-Mail-Benachrichtigung an Praxisinhaberin
  \item Alarminhalt: Zeitstempel, IP-Adresse, Gerätekennung
\end{itemize}

\textbf{Reaktion bei Auslösung:}
\begin{enumerate}
  \item Sofortige Prüfung: Legitimer Zugriff (Fehlalarm) oder Sicherheitsvorfall?
  \item Bei Verdacht: Incident-Response-Prozess gemäß vorherigem Abschnitt einleiten
  \item Dokumentation im Vorfallprotokoll
\end{enumerate}

\textit{Hinweis: Genaue Dateitypen, Namen und Speicherorte werden aus Sicherheitsgründen
nicht dokumentiert und sind nur der Praxisinhaberin bekannt.}

\subsection{Backup-Wiederherstellung}

Die Wiederherstellung erfolgt primär über lokales Time Machine-Backup, sekundär über das
NAS am \BackupStandort{}. Verantwortlich ist \ITAnsprechpartner{} mit Zugriff über
\ITAnsprechpartnerAdresse.

Der Wiederanlauf erfolgt in folgender Reihenfolge: Netzwerk und Internet, Betriebssystem
und Grundkonfiguration, Praxisverwaltungssoftware (\PVSName{}), E-Mail und Kommunikation,
Patientendaten und Dokumentation.

\subsection{IT-Risikobewertung}
\label{sec:risikobewertung}

\begin{center}
\small
\begin{tabular}{p{2.5cm}p{2cm}p{1.8cm}p{1.5cm}p{3cm}}
\toprule
\textbf{Bedrohung} & \textbf{Wahrschein-lichkeit} & \textbf{Schaden} & \textbf{Risiko-Score} & \textbf{Maßnahmen} \\
\midrule
Ransomware & Mittel & Kritisch & HOCH & Backup, Schulung, macOS Security \\
Gerätediebstahl & Niedrig & Hoch & MITTEL & FileVault, Remote Wipe \\
Internetausfall & Hoch & Mittel & MITTEL & Mobilfunk-Backup \\
Datenverlust & Niedrig & Kritisch & MITTEL & 3–2-1 Backup-Regel \\
Phishing & Mittel & Mittel & MITTEL & S/MIME, Schulung \\
\bottomrule
\end{tabular}
\end{center}
\normalsize

\textbf{Risiko-Score:} Niedrig×Niedrig=NIEDRIG, Niedrig×Mittel=NIEDRIG,
Mittel×Mittel=MITTEL, Hoch×Hoch=KRITISCH

\subsection{KI-Dokumentationsausfall}

Bei Ausfall der KI-gestützten Dokumentation erfolgt die Sitzungsdokumentation
direkt digital im PVS. Patientenbehandlung ist nicht beeinträchtigt, da die
manuelle Dokumentation den regulären Behandlungsstandard darstellt.

\subsection{Eskalation}

Die Eskalation erfolgt stufenweise: IT-Ansprechpartner (\ITAnsprechpartner), externe
Dienstleister (\PVSName{}, Provider), Behörden (bei Datenschutzvorfällen), Kammer und sofern
vorhanden Versicherung (bei schweren Vorfällen).

Die Erreichbarkeit ist während der Praxiszeiten nach Absprache mobil gewährleistet,
außerhalb der Zeiten über Privat \PraxisInhaberinPrivatTelefon. IT-Support erfolgt über
\ITAnsprechpartner{} \ITAnsprechpartnerTelefon.

\subsection{IT-Sicherheits-KPIs}
\label{sec:kpis}

\textbf{Monatliche Metriken:}
\begin{itemize}
  \item \textbf{Backup-Erfolgsrate:} Ziel >99\% (Time Machine + NAS)
  \item \textbf{Update-Compliance:} Ziel 100\% binnen 14 Tagen
  \item \textbf{Phishing-Tests:} Quartalsweise Selbsttest
  \item \textbf{Passwort-Hygiene:} \PasswortManager{} Security Score >80
\end{itemize}

\textbf{Jährliche Reviews:}
\begin{itemize}
  \item Penetration-Test (Selbsttest mit Online-Tools, angestrebt)
  \item Compliance-Audit (Eigenprüfung gegen KBV-Anforderungen)
  \item Risikobewertung-Update
  \item Notfallplan-Test (Tabletop-Übung, mindestens jährlich, angestrebt halbjährlich)
\end{itemize}

\textbf{Eskalation bei Unterschreitung:}
Backup <95\% → Sofortige Prüfung, Updates >30 Tage → Risikobewertung

\subsection{Jährliche Tabletop-Übung}

Zur Überprüfung der Notfallprozesse und Schulung der Beteiligten wird
mindestens jährlich, angestrebt halbjährlich, eine 30-minütige Tabletop-Übung durchgeführt. Typische Szenarien:
Ransomware-Angriff, Geräteverlust/Diebstahl, Netzwerkausfall oder Datenschutzvorfall.

\textbf{Ablauf:} Szenario-Präsentation, Rollenverteilung, Durchspielen der Maßnahmen
nach Notfallplan, Identifikation von Lücken und Verbesserungsmaßnahmen, Dokumentation
der Erkenntnisse und Ableitung von Anpassungen.

\textbf{Teilnehmende:} \PraxisInhaberBezeichnung{}, IT-Ansprechpartner, ggf. weitere Mitarbeitende.

\Dokumentation Strukturierte Protokollierung nach Arbeitsvorlage in
\cref{sec:vorlage-tabletop} mit Szenario, Ablauf, Erkenntnissen und Maßnahmen.
