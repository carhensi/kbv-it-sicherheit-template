\section{Technische und Organisatorische Maßnahmen}
\label{sec:toms}

\textbf{Hinweis zur Darstellung der TOMs nach Art.\ 32 DSGVO:}\\
Gemäß Art.\ 32 DSGVO sind alle Maßnahmen vollständig zu dokumentieren.
Die nachfolgende Übersicht stellt daher die TOMs in rechtlicher Vollständigkeit dar.
Zur Vermeidung von Redundanz verweisen einzelne Maßnahmen auf die entsprechenden Fachkapitel
(z.\,B.\ \ref{sec:netzwerk} Netzwerksicherheit,~\ref{sec:kommunikation} E-Mail-Kommunikation),
in denen die technische Umsetzung detailliert beschrieben ist.
Zusätzlich dient der Netzplan im Anhang zur Visualisierung der Konfiguration.

\subsection{Übersicht}
Die technischen und organisatorischen Maßnahmen nach Art.\ 32 DSGVO gewährleisten\ die
Datensicherheit durch ein mehrstufiges Schutzkonzept.

\subsection{Zugangs- und Zutrittskontrolle}

Die Praxis ist abschließbar mit Schlüsseln nur bei der \PraxisInhaberBezeichnung{}
und dem Reinigungspersonal (mit Verschwiegenheitserklärung). Benutzerkonten
existieren nur für autorisierte Personen mit starken Passwörtern (\PasswortManager{}) und
automatischer Bildschirmsperre. Zwei-Faktor-Authentifizierung wird wo möglich eingesetzt.

\subsection{Zugriffs- und Weitergabekontrolle}

Rollenbasierte Zugriffe im PVS (\PVSName{}) beschränken Berechtigungen auf erforderliche
Daten. Gemeinsam genutzte Accounts existieren nicht. Die systematische Zugriffskontrolle
ist in der Rechte- und Rollenmatrix (\cref{sec:rechte-rollenmatrix}) vollständig
dokumentiert. Bei Datenübertragung werden verschlüsselte Datenträger und sichere
Übertragungswege (HTTPS) verwendet.

\subsection{Eingabe- und Auftragskontrolle}

Dateneingaben erfordern Benutzer-Authentifizierung und werden in \PVSName{} protokolliert.
Änderungshistorien werden bei kritischen Daten geführt. Auftragsverarbeitungsverträge
(AVV) mit Dienstleistern regeln die Verarbeitung nach Weisung, ergänzt durch
Vertraulichkeitsvereinbarungen.

\subsection{Verfügbarkeits- und Trennungskontrolle}

Backup-Konzept und redundante Systeme sichern die Verfügbarkeit (Details siehe
\cref{sec:datensicherung}). Separate Benutzerkonten für verschiedene Aufgaben und
getrennte Netzwerke (Praxis vs. privat)\ gewährleisten\ die Zwecktrennung.

\subsection{Verschlüsselung und Pseudonymisierung}

Festplattenverschlüsselung (FileVault), verschlüsselte Backups und Wechseldatenträger
schützen personenbezogene Daten. E-Mail-Verschlüsselung siehe
\cref{sec:kommunikation}. Statistiken und Berichte werden pseudonymisiert erstellt.

\subsection{Netzwerk- und Systemsicherheit}

Netzwerksicherheit durch Firewall-Konfiguration (Details siehe
\cref{sec:netzwerk}). Systemlogs werden anlassbezogen ausgewertet, zusätzlich
stichprobenartige Sichtung im jährlichen Eigencheck. Automatische Software-Updates und
integrierte macOS-Sicherheitsfeatures (XProtect, Gatekeeper) gewährleisten aktuellen
Schutzstand (Details siehe \cref{sec:endgeraete}).

\subsection{Physische Sicherheit}

\textbf{Zutrittskonzept:} Die Praxis ist abschließbar, Schlüssel werden nur an
autorisiertes Personal ausgegeben. Bei zukünftigen
Einstellungen: Schlüsselvergabe dokumentiert, Rückgabe beim Ausscheiden.
Büro/IT-Räume sind zusätzlich gesichert und nur für Praxisinhaberin und IT-Support
unter Aufsicht zugänglich.

\textbf{Clean-Desk-Policy:} Digitale Arbeitsweise ohne Papierakten. Bildschirme werden
bei Verlassen des Arbeitsplatzes gesperrt (automatisch nach 5 Min). Wechseldatenträger
werden verschlüsselt und in abschließbaren Schränken aufbewahrt. Ausdrucke mit
Patientendaten werden unmittelbar nach Verwendung geschreddert.

\textbf{Besucherregelung:} Patienten haben Zugang zum Wartebereich und Behandlungsräumen,
nicht zu Büro/IT-Räumen. Externe Dienstleister (IT-Support, Wartung) arbeiten nur unter
Aufsicht oder außerhalb der Praxiszeiten. Alle externen Zugriffe werden dokumentiert
(siehe \cref{sec:vorlage-remote-support}).

\textbf{Geräte- und Bildschirmsicherheit:} IT-Geräte sind in abschließbaren Räumen
untergebracht. Mobile Geräte werden bei Nichtgebrauch sicher verwahrt. Bildschirme sind
vor Einsicht Unbefugter geschützt positioniert und mit Sichtschutzfolien ausgestattet.
Kameras sind mit Abdeckungen versehen.

\subsection{Regelmäßige Überprüfung}

Jährliche Überprüfung der TOMs und halbjährliche IT-Sicherheitsprüfungen (angestrebt) 
gewährleisten kontinuierliche Verbesserung. Für Einzelpraxen ist eine jährliche 
umfassende Prüfung ausreichend, ergänzt durch anlassbezogene Kontrollen bei 
Änderungen oder Sicherheitsvorfällen. Die Rechte- und Rollenmatrix wird jährlich auf Aktualität
und Angemessenheit geprüft. Anpassungen erfolgen bei Änderungen der Technik oder
Rechtslage mit vollständiger Dokumentation.

Die strukturierte Überprüfung aller TOMS-Bereiche erfolgt nach den Arbeitsvorlagen
in \cref{sec:vorlage-toms-pruefung} mit dokumentierten Prüfzyklen und Nachweisführung.

IT-Sicherheitsschulungen werden jährlich durch die \PraxisInhaberBezeichnung{} durchgeführt, 
quartalsweise angestrebt, und nach dem Schulungsplan in \cref{sec:vorlage-schulungsplan} dokumentiert. 
Zusätzliche Schulungen erfolgen anlassbezogen bei aktuellen Bedrohungen oder Sicherheitsvorfällen.

\textbf{Hinweis für Einzelpraxen:} Die gesetzliche Mindestanforderung (§390 SGB V) beträgt 
4 Stunden jährlich. Quartalsweise Kurzmodule sind wünschenswert, aber nicht verpflichtend.
Bei zukünftigen Einstellungen sind alle neuen Mitarbeitenden vor Systemzugang zu schulen.

Alle Prüfungen werden systematisch durchgeführt und 10 Jahre aufbewahrt.

\subsection{Betroffenenrechte nach Art. 12--23 DSGVO}

Die Praxis gewährleistet die ordnungsgemäße Bearbeitung aller Betroffenenanfragen
nach DSGVO Art. 15 (Auskunft), Art. 16 (Berichtigung), Art. 17 (Löschung),
Art. 18 (Einschränkung), Art. 20 (Datenübertragbarkeit) und Art. 21 (Widerspruch).

\textbf{Verfahren:} Eingehende Anträge werden registriert, die Identität des
Antragstellers geprüft (Geburtsdatum, Adresse) und binnen einem Monat bearbeitet.
Bei komplexen Anfragen erfolgt eine begründete Fristverlängerung um maximal zwei Monate.
Alle betroffenen Datenbestände (PVS, E-Mail, Papierakten, Backups) werden systematisch
geprüft und die Bearbeitung vollständig dokumentiert.

\textbf{Verantwortlichkeit:} \PraxisInhaberBezeichnung{} als Verantwortliche nach Art. 4 Nr. 7 DSGVO.

\Dokumentation Strukturierte Bearbeitung nach Arbeitsvorlage in
\cref{sec:vorlage-betroffenenrechte} mit Nachweis über Eingang, Identitätsprüfung,
Fristberechnung, Datenbestandsprüfung und Antwortversendung.
