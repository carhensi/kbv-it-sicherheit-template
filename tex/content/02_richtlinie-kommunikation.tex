\section{E-Mail und Kommunikation}
\label{sec:kommunikation}

\subsection{E-Mail-Sicherheit}

Als E-Mail-Anbieter wird \EmailAnbieter{} genutzt (Custom Domain \PraxisDomain).

\textbf{Wichtig:} \EmailAnbieter{} wird ausschließlich für organisatorische Kommunikation
verwendet (keine Gesundheitsdaten). AVV mit \EmailAnbieterFirma{}
vom 30.11.2025 abgeschlossen (siehe AVV-Register Anhang).

\textbf{Mail-Gateway-Architektur (Anlage 3):} Die Praxis nutzt \EmailAnbieter{}
als hochverfügbare Gateway-Lösung mit überlegenen Sicherheitsfeatures:

\begin{itemize}
  \item 24/7 Spam-/Virenfilter mit Machine Learning
  \item Automatische DKIM-Signierung und SPF-Validierung
  \item TLS 1.3 Transportverschlüsselung mit Perfect Forward Secrecy
  \item 99.9\% SLA-garantierte Verfügbarkeit
\end{itemize}

\textbf{Architektur-Vorteil:} Diese Managed-Service-Architektur übertrifft lokale
Gateway-Lösungen in Sicherheit, Verfügbarkeit und Wartungsaufwand. Lokale
Mail-Gateways erfordern kontinuierliche Updates, Monitoring und Expertise,
die in Einzelpraxen nicht wirtschaftlich darstellbar sind.

Digitale Signatur aller ausgehenden E-Mails (Authentizität, Integrität) per S\slash
MIME-Si\-gna\-tur standardmäßig aktiv.
Ende-zu-Ende-Verschlüsselung per S\slash MIME sofern ein gültiges Empfängerzertifikat
vorliegt; andernfalls erfolgt der Versand transportverschlüsselt (TLS). Patienten können
unverschlüsselt Kontakt aufnehmen.

\textbf{E-Mail-Policy für Patienten:} Sensible Inhalte (Diagnosen, Therapieinhalte,
Befunde) werden nicht per E-Mail übermittelt. Sichere Kommunikation erfolgt über
\PatKomTool{}-App, KIM (mit anderen Leistungserbringern, siehe \cref{sec:kim}) oder
persönlich in der Praxis.

\textbf{E-Mail-Signatur:} Alle ausgehenden E-Mails enthalten folgenden Hinweis:
\begin{quote}
\textit{Hinweis: Bitte keine medizinischen Inhalte per E-Mail.
Für sichere Kommunikation nutzen Sie \PatKomTool{} oder rufen Sie an.}
\end{quote}

\textbf{Arbeitsanweisung bei medizinischen E-Mail-Inhalten:} Erhält die Praxis
E-Mails mit medizinischen Inhalten (Diagnosen, Symptome, Befunde):
\begin{enumerate}
  \item Vorgang kurz im PVS protokollieren (Datum, Patient, \enquote{E-Mail erhalten, sicherer Kanal genutzt})
  \item E-Mail sofort löschen (nicht weiterleiten oder speichern)
  \item Patient telefonisch kontaktieren
  \item Inhalt über \PatKomTool{} oder persönlich anfordern
  \item Patienten über sichere Kommunikationswege informieren
\end{enumerate}

\textbf{Wichtig:} E-Mails mit Gesundheitsdaten werden nach Übernahme ins PVS in der
Mailbox gelöscht. Es erfolgt keine Ablage gesundheitsbezogener Inhalte in
iCloud-Mail, iCloud-Drive oder Kalendernotizen.

\textbf{Technische Umsetzung der E-Mail-Sicherheit:} Um diese organisatorischen
Maßnahmen technisch zu unterstützen, implementiert die Praxis folgende
Verschlüsselungs- und Authentifizierungsverfahren:

\textbf{S/MIME-Zertifikat (nur Signatur):} Ausgehende E-Mails werden mit S/MIME
signiert (nicht verschlüsselt). Die Signatur bestätigt die Authentizität des Absenders.
\begin{itemize}
  \item \textbf{Aussteller}: \SMIMEIssuer{}
  \item \textbf{Gültigkeit}: \SMIMEValidity{}
  \item \textbf{Speicherung}: \PasswortManager{} + macOS Keychain
\end{itemize}

\textbf{S/MIME-Rotation:}
\begin{enumerate}
  \item Kalender-Erinnerung 60 Tage vor Ablauf (\PraxisInhaberBezeichnung{})
  \item Neues Zertifikat beim Anbieter bestellen (IT-Ansprechpartner)
  \item Installation in macOS Keychain und Test (IT-Ansprechpartner)
  \item Altes Zertifikat in \PasswortManager{} archivieren
\end{enumerate}

Die Domain-Sicherheit wird durch folgende technisch implementierte Maßnahmen
gewährleistet (öffentlich prüfbar via DNS-Abfrage):
\begin{itemize}
  \item \textbf{DNSSEC} zur Authentizität von DNS-Einträgen
  \item \textbf{CAA-Records} zur Kontrolle der Zertifikatsausstellung
  \item \textbf{Registrar Lock} verhindert unbefugte Domain-Transfers
\end{itemize}

\subsubsection{E-Mail-Sicherheitsstandards}
Für die Domain \texttt{\PraxisDomain} sind folgende Sicherheitsstandards aktiv:
\begin{itemize}
\item \textbf{SPF}: Autorisiert \EmailAnbieter{} als legitimen Mailserver
\item \textbf{DKIM}: Digitale Signierung durch \EmailAnbieter{}
\item \textbf{DMARC}: Policy \texttt{p=reject} verwirft nicht-authentifizierte E-Mails
\item \textbf{MTA-STS}: Erzwingt verschlüsselten Transport (TLS 1.2+)
\end{itemize}

\textbf{Implementierung:} SPF/DKIM-Signierung über \EmailAnbieter{}, DMARC mit Policy \texttt{p=reject}.

\textbf{DMARC-Report-Strategie:} Da ausschließlich \EmailAnbieter{} als Managed Service
genutzt wird (keine eigenen Mailserver), erfolgt die Sicherheitsüberwachung durch:
\begin{itemize}
  \item \EmailAnbieter{} überwacht automatisch Zustellbarkeit und Authentifizierung
  \item DMARC-Reports an \texttt{\PostmasterEmail} (halbjährliche Prüfung angestrebt)
  \item Halbjährliche Prüfung (angestrebt): DKIM/SPF-Authentifizierung, Zustellbarkeit, Auffälligkeiten
  \item Jährliche DNS-Validierung im Rahmen der IT-Evaluation (\cref{sec:vorlage-eigenpruefung})
  \item Bei Zustellproblemen: Analyse über MXToolbox oder dmarcian
\end{itemize}

DNS-Änderungen werden über \IaCTool{} verwaltet und versioniert
(siehe \cref{sec:website-sicherheit}).

\subsection{Spam-Behandlung}

Gemäß Anlage 1 Nr. 41 der IT-Sicherheitsrichtlinie werden Spam-E-Mails grundsätzlich
ignoriert und gelöscht. Verdächtige E-Mails werden konsequent gelöscht ohne Öffnen von
Links oder Anhängen aus unbekannten Quellen. Links in verdächtigen E-Mails werden nicht
angeklickt, Anhänge aus unbekannten Quellen nicht geöffnet. Bei Unsicherheit wird die
E-Mail an den IT-Ansprechpartner weitergeleitet, ansonsten sofort gelöscht.

\subsection{E-Mail-Client-Konfiguration}

Apple Mail ist sicherheitsorientiert konfiguriert: Externe Inhalte werden standardmäßig
unterdrückt, aktive Inhalte sind deaktiviert (Anlage 1 Nr. 40), und die automatische
Bildanzeige ist deaktiviert. E-Mail-Anhänge werden vor dem Öffnen durch die integrierten
macOS-Sicherheitsmechanismen (XProtect, Gatekeeper) geprüft.

\subsection{Website-Sicherheit}
\label{sec:website-sicherheit}

Die Praxis-Website wird bei \HostingAnbieter{} (DSGVO-konform mit AVV) gehostet und nutzt erzwungene
HTTPS-Verschlüsselung. Infrastructure as Code via \IaCTool{} für sichere und nachvollziehbare
Konfiguration (privates Repository: \url{\GitHubRepoWeb}).
Security-Header konfiguriert (u.a. HSTS, CSP, X-Content-Type-Options, Referrer-Policy);
jährliche Sichtprüfung (siehe \cref{sec:vorlage-eigenpruefung}).

\textbf{Wartung:} IT-Ansprechpartner ist für Wartung und Updates verantwortlich.
Da statisches HTML ohne Datenbank oder CMS, sind regelmäßige Updates nicht erforderlich.
Inhaltliche Änderungen erfolgen anlassbezogen.

\textbf{Anlage~1 Nr.~45–48 (nicht anwendbar):} Die Website bietet keine Logins oder
geschützten Ressourcen. Authentifizierung, WAF und Anti-Automation-Maßnahmen sind
daher nicht erforderlich - die statische Informationsseite benötigt nur die
implementierten Security-Header.

Es werden keine Cookies, Tracking- oder Analysetools eingesetzt.

Technisch notwendige Server-Log-Dateien (Browsertyp, Betriebssystem, Referrer, Hostname,
Uhrzeit der Anfrage, IP-Adresse) werden 30 Tage gespeichert und danach automatisch gelöscht.
Eine Zusammenführung mit anderen Datenquellen findet nicht statt. Rechtsgrundlage ist
Art.\ 6 Abs.\ 1 lit. f\ DSGVO\ (berechtigtes Interesse an Betrieb, Sicherheit und
Optimierung der Website).
