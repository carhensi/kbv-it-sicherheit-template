\section{Verschwiegenheit und externe Dienstleister}
\label{sec:verschwiegenheit}

\subsection{Grundsätze}
Alle externen Dienstleister, die Zugang zu Praxisdaten haben könnten, werden zur
Verschwiegenheit verpflichtet. Dies umfasst IT-Support, Wartungsdienste und andere
Auftragsverarbeiter.

\subsection{Auftragsverarbeitung}
Mit allen Dienstleistern werden Auftragsverarbeitungsverträge (AVV) nach Art.~28 DSGVO
geschlossen. Diese regeln die Verarbeitung nach Weisung, technische und organisatorische
Maßnahmen sowie Löschungsfristen (siehe \cref{sec:avv-register}).

\subsection{Zugangskontrollen}
Externe Zugriffe erfolgen nur nach vorheriger Terminabsprache und unter Aufsicht.
Remote-Zugriffe sind organisatorisch untersagt. IT-Support erfolgt ausschließlich
vor Ort oder unter direkter Bildschirmaufsicht (siehe SOP \cref{sec:vorlage-remote-support}).

\subsection{Dokumentation}
Alle Verschwiegenheitserklärungen externer Dienstleister sind in
\cref{sec:anhang-verschwiegenheit} dokumentiert und unterzeichnet (siehe
Verschwiegenheitserklärung externe Dienstleister).
