\section{Vorlage – Tabletop-Übung (Notfallsimulation)}
\label{sec:vorlage-tabletop}

\Hinweis Template für jährliche Notfall-Tabletop-Übungen zur Überprüfung
der Business-Continuity-Prozesse. 10 Jahre Aufbewahrung.

%% Präambel - definiere nur wenn nicht schon vorhanden
\providecommand{\qsize}{1.05ex}
\renewcommand{\qsize}{1.05ex}
\providecommand{\q}{\fbox{\rule{0pt}{\qsize}\rule{\qsize}{0pt}}}
\renewcommand{\q}{\fbox{\rule{0pt}{\qsize}\rule{\qsize}{0pt}}}

{%
\renewcommand{\arraystretch}{1.0}
\setlength{\tabcolsep}{3pt}
\footnotesize

\begin{longtable}{p{0.30\textwidth} p{0.66\textwidth}}
\toprule
\AccessibleTableHeader{Tabletop-Übung Protokoll}{}
\midrule
\textbf{Datum / Uhrzeit} & \FormField[4cm] \quad Dauer: \FormField[2cm] Min \\[0.15cm]
\textbf{Teilnehmende} & \FormField[6cm] \\
& \FormField[6cm] \\[0.15cm]
\textbf{Szenario} & \q~Ransomware \q~Geräteverlust \q~Netzwerkausfall \q~Datenschutzvorfall \\
& \q~Sonstiges: \FormField[4cm] \\[0.15cm]
\textbf{Szenario-Details} & Beschreibung der simulierten Situation: \\
& \FormField[6cm] \\
& \FormField[6cm] \\[0.15cm]
\textbf{Rollenverteilung} & Incident Commander: \FormField[3cm] \\
& IT-Ansprechpartner: \FormField[3cm] \\
& Weitere Rollen: \FormField[3cm] \\[0.15cm]
\textbf{Durchgeführte Maßnahmen} & \q~Sofortmaßnahmen \q~Eskalationswege \q~Kommunikation \\
& \q~Wiederherstellung \q~Dokumentation \\[0.15cm]
\textbf{Erkenntnisse} & Was lief gut? \FormField[5cm] \\
& Was war unklar/problematisch? \FormField[4cm] \\[0.15cm]
\textbf{Identifizierte Lücken} & Prozess-Lücken: \FormField[4cm] \\
& Schulungsbedarf: \FormField[4cm] \\
& Technische Verbesserungen: \FormField[4cm] \\[0.15cm]
\textbf{Maßnahmen} & 1. \FormField[4cm] bis \FormField[1.5cm] \\
& 2. \FormField[4cm] bis \FormField[1.5cm] \\
& 3. \FormField[4cm] bis \FormField[1.5cm] \\[0.15cm]
\textbf{Nächste Übung} & Geplant für: \FormField[3cm] Fokus: \FormField[3cm] \\[0.15cm]
\textbf{Freigabe} & Übungsleiter: \FormField[3cm] Datum: \FormField[2cm] \\
& Unterschrift: \FormField[4cm] \\
\bottomrule
\end{longtable}

}% Ende der footnotesize-Gruppe

\textbf{Typische Szenarien:}

\textbf{Ransomware-Angriff:}
\begin{itemize}
  \item Verschlüsselung der Praxis-IT • Fokus: Isolation, Backup-Wiederherstellung, Meldung
\end{itemize}

\textbf{Geräteverlust/Diebstahl:}
\begin{itemize}
  \item MacBook mit Patientendaten gestohlen • Fokus: Meldung, Betroffeneninformation
\end{itemize}

\textbf{Netzwerkausfall:}
\begin{itemize}
  \item Internet/Telefon ausgefallen • Fokus: Alternative Kommunikation, Wiederanlauf
\end{itemize}

\BestandteilZehnJahre

\Legende \fbox{\cmark} = Erfolgreich durchgeführt, \fbox{\xmark} = Verbesserungsbedarf, \q = Noch nicht bearbeitet
