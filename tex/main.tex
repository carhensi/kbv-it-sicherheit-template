% PDF/A-3u + optional PDF/UA-1 Compliance
% Zwei Varianten: Standard (kompakt) vs. Accessible (mit Tagging)
% Build: make build-standard / make build-accessible

% Switch für Build-Varianten (wird per CLI gesetzt: \def\accessible{})
\newif\ifaccessible
\ifdefined\accessible\accessibletrue\else\accessiblefalse\fi

\ifaccessible
  % Accessible-Version: PDF/A-3u + PDF/UA-1 mit Tagging (article statt scrartcl)
  \DocumentMetadata{
    pdfversion  = 1.7,
    pdfstandard = {a-3u, ua-1},
    lang        = de-DE,
    testphase   = {phase-II,phase-III}
  }
  \PassOptionsToClass{11pt,twoside}{article}
  \documentclass[ngerman]{article}
\else
  % Standard-Version: PDF/A-3u ohne Tagging-Infrastruktur (scrartcl)
  \RequirePackage{pdfmanagement}
  \SetKeys[document/metadata]{
    pdfversion  = 1.7,
    pdfstandard = a-3u,
    lang        = de-DE
  }
  \PassOptionsToClass{11pt}{scrartcl}
  \documentclass[ngerman,twoside]{scrartcl}
\fi

% Tagging-Setup für Accessible-Version (nach \documentclass!)
\ifaccessible
  \tagpdfsetup{
    activate/all,         % Korrekter Name (nicht deprecated activate-all)
    role/map-tags=pdf,    % Standard-PDF-Tags für alle LaTeX-Strukturen
    para/maintag = Div    % /Div statt /Part für Absatz-Wrapper (PAC-freundlicher)
  }
\fi

% Flag für Hauptdokument vs. Arbeitsvorlagen
\newif\ifmaindoc
\maindoctrue

% Flag für Sample-Build (wird per CLI gesetzt: \def\samplebuild{})
\newif\ifsamplebuild
\ifdefined\samplebuild\samplebuildtrue\else\samplebuildfalse\fi

% Konfiguration und Metadaten
\input{config/metadata}

% Pakete
% Tabellen-Pakete
\usepackage{makecell}

% Layout und Typografie
\usepackage[main=ngerman]{babel}
% LuaLaTeX nutzt automatisch OpenType-Fonts ohne fontspec
\usepackage{microtype}

% Symbol-Fallback für Unicode-Zeichen (PDF/A-2u)
\usepackage{fontspec}
\newfontfamily\symbolsfont{DejaVu Sans}[Scale=MatchLowercase]
\usepackage{newunicodechar}
\newunicodechar{✓}{{\symbolsfont ✓}} % U+2713
\newunicodechar{✗}{{\symbolsfont ✗}} % U+2717
\newunicodechar{□}{{\symbolsfont □}} % U+25A1
\newunicodechar{☑}{{\symbolsfont ☑}} % U+2611

% Seitenlayout (print-optimiert)
\usepackage[a4paper,inner=28mm,outer=22mm,top=22mm,bottom=26mm]{geometry}
\usepackage{setspace}
\usepackage[absolute,overlay]{textpos}
\setstretch{1.15}

% PDF und Hyperlinks
\usepackage[
    bookmarks=true,
    bookmarksopen=true,
    bookmarksnumbered=true,
    hidelinks,
    colorlinks=true,
    linkcolor=black,
    urlcolor=blue,
    citecolor=black,
    pdfstartview=FitH,
    pdfpagelayout=OneColumn,
    pdfa
]{hyperref}
\usepackage{hyperxmp} % richer PDF metadata (load after hyperref)

% Fix für hyperref PDF-String Warnungen
\pdfstringdefDisableCommands{%
  \def\\{ }% Zeilenumbruch als Leerzeichen in PDF-Strings
  \let\tel\@firstofone% \tel{123} wird zu 123 in PDF-Strings
}
\usepackage{bookmark}

% Tabellen und Daten
\usepackage{tabularx}
\usepackage{longtable}
\usepackage{booktabs}
\usepackage{csvsimple}
\usepackage{array}
\usepackage{adjustbox}

% Tabellenlinien für sauberen Druck optimieren
\setlength{\arrayrulewidth}{0.5pt}

% Grafiken und Diagramme
\usepackage{graphicx}
\usepackage{tikz}
\usetikzlibrary{arrows.meta,positioning,shapes,calc}
\usepackage[rgb]{xcolor}  % RGB-Modus für PDF/A-2u
\usepackage{colortbl}

% Listen und Formatierung - Standard LaTeX für besseres Tagging
\ifaccessible
  % Accessible: Standard-Listen ohne enumitem für korrektes Tagging
  \setlength{\topsep}{0pt}
  \setlength{\partopsep}{0pt}
  \setlength{\itemsep}{0pt}
  \setlength{\parsep}{0pt}
\else
  % Standard: enumitem für kompakte Listen
  \usepackage{enumitem}
  \setlist{
    topsep=0pt,
    partopsep=0pt,
    itemsep=0pt,
    parsep=0pt,
    before={\setlength{\parskip}{0pt}},
    after={}
  }
\fi

% Code und Listings
\usepackage{listings}
\lstset{
  basicstyle=\ttfamily\small,
  frame=single,
  columns=fullflexible,
  breaklines=true
}

% Seitenlayout
\ifaccessible
  % Accessible: fancyhdr statt scrlayer-scrpage
  \usepackage{fancyhdr}
\else
  % Standard: KOMA scrlayer-scrpage
  \usepackage{scrlayer-scrpage}
\fi
\usepackage{refcount}
\usepackage{pdflscape}

% Deutsche Typografie
\usepackage[babel,german=quotes]{csquotes}
\usepackage{xspace}

% VVT Table Macros
\usepackage{ragged2e}   % \RaggedRight in Tabellen

% Hilfsmakro: farbige Kopfzeile ohne Weißsaum
\newcommand{\VVTHeader}[2]{%
  \noalign{\global\aboverulesep=0pt\global\belowrulesep=0pt}%
  \specialrule{\heavyrulewidth}{0pt}{0pt}% oben (entspricht \toprule)
  \rowcolor{#1}\multicolumn{2}{c}{\textbf{#2}}\\
  \specialrule{\lightrulewidth}{0pt}{0pt}% unten (entspricht \midrule)
  \noalign{\global\aboverulesep=0.4ex\global\belowrulesep=0.65ex}%
}

% Tabellenspalten: links 26%, rechts 66% (Summe < 100%)
\newcolumntype{L}{>{\RaggedRight\arraybackslash}p{0.26\textwidth}}
\newcolumntype{R}{>{\RaggedRight\arraybackslash}p{0.66\textwidth}}


% PDF/UA: DisplayDocTitle aktivieren
\hypersetup{
  pdfdisplaydoctitle=true,
  pdftitle={IT-Sicherheitsdokumentation nach §390 SGB V},
  pdfauthor={\PraxisInhaberin},
  pdfa=true,              % PDF/A-Modus
  hidelinks=true,         % Entfernt Link-Boxen
  unicode=true,           % Unicode-Support
  linktoc=all             % Titel UND Seitenzahl klickbar
}

% Deutsche Referenz-Namen für autoref
\renewcommand{\sectionautorefname}{Kapitel}
\renewcommand{\subsectionautorefname}{Abschnitt}
\renewcommand{\subsubsectionautorefname}{Unterabschnitt}
\renewcommand{\paragraphautorefname}{Absatz}
\renewcommand{\figureautorefname}{Abbildung}
\renewcommand{\tableautorefname}{Tabelle}

% Styling und Makros
\usepackage{style/theme}
\usepackage{style/macros}
\usepackage{style/signatures}
% Spaltentypen für professionelle Tabellen
% P-Spaltentyp für bessere Tabellen (linksbündig mit fester Breite)
\newcolumntype{P}[1]{>{\raggedright\arraybackslash}p{#1}}

% Y-Spaltentyp für sehr schmale Spalten
\newcolumntype{Y}{>{\centering\arraybackslash}p{1cm}}


% KOMA-Script Optionen
\AccessibleSwitch{
  % Accessible: Standard article - keine KOMA-Optionen
  \setlength{\parskip}{0.5\baselineskip}
  \setlength{\parindent}{0pt}
}{
  % Standard: KOMA-Script Optionen
  \KOMAoptions{parskip=half}
}

% Optional Git-Version aus Repository (falls vorhanden, als Fallback)
\IfFileExists{gitHeadInfo.gin}{%
  \usepackage[dirty]{gitinfo2}%
  \newcommand{\Version}{\DocumentVersion-\gitAbbrevHash}%
}{%
  \newcommand{\Version}{\DocumentVersion}%
}

% TOC-Seitenzahl-Box vergrößern für Anhang-Nummern (A-1, A-2, etc.)
\makeatletter
\renewcommand{\@pnumwidth}{2em}  % Standard: 1.55em
\makeatother

\title{IT-Sicherheitsdokumentation nach §~390 SGB~V\\ \large Praxis
für Psychotherapie \PraxisInhaberin}
\author{Praxisinhaberin: \PraxisInhaberin}
\date{\DocumentDate}

% Kopf-/Fußzeile konfigurieren
\ifaccessible
  % Accessible: fancyhdr statt scrlayer-scrpage
  \usepackage{fancyhdr}
  \pagestyle{fancy}
  \fancyhf{}
  \fancyhead[LE,RO]{\PraxisInhaberin}
  \fancyhead[LO,RE]{IT-Sicherheitsdokumentation}
  \fancyfoot[C]{%
    \ifmaindoc
    Seite \thepage{} von~\pageref*{LastPage} \quad | \quad Version \Version\TemplateAttribution%
    \else
    \thepage{} \quad | \quad Version \Version\TemplateAttribution%
    \fi
  }
  \renewcommand{\headrulewidth}{0.4pt}
\else
  % Standard: KOMA scrlayer-scrpage
  \clearpairofpagestyles
  \ihead{Praxis für Psychotherapie \PraxisInhaberin}
  \ohead{IT-Sicherheitsdokumentation}
  \cfoot{%
    \ifmaindoc
    Seite \thepage{} von~\pageref*{LastPage} \quad | \quad Version \Version\TemplateAttribution%
    \else
    \thepage{} \quad | \quad Version \Version\TemplateAttribution%
    \fi
  }
  \setkomafont{pageheadfoot}{\normalfont\small}
  \KOMAoptions{headsepline=0.4pt}
  \pagestyle{scrheadings}
\fi

\begin{document}
\raggedbottom

% Trennungsregeln für die wichtigsten technischen Begriffe

\hyphenation{
  IT-Si-cher-heits-kon-zept
  IT-Si-cher-heits-do-ku-men-ta-tion
  Back-up-Wie-der-her-stel-lung
  Cloud-An-wen-dun-gen
  App-Be-rech-ti-gun-gen
  WLAN-Si-cher-heit
  Fire-wall-Kon-fi-gu-ra-tion
  Ver-schlüs-se-lung
  Da-ten-schutz
  Si-cher-heits-vor-fall
  Da-ten-ka-te-go-rien
  arzt-di-rekt
  Arzt-Di-rekt
  Zoll-soft
  Dienst-leis-ter
  To-me-do
  Netz-ma-nage-ment-Au-then-ti-sie-rung
  Ma-nage-ment-Zu-griff
  Netz-kom-po-nen-ten
  Gäs-te-WLAN
  Netz-werk-tren-nung
  Pa-ti-en-ten-da-ten
  Cloud-Diens-ten
  öf-fent-li-chen
  Pra-xis-ge-mein-schaft
  Time-Ma-chi-ne
}


\begin{titlepage}
  \centering

  \praxislogo

  \textbf{\LARGE IT-Sicherheitsdokumentation}\\[2mm]
  {\large nach §~390 SGB~V}\\[8mm]
  {\Large \PraxisName}

  \vspace{2cm}

  % Praxis-Info in schöner Box
  \ifaccessible
    % Accessible: Einfache Struktur statt tikzpicture
    \begin{center}
      {\large Praxisinhaberin: \PraxisInhaberin}\\[0.3cm]
      \PraxisVollAdresse\\
      Telefon: \PraxisTelefon\\
      E-Mail: \href{mailto:\PraxisEmail}{\PraxisEmail}\\
      \url{\PraxisWebsite}
    \end{center}
  \else
    % Standard: TikZ-Box
    \begin{tikzpicture}
      \node[rectangle, draw=InfoBoxBorder, fill=InfoBoxFill, rounded corners=5pt,
      text width=10cm, align=center, inner sep=1cm] {
        {\large Praxisinhaberin: \PraxisInhaberin}\\[0.3cm]
        \PraxisVollAdresse\\
        Telefon: \PraxisTelefon\\
        E-Mail: \href{mailto:\PraxisEmail}{\PraxisEmail}\\
        \url{\PraxisWebsite}
      };
    \end{tikzpicture}
  \fi

  \vfill

  % Versions-Info am unteren Rand mit Klassifizierung
  \ifaccessible
    % Accessible: Einfacher Text statt tikzpicture
    \Label{Stand} \DocumentDate\\
    \Label{Gültig bis} \ValidUntil \quad | \quad \Label{Klassifizierung} \DokumentKlassifizierung{}\\
    \Label{Nächste Überprüfung} \NextReview
  \else
    % Standard: TikZ-Box
    \begin{tikzpicture}
      \node[rectangle, draw=VersionBoxBorder, fill=VersionBoxFill, rounded corners=3pt,
      text width=12cm, align=center, inner sep=0.8cm] {
        \Label{Stand} \DocumentDate\\[0.2cm]
        \Label{Gültig bis} \ValidUntil \quad | \quad \Label{Klassifizierung} \DokumentKlassifizierung{}\\[0.2cm]
        \textbf{Nächste Überprüfung:} \NextReview
      };
    \end{tikzpicture}
  \fi

  \vspace{1cm}
\end{titlepage}

\clearpage
\thispagestyle{empty}
\mbox{}

\clearpage
% TOC nur bis subsection (Level 2) für bessere Übersichtlichkeit
% subsubsection würde TOC zu detailliert machen
\setcounter{tocdepth}{2}
\tableofcontents

\cleardoublepage

\section{Einleitung und Zweck}

\subsection{Zweck der Dokumentation}

Diese IT-Sicherheitsdokumentation wurde gemäß §~390 SGB~V erstellt und dient dem Nachweis
der ordnungsgemäßen Umsetzung der gesetzlich vorgeschriebenen
IT-Sicher\-heits\-maß\-nah\-men in der \PraxisName.

\textbf{Praxisprofil:}
\begin{itemize}
  \item \textbf{Praxisgröße:} \PraxisGroesse
  \item \textbf{Fachrichtung:} \PraxisFachrichtung
  \item \textbf{IT-Infrastruktur:} \PraxisITInfrastruktur
\end{itemize}

\textbf{Rechtsgrundlage:} KBV-Richtlinie nach §~390 SGB~V, Version 1.1 (ab
01.10.2025). Pflicht zur jährlichen Evaluation gem. Abschnitt A.IV letzter Abs.\ (siehe
\cref{sec:vorlage-eigenpruefung}).

Die Dokumentation richtet sich an Prüfer der \KVLang{}
(\KV) und andere befugte Kontrollinstanzen zur Überprüfung der Einhaltung der
IT-Sicherheitsrichtlinie.

\subsection{Rechtliche Grundlagen}

\begin{itemize}
  \item \S~390 SGB~V – IT-Sicherheit in der vertragsärztlichen Versorgung und
    –psychotherapeutischen Versorgung
  \item Anlage~1 zur IT-Sicherheitsrichtlinie (Richtlinie über die Anforderungen an die
    IT-Sicherheit)
  \item Art.~32 DSGVO – Sicherheit der Verarbeitung
  \item \S\S~203, 204 StGB – Verletzung von Privatgeheimnissen
  \item Berufsordnung der \KammerLang
\end{itemize}

\subsection{Dokumentenerstellung}

Diese Dokumentation wird regelmäßig aktualisiert und versioniert. Änderungen werden
nachvollziehbar dokumentiert und sicher archiviert. Das private Repository ist verfügbar
unter: \url{\GitHubRepoDoc} (Zugang nur für autorisierte Personen).

\textbf{Hinweis zur Druckversion:} Bei geringfügigen Änderungen wird die Druckversion
nicht neu erstellt. Die jeweils aktuelle digitale Fassung ist maßgeblich und im
Repository hinterlegt.

Sicherheitsmerkmale: SHA256-Checksummen für Integrität, versionierte
Änderungsnachverfolgung, sichere digitale Archivierung.

\subsection{Geltungsbereich}

Diese Dokumentation umfasst alle IT-Systeme, Netzwerke und Prozesse der
\PraxisName, die zur Verarbeitung von Patientendaten oder anderen
schützenswerten Informationen eingesetzt werden.

\textbf{Räumlicher Geltungsbereich:} Praxisräume \PraxisVollAdresse

\textbf{Zeitlicher Geltungsbereich:} Gültig vom \DocumentDate{} bis \ValidUntil

\subsection{Aufbau der Dokumentation}

Die Dokumentation ist thematisch strukturiert und umfasst folgende Hauptbereiche:

\begin{itemize}
  \item \textbf{IT-Sicherheitskonzept} (\cref{sec:it-sicherheit}) -- Allgemeine
    Sicherheitsprinzipien, Personal, Netzwerk, Updates und Endgeräte-Sicherheit
  \item \textbf{E-Mail und Kommunikation} (\cref{sec:kommunikation}) -- Sichere
    E-Mail-Konfiguration, Spam-Behandlung und Website-Sicherheit
  \item \textbf{Telematikinfrastruktur} (\cref{sec:ti}) -- TI-Komponenten und
    sichere Anbindung
  \item \textbf{Mobile Geräte} (\cref{sec:mobile-geraete}) -- Umfassende
    Richtlinien für iPhone/iPad inkl. Apps und Datenverarbeitung
  \item \textbf{Mobile Device Management} (\cref{sec:mdm}) -- Zentrale
    Geräteverwaltung, Sicherheitsrichtlinien und Compliance
  \item \textbf{Wechseldatenträger} (\cref{sec:wechseldatentraeger}) --
    Verschlüsselung, Mitnahme-Regelungen und sichere Entsorgung
  \item \textbf{Verschwiegenheit und externe Dienstleister}
    (\cref{sec:verschwiegenheit}) -- AVV und Vertraulichkeitsvereinbarungen
  \item \textbf{Notfall- und Incident-Management} (\cref{sec:notfall}) --
    Backup-Konzept und Wiederherstellungsverfahren
  \item \textbf{Technische und Organisatorische Maßnahmen} (\cref{sec:toms}) --
    DSGVO Art.\ 32 Compliance
  \item \textbf{Rechte- und Rollenmatrix} (\cref{sec:rechte-rollenmatrix}) --
    Systematische Zugriffsberechtigung auf alle IT-Systeme
  \item \textbf{Referenzdokumente} -- Netzplan, Geräteliste, AVV-Register,
    Verzeichnis von Verarbeitungstätigkeiten (VVT)
  \item \textbf{Anhänge} -- IT-Notfall-Checkliste, Selbstverpflichtungen,
    Verschwiegenheitserklärungen, KBV-Compliance-Mapping, Changelog
\end{itemize}

\textbf{Arbeitsmappe (getrennt von der Dokumentation):} Die Arbeitsmappe enthält
regelmäßig zu füllende Arbeitsblätter: Restore-Tests (mindestens jährlich, angestrebt halbjährlich),
TOMs-Prüfung (jährlich/quartalsweise) und IT-Sicherheitsevaluation (jährlich).
Diese werden kontinuierlich gepflegt, während die Hauptdokumentation als
Regelwerk weitgehend stabil bleibt.

Jedes Kapitel behandelt die jeweiligen Anforderungen der IT-Sicherheitsrichtlinie
vollständig und praxisnah. Querverweise ermöglichen eine themenübergreifende Navigation.

\textbf{Hinweis für Prüfer:} Diese Dokumentation muss nicht linear gelesen werden.
Das \textbf{KBV-Compliance-Mapping} (\cref{sec:kbv-compliance}) ermöglicht die gezielte
Prüfung einzelner Anforderungen: Jede Anforderung der Anlage~1, 2 und 5 ist dort
mit Erfüllungsstatus und Kapitelreferenz aufgeführt. So kann z.\,B. direkt
\enquote{Anlage~1 Nr.~21 -- Datensicherung} nachgeschlagen und im referenzierten Kapitel
vertieft werden.

\textbf{Arbeitsmappe:} Ausgefüllte Vorlagen liegen in der separaten Arbeitsmappe
(Arbeitsvorlagen, On-/Offboarding-Vorlagen, Protokolle, regelmäßige Checks).

\subsection{Verantwortlichkeiten}

\textbf{Gesamtverantwortung:} \Gesamtverantwortung

\textbf{IT-Sicherheitsbeauftragte:} \ITSicherheitsbeauftragte

\textbf{IT-Ansprechpartner:} \ITAnsprechpartner{} (externer IT-Ansprechpartner)

Die \PraxisInhaberBezeichnung{} trägt die Gesamtverantwortung für die Umsetzung und Einhaltung aller
IT-Sicher\-heits\-maß\-nah\-men gemäß §~390 SGB~V.

\subsection{Datenschutz-Information für Patienten}

Die Datenschutz-Information nach Art. 13 DSGVO ist in der Praxis ausgehängt und
online unter \url{https://\PraxisDomain/#datenschutz} abrufbar. Änderungen werden
versioniert dokumentiert und Patienten bei wesentlichen Änderungen informiert.

\section{IT-Sicherheitskonzept}
\label{sec:it-sicherheit}

\subsection{Praxisdaten und Verantwortlichkeiten}

\begin{tabular}{p{5cm} p{7cm}}
  \toprule
  \textbf{Praxis} & {\def\\{\newline}\PraxisNameMitUmbruch} \\
  & \PraxisAdresse \\
  & \PraxisOrt \\
  & Tel.: \PraxisTelefon \\[0.5em]

  \textbf{\PraxisInhaberBezeichnung{}} & \PraxisInhaberin \\
  & \ITAnsprechpartnerAdresse \\
  & Tel.: \PraxisInhaberinPrivatTelefon \\[0.5em]

  \textbf{IT-Ansprechpartner} & \ITAnsprechpartner \\
  & Tel.: \ITAnsprechpartnerTelefon \\[0.5em]

  \textbf{Passwortmanagement} & \PasswortManager{} (Notfallzugang geregelt) \\
  \bottomrule
\end{tabular}

\textbf{\PasswortManager{} Notfallzugang:} Emergency Kit sicher verwahrt: \NotfallkitVerwahrung.
Physischer Zugriff ist sowohl für die \PraxisInhaberBezeichnung{} als auch den IT-Ansprechpartner
jederzeit eigenständig möglich.
\NotfallVollmachtPerson{} ist als Emergency Contact hinterlegt und
kann nach einer Wartezeit von einer Woche Zugriff auf den Praxis-Vault anfordern.

\subsection{Personal}
\label{sec:personal}

\textbf{Mitarbeitende:} In der Praxis sind keine Mitarbeitenden beschäftigt. Die
\PraxisInhaberBezeichnung{} führt alle Tätigkeiten eigenverantwortlich durch.

\textbf{Einstellungsverfahren:} Bei zukünftigen Einstellungen wird besonders auf
Vertrauenswürdigkeit geachtet, einschließlich der Prüfung vorliegender Arbeitszeugnisse
und Referenzen. Alle Angaben werden auf Glaubhaftigkeit kontrolliert.

\textbf{Einarbeitung:} Bei zukünftigen Einstellungen erfolgt systematische Einarbeitung
neuer Mitarbeitender nach strukturierter Checkliste (siehe \cref{sec:vorlage-einarbeitung}).
Dies umfasst IT-Sicherheitsschulung, Kontenerstellung, Arbeitsplatzeinweisung und
Verständnisprüfung aller Sicherheitsrichtlinien.

\textbf{Weggang von Mitarbeitenden:} Strukturiertes Offboarding nach Checkliste
(siehe \cref{sec:vorlage-offboarding}) mit vollständiger Rückgabe aller Unterlagen,
Deaktivierung von Zugängen und Betonung fortdauernder Verschwiegenheitspflichten.

\textbf{Externe Dienstleister:} IT-Support erfolgt durch \PVSHersteller{} (PVS-Hersteller) und IT-Ansprechpartner \ITAnsprechpartner. Beide sind über Verschwiegenheitserklärungen für externe Dienstleister verpflichtet (siehe \cref{sec:anhang-verschwiegenheit}). Zugangsberechtigungen werden restriktiv vergeben und bei Beendigung der Zusammenarbeit entzogen.

\textbf{Vertrauenswürdigkeit:} Die Auswahl externer Dienstleister erfolgt nach
Vertrauenswürdigkeit und fachlicher Qualifikation. Alle externen Zugriffe erfolgen nur
nach vorheriger Terminabsprache und unter Aufsicht.

\textbf{Beaufsichtigung:} Externes Personal (z.\,B.\ IT-Techniker, Wartungspersonal) wird
in sicherheitsrelevanten Bereichen kontinuierlich beaufsichtigt. Zugangsberechtigungen
werden so restriktiv wie möglich gehalten und nach Arbeitsende sofort entzogen.

\subsection{Sensibilisierung und Schulung}
\label{sec:sensibilisierung}

\textbf{Strukturierte Selbstschulung:} Die \PraxisInhaberBezeichnung{} hält sich über IT-Sicherheit
durch dokumentierte Weiterbildung auf dem Laufenden:
\begin{itemize}
  \item KV-Rundschreiben und -Newsletter
  \item \PVSName{}-Newsletter
  \item BSI für Bürger: IT-Sicherheit im Homeoffice und Praxis
  \item BSI-Grundschutz-Kompendium: Bausteine für kleine Unternehmen
  \item KBV-Praxisnachrichten: Aktuelle IT-Sicherheitswarnungen und -empfehlungen
  \item Fachspezifische Fortbildungen zu IT-Sicherheit und Datenschutz
\end{itemize}

\Dokumentation Selbstschulungsmaßnahmen werden im Kalender mit Datum, Quelle und
Thema dokumentiert und sind dadurch nachvollziehbar. Jährlich wird mindestens eine
strukturierte Schulung zu IT-Sicherheit durchgeführt (mindestens 4 Stunden gemäß §390 SGB V);
ergänzend werden quartalsweise Kurzmodule angestrebt (siehe \cref{sec:vorlage-schulungsplan}).

\textbf{KBV-Ressourcen und Fortbildungen:}
\begin{itemize}
  \item \textbf{IT-Sicherheit Übersicht:} \\
    \url{https://www.kbv.de/praxis/digitalisierung/it-sicherheit}
  \item \textbf{KBV-Hub (Musterdokumente, FAQs):} \\
    \url{https://hub.kbv.de/display/itsrl}
  \item \textbf{Fortbildungsportal:} \\
    \url{https://www.kbv.de/praxis/tools-und-services/fortbildungsportal}
  \item Basis-Schulung für MFA (PDF im KBV-Hub)
  \item Phishing-Schulung für MFA (PDF im KBV-Hub)
  \item IT-Sicherheit in der Praxis für Niedergelassene (PDF im KBV-Hub)
  \item Zertifizierte Online-Fortbildungen (bis zu 6 Fortbildungspunkte, kostenfrei)
\end{itemize}

\textbf{Sensibilisierung:} Regelmäßige Überprüfung der Sicherheitsmaßnahmen durch
Eigenprüfung (siehe \cref{sec:vorlage-eigenpruefung}). Bei Sicherheitsvorfällen
erfolgt unverzügliche Meldung an den IT-Ansprechpartner.

\subsection{Netzwerksicherheit}
\label{sec:netzwerk}

\textbf{Router und Firewall:} \RouterModell{} mit sicherer Grundkonfiguration (Details
siehe Netzplan~\ref{sec:anhang-netzplan}). Automatische Firmware-Updates, geänderte
Standard-Passwörter, keine Portfreigaben.

\textbf{Firewall-Begründung (Anlage~1 Nr.~11):} Die \RouterModell{} erfüllt als
Stateful-Packet-Inspection-Firewall die KBV-Anforderungen zum Netzübergangsschutz.

Die \RouterModell{} bietet:
\begin{itemize}
  \item Stateful Packet Inspection (SPI)
  \item NAT als zusätzliche Schutzschicht
  \item Automatische Firmware-Updates
  \item Keine Portfreigaben konfiguriert
  \item UPnP deaktiviert
\end{itemize}

\textbf{Firewall-Architektur (Anlage 1 Nr. 11):} Die Praxis implementiert eine
risikobasierte Firewall-Strategie. Der \RouterModell{} mit Stateful Packet Inspection
erfüllt die KBV-Anforderungen vollständig:

\begin{itemize}
  \item Stateful Packet Inspection blockiert unerwünschte Verbindungen
  \item NAT bietet zusätzliche Schutzschicht durch IP-Maskierung
  \item Extern gehostetes TI-Gateway eliminiert lokale Angriffsfläche (siehe \cref{sec:ti})
  \item Automatische Firmware-Updates gewährleisten aktuellen Schutzstand
\end{itemize}

\textbf{Compliance-Bewertung:} Die \KV-FAQ empfiehlt Hardware-Firewall oder
``Einsatz des Konnektors im Reihenbetrieb''. Da das TI-Gateway extern gehostet
ist, entfällt die Reihenschaltung. Die gewählte Architektur entspricht dem
Bedrohungsmodell und übertrifft die Mindestanforderungen für Einzelpraxen
ohne exponierte Dienste. Jährliche Neubewertung erfolgt im Rahmen der
IT-Sicherheitsevaluation.

\textbf{Netzmanagement-Authentisierung:} Der Management-Zugriff auf Netz\-kom\-po\-nen\-ten (\RouterKurz{}) erfolgt ausschließlich über sichere Authentisierung mit starken, individuellen Passwörtern. Standard-Zugangsdaten wurden geändert und werden sicher in \PasswortManager{} verwaltet.

\textbf{WLAN-Sicherheit:} WPA2\slash WPA3-Ver\-schlüs\-se\-lung, automatische Abschaltung nachts
(\WLANAbschaltung).

\textbf{Gäste-WLAN:} Separater WLAN-Gastzugang \enquote{\WLANGastName} für Patientinnen mit
WPA2\slash WPA3-Ver\-schlüs\-se\-lung und starkem Passwort. Vollständige Netz\-werk\-tren\-nung vom
Praxisnetz, Gastgeräte sind isoliert voneinander. Automatische Abschaltung nachts
(\WLANAbschaltung).

\textbf{Logging und Überwachung (freiwillig, Anlage~2 Nr.~1+2):} Systemprotokoll
aktiviert. Protokolle (\RouterKurz{}, macOS, \PVSName{}) werden anlassbezogen (z.\,B.\ bei
Sicherheitsvorfällen) ausgewertet. Im Rahmen der jährlichen Eigenprüfung erfolgt
zusätzlich eine stichprobenartige Sichtung zur Verifizierung.

\textbf{Monitoring-Strategie (Anlage 3):} Die Praxis implementiert eine
praxisgerechte Überwachungsarchitektur statt zentralem SIEM:

\begin{itemize}
  \item Dezentrale Logs (Router, macOS, PVS) für gezielte Analyse
  \item Honeypots (Canary Tokens) für Anomalie-Erkennung (siehe \cref{sec:honeypots})
  \item Quartalsweise Stichproben-Auswertung in TOMs-Prüfung
  \item Ereignisbasierte Tiefenanalyse bei Sicherheitsvorfällen
\end{itemize}

\textbf{Architektur-Vorteil:} Diese Lösung bietet effektive Überwachung ohne
SIEM-Overhead und entspricht der Einzelpraxis-Infrastruktur optimal.
SIEM-Systeme erfordern kontinuierliche Analyse-Kapazität und spezialisierte
Expertise, die in Einzelpraxen nicht wirtschaftlich darstellbar ist.
Bei Wachstum wird diese Strategie neu bewertet.

\subsection{Patch- und Änderungsmanagement}
\label{sec:patch}

\textbf{Verantwortlichkeiten:} Die \PraxisInhaberBezeichnung{} ist für die zeitnahe Installation von
Updates verantwortlich. Der IT-Ansprechpartner unterstützt bei kritischen Updates und Problemen.

\textbf{Update-Verfahren:}
\begin{itemize}
  \item \textbf{macOS:} Automatische kritische Updates und Minor-Updates aktiviert, Major-Updates nur in Absprache mit \PVSName{} und \PraxisInhaberBezeichnung{}
  \item \textbf{\PVSName{} PVS:} Automatische Updates außerhalb der Praxiszeiten konfiguriert, manuelle Updates nach Herstellerfreigabe binnen 14 Tagen. Automatische Backup-Erstellung vor jedem Update durch Server-Tools
  \item \textbf{Microsoft 365:} Automatische Updates aktiviert
  \item \textbf{\RouterKurz{}:} Automatische Firmware-Updates aktiviert
  \item \textbf{iOS/iPadOS:} Automatische Sicherheitsupdates aktiviert
\end{itemize}

\textbf{Identifizierung ausbleibender Updates:}
\begin{itemize}
  \item \textbf{macOS/iOS:} Systemeinstellungen → Softwareupdate (wöchentliche Kontrolle)
  \item \textbf{\MDMTool{}:} MDM-Dashboard zeigt Update-Status aller verwalteten Geräte
  \item \textbf{\PVSName{}:} Automatische Update-Benachrichtigungen im System
  \item \textbf{\RouterKurz{}:} Admin-Interface zeigt verfügbare Firmware-Updates
\end{itemize}

\textbf{End-of-Life-Strategie:} Geräte ohne verfügbare Sicherheitsupdates (EOL) werden
rechtzeitig durch aktuelle Modelle ersetzt oder durch zusätzliche Sicherheitsmaßnahmen
kompensiert. Austausch spätestens bei Hersteller-EOL oder ausbleibenden
Sicherheitsupdates; planmäßige Bewertung und ggf. Ersatz bis zur nächsten
Dokumentüberprüfung (\NextReview). Die Bewertung erfolgt im Rahmen der jährlichen
IT-Sicherheitsevaluation (siehe \cref{sec:vorlage-eigenpruefung}).

\Dokumentation Bei Problemen erfolgt eine Dokumentation und Wiederherstellung
über Backup-Restore.

\subsection{Endgeräte-Sicherheit}
\label{sec:endgeraete}

\textbf{Grundschutz:} Alle Endgeräte sind mit aktuellen Betriebssystemen und Virenschutz
(macOS XProtect) ausgestattet. Automatische Bildschirmsperre nach 5 Minuten Inaktivität.

\textbf{Integrierte Sicherheitsarchitektur (Anlage 1 Nr. 20):} Die Praxis nutzt
eine Defense-in-Depth-Strategie basierend auf dem macOS-Sicherheitsstack:

\begin{itemize}
  \item XProtect: Tägliche Malware-Signaturen von Apple
  \item Gatekeeper: Automatische Code-Signatur-Validierung
  \item SIP: Hardware-basierter Systemschutz
  \item Sandboxing: Anwendungsisolierung auf Kernel-Ebene
  \item Secure Boot: Firmware-Integritätsprüfung
\end{itemize}

\textbf{Technische Überlegenheit:} Diese integrierte Lösung übertrifft herkömmliche
Virenschutz-Software durch tiefe Systemintegration ohne Performance-Einbußen oder
Kompatibilitätskonflikte. Apple dokumentiert im Platform Security Guide die
Ausreichend dieser Architektur für Enterprise-Umgebungen. Dritt-Anbieter-Software
würde Kernel-Zugriff erfordern und potenzielle Angriffsfläche schaffen.

\textbf{Compliance-Nachweis:} Der Apple Platform Security Guide bestätigt, dass
\enquote{das System so konzipiert ist, dass zusätzliche Antivirus-Software nicht
erforderlich ist} (Quelle: \url{https://support.apple.com/guide/security/}).
Diese Architektur erfüllt die KBV-Anforderungen vollständig.

\textbf{macOS Security-Baseline (CIS Level~1):} Die Konfiguration orientiert sich am
CIS Apple macOS Benchmark (Level~1). Umgesetzte Maßnahmen: Festplattenverschlüsselung
(FileVault) mit sicher hinterlegtem Recovery-Key, macOS-Firewall aktiviert, Gatekeeper
und XProtect für Malware-Schutz, System Integrity Protection (SIP) aktiviert, Secure Boot
auf \enquote{Full Security} (Apple Silicon), USB-Zubehör erfordert explizite Freigabe,
alle Sharing-Dienste deaktiviert, Gastaccount deaktiviert, separater Admin-Account für
Systemverwaltung, automatisches Login deaktiviert, Passwort sofort nach Bildschirmschoner
erforderlich, Safari Auto-Open Downloads deaktiviert, Remote Management deaktiviert,
Standortdienste nur für \enquote{Wo ist?} (Fernlöschung bei Verlust).
Halbjährliche Überprüfung der Baseline-Konfiguration in den TOMs-Prüfungen.

\textbf{Zugangsschutz:} Starke Passwörter/Biometrie für alle Benutzerkonten. Keine
Gastkonten oder geteilte Accounts. Separater lokaler Admin-Account (\texttt{\AdminUser{}}) für
Systemverwaltung, tägliche Arbeit mit Standard-Benutzerrechten.

\textbf{Sperr-/Abmeldepflicht:} Nach Ende der Nutzung sofort Bildschirm sperren/abmelden
(macOS: Ctrl+Cmd+Q). Auto-Sperre bleibt Sicherheitsnetz (5 Min).

\textbf{Gruppenberechtigungen:} Datei- und Freigabeberechtigungen sind pro Personengruppe
(Praxisinhaber, externe Dienstleister) und pro Person individuell geregelt. Zugriffe
folgen dem Need-to-know-Prinzip.

\textbf{Verschlüsselung:} FileVault-Festplattenverschlüsselung und TLS-Datenübertragung
aktiviert. Details zu Wechseldatenträgern siehe \cref{sec:wechseldatentraeger}.

\textbf{Software-Beschränkung:} Installation nur aus vertrauenswürdigen Quellen (Mac App
Store, Hersteller-Websites). Keine P2P-Software oder unsichere Downloads.
Sicherheitsfeatures siehe macOS Security-Baseline oben.

\textbf{Mikrofon/Kamera-Schutz (Anlage~1 Nr.~18):} Mikrofon und Kamera sind in den
macOS-Systemeinstellungen grundsätzlich für alle Anwendungen deaktiviert. Aktivierung nur
bei konkretem Bedarf (z.\,B.\ Videosprechstunde) und anschließende Deaktivierung. Keine
permanenten Berechtigungen für nicht-medizinische Apps.

\textbf{Drucker-Sicherheit:} Brother DCP-L2660DW Laserdrucker mit folgenden Sicherheitsmaßnahmen:
\begin{itemize}
  \item \textbf{Standortschutz}: Drucker steht direkt neben Arbeitsplatz (physische Kontrolle)
  \item \textbf{Cloud-Print deaktiviert}: Keine Verbindung zu externen Print-Diensten
  \item \textbf{WLAN-Sicherheit}: WPA2\slash WPA3-ver\-schlüs\-sel\-te Verbindung zum Praxis-WLAN
  \item \textbf{Admin-PIN}: Gerätezugriff durch PIN geschützt
  \item \textbf{Kein PIN-Druck}: Nicht erforderlich (Drucker direkt am Arbeitsplatz $<$2m,
    Einzelpraxis, Patienten nicht allein im Raum, sofortige Entnahme, Druck nur bei
    Anwesenheit im Praxis-WLAN möglich)
  \item \textbf{Speicher}: Druckaufträge werden nicht dauerhaft im Gerät gespeichert
\end{itemize}

\subsection{Windows-Endgeräte}
\label{sec:windows}

\textbf{Nicht anwendbar:} In der Praxis werden ausschließlich Apple-Geräte eingesetzt.
Keine Windows-Systeme vorhanden (auch nicht virtualisiert).

\subsection{Internet- und Cloud-Anwendungen}
\label{sec:internet}

\textbf{Cloud-Nutzung:}
\begin{itemize}
  \item \textbf{iCloud Backup:} Verschlüsselte Backups für iOS/macOS-Geräte über Apple Business Manager (MDM-verwaltet, AVV vorhanden). Keine Gesundheitsdaten in iCloud Drive oder iCloud Mail.
  \item \textbf{Microsoft 365:} Nur lokale Office-Apps (Word, Excel), keine Cloud-Speicherung
  \item \textbf{Keine weiteren Cloud-Dienste} für Patientendaten
\end{itemize}

\textbf{Authentisierung:} Passwort-Richtlinie:
\begin{itemize}
  \item Mindestlänge: \PasswortMindestlaenge{} Zeichen (\PasswortManager{}-generiert)
  \item Keine Wiederverwendung
  \item 2FA wo möglich (iCloud, \PasswortManager{}, \HostingAnbieter{})
  \item Passwort-Rotation: Nur bei Verdacht auf Kompromittierung
  \item \PasswortManager{} Sicherheits-Dashboard Score: Ziel $\geq$\PasswortSecurityScore{} (quartalsweise Prüfung in TOMs, Abweichungen werden dokumentiert und bewertet)
\end{itemize}

\textbf{Datenschutz:} Patientendaten werden nicht in öffentlichen Cloud-Diensten gespeichert.
Gesundheits- und Sozialdaten werden nicht in Cloud-Speichern (iCloud Drive, OneDrive, etc.) abgelegt.

Kalender wird über \EmailAnbieter{} oder lokal verwaltet. Bei Kalender-Synchronisation werden
ausschließlich anonymisierte Patientenkürzel für die Terminorganisation verwendet.
Dabei werden keine Namen, Diagnosen oder medizinischen Inhalte übertragen.
Die Verarbeitung ist zweckgebunden und DSGVO-konform dokumentiert.

\textbf{Cloud-Compliance:} Soweit Sozial- oder Gesundheitsdaten im Wege des
Cloud-Computing verarbeitet werden, verfügt der Anbieter über ein aktuelles C5-Testat
entsprechend § 393 SGB V in Verbindung mit § 384 SGB V (\KIToolName{}, \PVSName{} TI-Gateway).
Alle Cloud-Dienste werden im Verzeichnis von Verarbeitungstätigkeiten (VVT, \cref{sec:vvt}) dokumentiert.

\subsection{Datensicherung}
\label{sec:datensicherung}

\textbf{\PVSName{}-Backup-Konzept:} \PVSName{} läuft lokal als Server und Client auf dem MacBook Air
(siehe Geräteliste \cref{sec:geraeteliste}). \PVSName{} erstellt automatisch primäre Backups aller
Patientendaten und Systemkonfigurationen in einem lokalen Backup-Verzeichnis auf der
System-SSD (FileVault-verschlüsselt).
Diese primären Backups werden anschließend über Time Machine auf verschlüsselte externe
SSD und verschlüsseltes NAS-Backup gesichert. Bei MacBook-Ausfall erfolgt Wiederherstellung
aus Backup gemäß RTO/RPO (siehe unten).

Das Backup-Konzept umfasst lokale Time Machine-Sicherung auf externe Festplatte sowie
Offsite-Backup auf NAS. Wöchentliche Vollsicherung, tägliche inkrementelle Sicherung.
Restore-Tests werden mindestens einmal jährlich, angestrebt halbjährlich (Januar/Juli)
sowie anlassbezogen durchgeführt und dokumentiert.

\textbf{Wiederanlaufziele:}
\begin{itemize}
  \item \textbf{RTO (Recovery Time):} $\leq$4h ab Bereitstellung der Ersatzhardware
    (Neugerät ggf.\ kurzfristig im Handel erhältlich; Gesamtwiederherstellung
    inkl.\ Hardwarebeschaffung im Worst Case bis zu 24h)
  \item \textbf{RPO (Recovery Point):} $\leq$24h (konservativ gerechnet)
\end{itemize}

\textbf{Verantwortlichkeit:} Die \PraxisInhaberBezeichnung{} ist für die ordnungsgemäße Durchführung
der Datensicherung, tägliche Backup-Kontrolle, Restore-Tests und Meldung bei Problemen verantwortlich.
Der IT-Ansprechpartner ist für die technische Konfiguration zuständig und
steht bei Fragen zur Verfügung.

\textbf{Tägliche Backup-Kontrolle:} Die \PraxisInhaberBezeichnung{} prüft täglich den Time Machine-Status
(Letztes Backup < 24h). \PVSName{} meldet automatisch bei nicht erfolgreichem Backup.
Bei Fehlern wird der IT-Ansprechpartner unverzüglich informiert.

Das Offsite-Backup wird verschlüsselt auf einem NAS am \BackupStandort{}
gespeichert. Durch die Verschlüsselung ist ein Zugriff auf Praxisdaten durch andere
Dritte ausgeschlossen. Auch NAS-Administratoren haben keinen Zugriff auf
die verschlüsselten Time Machine-Container ohne die Backup-Schlüssel. Die
Datenübertragung erfolgt lokal oder verschlüsselt per IPSec-VPN (\RouterKurz{} zu NAS).
VPN-Konfiguration und -Betrieb erfolgen durch den IT-Ansprechpartner.

Die Backup-Schlüssel für die verschlüsselten Time Machine-Container werden
ausschließlich von der \PraxisInhaberBezeichnung{} verwaltet und sicher in \PasswortManager{}
gespeichert. Das NAS steht gesichert am \BackupStandort{}.

\textbf{Backup-Aufbewahrung und Ausdünnung:} Time Machine dünnt Backups automatisch aus:
Stündliche Backups der letzten 24 Stunden, tägliche des letzten Monats, danach
wöchentliche. Bei Platzmangel werden die ältesten Backups automatisch gelöscht.
Gelöschte Patientendaten können daher in verschlüsselten Backups verbleiben, bis diese
turnusmäßig überschrieben werden. Dies ist datenschutzrechtlich zulässig, da die Backups
verschlüsselt, nicht aktiv zugänglich und ausschließlich für die Notfallwiederherstellung
bestimmt sind.

\subsection{Sichere Datenträgerentsorgung}
\label{sec:entsorgung}

Vor Entsorgung, Verkauf oder Weitergabe von IT-Geräten erfolgt sichere Löschung aller
Daten durch Überschreibung oder physische Zerstörung der Datenträger. Bei SSDs wird die
Herstellersoftware für Secure Erase verwendet. Festplatten werden mindestens dreifach
überschrieben oder mechanisch zerstört.

\Dokumentation Jede Entsorgung wird mit Gerät, Datum, Löschmethode und
verantwortlicher Person in der Geräteliste (siehe \cref{sec:geraeteliste})
dokumentiert.

\subsection{Aufbewahrungsplan für nicht-medizinische Daten}
\label{sec:aufbewahrungsplan}

Ergänzend zu den medizinischen Aufbewahrungspflichten (§ 630f BGB - 10 Jahre)
gelten für organisatorische und technische Daten folgende Aufbewahrungsfristen:

\begin{center}
\begin{tabular}{p{5cm}p{2.5cm}p{6cm}}
\toprule
\textbf{Datenart} & \textbf{Aufbewahrung} & \textbf{Rechtsgrundlage/Zweck} \\
\midrule
System-/Sicherheitslogs & 90 Tage & IT-Betrieb, Incident-Analyse \\
Organisatorische E-Mails & 12 Monate & Geschäftskorrespondenz \\
Patientenbezogene E-Mails & Sofort in PVS & Siehe \cref{sec:kommunikation} \\
Support-/Änderungsprotokolle & 24 Monate & Nachvollziehbarkeit, Wartung \\
Schulungsnachweise & 36 Monate & DSGVO-Rechenschaftspflicht \\
AVV/Vertragsunterlagen & Laufzeit + 3 Jahre & Verjährungsfristen \\
Website-Logs (\HostingAnbieter{}) & 30 Tage & Automatische Löschung \\
Backup-Daten & Entsprechend Ursprungsdaten & Zweckbindung \\
\bottomrule
\end{tabular}
\end{center}

\textbf{Löschprozess:} Automatische Löschung wo technisch möglich (\HostingAnbieter{}-Logs),
manuelle Löschung nach Fristablauf mit Dokumentation. Ausnahmen nur bei
rechtlichen Aufbewahrungspflichten oder laufenden Verfahren.

\textbf{Verweis:} Detaillierte Aufbewahrungsfristen siehe VVT (\cref{sec:vvt}).

\subsection{Patientenaufnahme und Datenschutz-Information}
\label{sec:patientenaufnahme}

\subsubsection{Datenschutz-Information}

Die Datenschutz-Information nach Art.\ 13 DSGVO ist in der Praxis ausgehängt und online
unter \url{\PraxisWebsiteDatenschutz} abrufbar. Änderungen werden
versioniert dokumentiert und Patienten bei wesentlichen Änderungen informiert.

\subsubsection{Digitaler Aufnahmeprozess}

Die Patientenaufnahme erfolgt vollständig digital im PVS mit minimalen, kompakten Formularen:
\begin{enumerate}
  \item Datenschutz-Information (Aushändigung/Online-Verweis)
  \item KI-Einwilligung (bei gewünschter KI-Dokumentation)
  \item Behandlungsvertrag und Anamnesebogen
\end{enumerate}

Alle Einwilligungen werden direkt im PVS dokumentiert und versioniert gespeichert.
Der Widerruf von Einwilligungen ist jederzeit mündlich oder schriftlich möglich
und wird sofort wirksam. Widerrufe werden im PVS dokumentiert.

\subsubsection{KI-gestützte Dokumentation}

\textbf{\KIToolName{} Integration:} Die Praxis nutzt \KIToolName{} für die
KI-gestützte Erstellung von Sitzungsnotizen und psychologischen Berichten. \KIToolName{}
ist C5-zertifiziert und DSGVO-konform (Details: \url{\KIToolSicherheitURL}).

\textbf{Patienteneinwilligung:} Vor der ersten KI-gestützten Sitzung wird eine explizite
schriftliche Einwilligung nach Art.\ 9 Abs.\ 2 lit.\ a DSGVO eingeholt. Die Einwilligung
erfolgt über die in \KIToolName{} integrierte, anpassbare Vorlage und wird im PVS dokumentiert.

\textbf{Datenverarbeitung:} Audioaufzeichnung und Transkription erfolgen temporär zur
Erstellung der Sitzungsnotizen. \KIToolName{} speichert keine Patientendaten dauerhaft.
Alle Daten werden nach der Verarbeitung automatisch gelöscht. Die generierten Protokolle
werden ausschließlich im lokalen PVS (\PVSName{}) gespeichert.

\textbf{Widerruf:} Patienten können die Einwilligung jederzeit ohne Angabe von Gründen
widerrufen. Bei Widerruf erfolgt die Dokumentation wieder manuell.

\subsubsection{Schweigepflichtentbindungen}

Für Berichte an andere Behandler oder Institutionen werden bei Bedarf spezifische
Schweigepflichtentbindungen eingeholt. Diese werden zweckgebunden, zeitlich begrenzt
und dokumentiert im PVS verwaltet. Auskünfte an Krankenkassen erfolgen nur im
gesetzlich vorgesehenen Rahmen (Antragsverfahren, PTV-Formulare).

\subsubsection{Aufbewahrung und Löschung}

\textbf{Aufbewahrungsfristen:} Patientendaten werden gemäß § 630f BGB für 10 Jahre nach
Behandlungsende aufbewahrt. KI-generierte Sitzungsnotizen unterliegen denselben Fristen
wie manuell erstellte Dokumentation.

\textbf{Löschkonzept:} Nach Ablauf der Aufbewahrungsfristen erfolgt sichere Löschung
aller Patientendaten aus dem PVS. In verschlüsselten Backups können gelöschte Daten
länger verbleiben (siehe \cref{sec:datensicherung}). Bei einer Wiederherstellung aus
Backup wird geprüft, ob zwischenzeitlich gelöschte Daten erneut zu löschen sind.

\subsubsection{Betroffenenrechte}

Patienten haben das Recht auf Auskunft, Berichtigung, Löschung und Datenübertragbarkeit
ihrer Daten. Anfragen werden binnen 30 Tagen bearbeitet. Der Prozess ist in der
Vorlage Betroffenenrechte (\cref{sec:vorlage-betroffenenrechte}) dokumentiert.

\textbf{Zusammenfassung:} Das IT-Sicherheitskonzept etabliert die technischen und
organisatorischen Grundlagen für den sicheren Praxisbetrieb. Aufbauend auf dieser
Infrastruktur behandelt das folgende Kapitel die sichere Kommunikation mit Patienten
und externen Partnern.

\section{E-Mail und Kommunikation}
\label{sec:kommunikation}

\subsection{E-Mail-Sicherheit}

Als E-Mail-Anbieter wird \EmailAnbieter{} genutzt (Custom Domain \PraxisDomain).

\textbf{Wichtig:} \EmailAnbieter{} wird ausschließlich für organisatorische Kommunikation
verwendet (keine Gesundheitsdaten). AVV mit \EmailAnbieterFirma{}
vom 30.11.2025 abgeschlossen (siehe AVV-Register Anhang).

\textbf{Mail-Gateway-Architektur (Anlage 3):} Die Praxis nutzt \EmailAnbieter{}
als hochverfügbare Gateway-Lösung mit überlegenen Sicherheitsfeatures:

\begin{itemize}
  \item 24/7 Spam-/Virenfilter mit Machine Learning
  \item Automatische DKIM-Signierung und SPF-Validierung
  \item TLS 1.3 Transportverschlüsselung mit Perfect Forward Secrecy
  \item 99.9\% SLA-garantierte Verfügbarkeit
\end{itemize}

\textbf{Architektur-Vorteil:} Diese Managed-Service-Architektur übertrifft lokale
Gateway-Lösungen in Sicherheit, Verfügbarkeit und Wartungsaufwand. Lokale
Mail-Gateways erfordern kontinuierliche Updates, Monitoring und Expertise,
die in Einzelpraxen nicht wirtschaftlich darstellbar sind.

Digitale Signatur aller ausgehenden E-Mails (Authentizität, Integrität) per S\slash
MIME-Si\-gna\-tur standardmäßig aktiv.
Ende-zu-Ende-Verschlüsselung per S\slash MIME sofern ein gültiges Empfängerzertifikat
vorliegt; andernfalls erfolgt der Versand transportverschlüsselt (TLS). Patienten können
unverschlüsselt Kontakt aufnehmen.

\textbf{E-Mail-Policy für Patienten:} Sensible Inhalte (Diagnosen, Therapieinhalte,
Befunde) werden nicht per E-Mail übermittelt. Sichere Kommunikation erfolgt über
\PatKomTool{}-App, KIM (mit anderen Leistungserbringern, siehe \cref{sec:kim}) oder
persönlich in der Praxis.

\textbf{E-Mail-Signatur:} Alle ausgehenden E-Mails enthalten folgenden Hinweis:
\begin{quote}
\textit{Hinweis: Bitte keine medizinischen Inhalte per E-Mail.
Für sichere Kommunikation nutzen Sie \PatKomTool{} oder rufen Sie an.}
\end{quote}

\textbf{Arbeitsanweisung bei medizinischen E-Mail-Inhalten:} Erhält die Praxis
E-Mails mit medizinischen Inhalten (Diagnosen, Symptome, Befunde):
\begin{enumerate}
  \item Vorgang kurz im PVS protokollieren (Datum, Patient, \enquote{E-Mail erhalten, sicherer Kanal genutzt})
  \item E-Mail sofort löschen (nicht weiterleiten oder speichern)
  \item Patient telefonisch kontaktieren
  \item Inhalt über \PatKomTool{} oder persönlich anfordern
  \item Patienten über sichere Kommunikationswege informieren
\end{enumerate}

\textbf{Wichtig:} E-Mails mit Gesundheitsdaten werden nach Übernahme ins PVS in der
Mailbox gelöscht. Es erfolgt keine Ablage gesundheitsbezogener Inhalte in
iCloud-Mail, iCloud-Drive oder Kalendernotizen.

\textbf{Technische Umsetzung der E-Mail-Sicherheit:} Um diese organisatorischen
Maßnahmen technisch zu unterstützen, implementiert die Praxis folgende
Verschlüsselungs- und Authentifizierungsverfahren:

\textbf{S/MIME-Zertifikat (nur Signatur):} Ausgehende E-Mails werden mit S/MIME
signiert (nicht verschlüsselt). Die Signatur bestätigt die Authentizität des Absenders.
\begin{itemize}
  \item \textbf{Aussteller}: \SMIMEIssuer{}
  \item \textbf{Gültigkeit}: \SMIMEValidity{}
  \item \textbf{Speicherung}: \PasswortManager{} + macOS Keychain
\end{itemize}

\textbf{S/MIME-Rotation:}
\begin{enumerate}
  \item Kalender-Erinnerung 60 Tage vor Ablauf (\PraxisInhaberBezeichnung{})
  \item Neues Zertifikat beim Anbieter bestellen (IT-Ansprechpartner)
  \item Installation in macOS Keychain und Test (IT-Ansprechpartner)
  \item Altes Zertifikat in \PasswortManager{} archivieren
\end{enumerate}

Die Domain-Sicherheit wird durch folgende technisch implementierte Maßnahmen
gewährleistet (öffentlich prüfbar via DNS-Abfrage):
\begin{itemize}
  \item \textbf{DNSSEC} zur Authentizität von DNS-Einträgen
  \item \textbf{CAA-Records} zur Kontrolle der Zertifikatsausstellung
  \item \textbf{Registrar Lock} verhindert unbefugte Domain-Transfers
\end{itemize}

\subsubsection{E-Mail-Sicherheitsstandards}
Für die Domain \texttt{\PraxisDomain} sind folgende Sicherheitsstandards aktiv:
\begin{itemize}
\item \textbf{SPF}: Autorisiert \EmailAnbieter{} als legitimen Mailserver
\item \textbf{DKIM}: Digitale Signierung durch \EmailAnbieter{}
\item \textbf{DMARC}: Policy \texttt{p=reject} verwirft nicht-authentifizierte E-Mails
\item \textbf{MTA-STS}: Erzwingt verschlüsselten Transport (TLS 1.2+)
\end{itemize}

\textbf{Implementierung:} SPF/DKIM-Signierung über \EmailAnbieter{}, DMARC mit Policy \texttt{p=reject}.

\textbf{DMARC-Report-Strategie:} Da ausschließlich \EmailAnbieter{} als Managed Service
genutzt wird (keine eigenen Mailserver), erfolgt die Sicherheitsüberwachung durch:
\begin{itemize}
  \item \EmailAnbieter{} überwacht automatisch Zustellbarkeit und Authentifizierung
  \item DMARC-Reports an \texttt{\PostmasterEmail} (halbjährliche Prüfung angestrebt)
  \item Halbjährliche Prüfung (angestrebt): DKIM/SPF-Authentifizierung, Zustellbarkeit, Auffälligkeiten
  \item Jährliche DNS-Validierung im Rahmen der IT-Evaluation (\cref{sec:vorlage-eigenpruefung})
  \item Bei Zustellproblemen: Analyse über MXToolbox oder dmarcian
\end{itemize}

DNS-Änderungen werden über \IaCTool{} verwaltet und versioniert
(siehe \cref{sec:website-sicherheit}).

\subsection{Spam-Behandlung}

Gemäß Anlage 1 Nr. 41 der IT-Sicherheitsrichtlinie werden Spam-E-Mails grundsätzlich
ignoriert und gelöscht. Verdächtige E-Mails werden konsequent gelöscht ohne Öffnen von
Links oder Anhängen aus unbekannten Quellen. Links in verdächtigen E-Mails werden nicht
angeklickt, Anhänge aus unbekannten Quellen nicht geöffnet. Bei Unsicherheit wird die
E-Mail an den IT-Ansprechpartner weitergeleitet, ansonsten sofort gelöscht.

\subsection{E-Mail-Client-Konfiguration}

Apple Mail ist sicherheitsorientiert konfiguriert: Externe Inhalte werden standardmäßig
unterdrückt, aktive Inhalte sind deaktiviert (Anlage 1 Nr. 40), und die automatische
Bildanzeige ist deaktiviert. E-Mail-Anhänge werden vor dem Öffnen durch die integrierten
macOS-Sicherheitsmechanismen (XProtect, Gatekeeper) geprüft.

\subsection{Website-Sicherheit}
\label{sec:website-sicherheit}

Die Praxis-Website wird bei \HostingAnbieter{} (DSGVO-konform mit AVV) gehostet und nutzt erzwungene
HTTPS-Verschlüsselung. Infrastructure as Code via \IaCTool{} für sichere und nachvollziehbare
Konfiguration (privates Repository: \url{\GitHubRepoWeb}).
Security-Header konfiguriert (u.a. HSTS, CSP, X-Content-Type-Options, Referrer-Policy);
jährliche Sichtprüfung (siehe \cref{sec:vorlage-eigenpruefung}).

\textbf{Wartung:} IT-Ansprechpartner ist für Wartung und Updates verantwortlich.
Da statisches HTML ohne Datenbank oder CMS, sind regelmäßige Updates nicht erforderlich.
Inhaltliche Änderungen erfolgen anlassbezogen.

\textbf{Anlage~1 Nr.~45–48 (nicht anwendbar):} Die Website bietet keine Logins oder
geschützten Ressourcen. Authentifizierung, WAF und Anti-Automation-Maßnahmen sind
daher nicht erforderlich - die statische Informationsseite benötigt nur die
implementierten Security-Header.

Es werden keine Cookies, Tracking- oder Analysetools eingesetzt.

Technisch notwendige Server-Log-Dateien (Browsertyp, Betriebssystem, Referrer, Hostname,
Uhrzeit der Anfrage, IP-Adresse) werden 30 Tage gespeichert und danach automatisch gelöscht.
Eine Zusammenführung mit anderen Datenquellen findet nicht statt. Rechtsgrundlage ist
Art.\ 6 Abs.\ 1 lit. f\ DSGVO\ (berechtigtes Interesse an Betrieb, Sicherheit und
Optimierung der Website).

\section{Telematikinfrastruktur}
\label{sec:ti}

\subsection{TI-Komponenten}

Die Praxis nutzt das von \PVSName{} gehostete TI-Gateway (Managed Service).
Die Verbindung erfolgt verschlüsselt über VPN-Softclient gemäß \PVSName{}-Vorgaben.

\textbf{Vorteile des gehosteten TI-Gateway:}
\begin{itemize}
  \item Keine lokale Hardware erforderlich
  \item Automatische Updates durch \PVSName{}
  \item 24/7-Monitoring durch \PVSName{}
  \item Verschlüsselte VPN-Verbindung
\end{itemize}

Zusätzlich wird ein Kartenterminal am Praxisstandort für Kartenzugriffe genutzt
(Details siehe Geräteliste \cref{sec:geraeteliste}).

\subsection{Sicherheitsmaßnahmen}

Alle Standard-Passwörter der TI-Komponenten wurden geändert. Updates werden zeitnah nach
Verfügbarkeit eingespielt. Die Administrationsdaten werden sicher aufbewahrt, wobei der
Praxis-Zugriff auf alle TI-Komponenten gewährleistet bleibt.

Die SMC-B-Karte (Praxisausweis) bleibt dauerhaft im Terminal gesteckt und wird versiegelt.
Der eHBA (Therapeutenausweis) wird nur bei Bedarf eingesteckt und anschließend sicher verwahrt.

\subsection{KIM – Kommunikation im Medizinwesen}
\label{sec:kim}

Für die sichere Kommunikation mit anderen Leistungserbringern wird KIM (Kommunikation
im Medizinwesen) genutzt. KIM ist der offizielle TI-Dienst für verschlüsselte
Nachrichten zwischen Praxen, Krankenhäusern und anderen TI-Teilnehmern.

\textbf{Nutzung:} KIM wird direkt über das PVS (\PVSName{}) verwendet – nicht als
separater E-Mail-Client. Arztbriefe, Befunde und andere medizinische Dokumente
werden ausschließlich über KIM versendet, nicht per unverschlüsselter E-Mail oder Fax.

\textbf{Sicherheitsmerkmale:}
\begin{itemize}
  \item Ende-zu-Ende-Verschlüsselung
  \item Authentifizierung über TI-Zertifikate
  \item Zustellung nur an verifizierte TI-Teilnehmer
\end{itemize}

\subsection{Außerbetriebnahme und Entsorgung}
\label{sec:ti-ausserbetriebnahme}

Bei Außerbetriebnahme von TI-Komponenten (Kartenterminal, Konnektor):
\begin{enumerate}
  \item SMC-B und gSMC-KT aus Terminal entfernen
  \item Karten über Herstellerportal (D-Trust, medisign oder T-Systems) sperren
  \item Terminal/Konnektor auf Werkseinstellungen zurücksetzen
  \item Geräte an Lieferanten zurückgeben oder zertifiziert entsorgen
  \item PIN/PUK-Dokumentation datenschutzkonform vernichten
\end{enumerate}

\textbf{Bei Verlust oder Diebstahl von SMC-B/eHBA:}
Sofortige Sperrung über Herstellerportal, ersatzweise über \KV{} (Arztregister).
Anschließend Incident-Response-Prozess gemäß \cref{sec:notfall} einleiten.

\section{Mobile Geräte}
\label{sec:mobile-geraete}

\subsection{Geltungsbereich}
Diese Richtlinie gilt für die \PraxisInhaberBezeichnung{} als Einzelpraxis ohne weitere Beschäftigte.
iPhone und iPad werden ausschließlich dienstlich genutzt, private Geräte sind strikt
getrennt (kein\ BYOD).

\subsection{Kernmaßnahmen}
Mobile Geräte sind mit biometrischer Authentifizierung (FaceID) und komplexen
Gerätesperrcodes geschützt. Technische Konfiguration und Sicherheitsrichtlinien
→ geregelt in \cref{sec:mdm}.

Details verbindlich in \cref{sec:anhang-mobile}.

\textbf{Genutzte Apps:} iOS-Standard-Apps (Mail, Kalender, Safari) und \PVSName{}-App.
Für Kurznachrichten ausschließlich \enquote{Nachrichten} (SMS).

\textbf{App-Richtlinie:} Es gilt das Allowlist-Prinzip – nur explizit freigegebene
Apps dürfen auf Praxisgeräten installiert werden. Explizit verboten sind insbesondere:
WhatsApp, Telegram, Signal (für berufliche Kommunikation), Social-Media-Apps
(Facebook, Instagram, TikTok) sowie private Cloud-Dienste (Dropbox, Google Drive).

\textbf{App-Berechtigungen:} → geregelt in \cref{sec:mdm}.

\textbf{App-Datenschutz:} → geregelt in \cref{sec:mdm}.

\textbf{Sprachassistent:} Siri ist auf allen Geräten deaktiviert, um unbeabsichtigte
Datenpreisgabe und Abhörrisiken zu vermeiden (Anlage~2 Anforderung).

\subsection{Smart Devices und Wearables}

\textbf{Smart Home Geräte:} Keine Smart Home Geräte (HomePod, Alexa, etc.) in den
Praxisräumen, um Abhörrisiken und unbeabsichtigte Datenpreisgabe zu vermeiden.

\textbf{Wearables:} Private Wearables (Apple Watch, etc.) werden während der Behandlung
verschlossen aufbewahrt oder haben deaktivierte Sprachassistenten. Aufzeichnungen nur
mit expliziter Patienteneinwilligung und über C5-zertifizierte, DSGVO-konforme Anwendungen.

\subsection{Datenverarbeitung}
Patientendaten werden nur verschlüsselt übertragen (TLS) und nicht in nicht-konformen
Cloud-Diensten gespeichert. Bei Verlust erfolgt unverzügliche Meldung und Fernlöschung
über \enquote{Mein iPhone suchen}.

\textbf{Kontaktdaten-Management:} Patientenkontakte werden ausschließlich im
Praxisverwaltungssystem (\PVSName{}) gepflegt. E-Mail-Adressen sind nur temporär in der
E-Mail-Inbox (\EmailAnbieter{}) für organisatorische Kommunikation vorhanden, werden aber
nicht als separate Kontakte gespeichert. Die Verarbeitung ist zweckgebunden und im
Verzeichnis von Verarbeitungstätigkeiten gemäß DSGVO Art.\ 30 dokumentiert.

\subsection{Dokumentation}
Die detaillierte Selbstverpflichtung zur Einhaltung der Mobile-Device-Richtlinie ist in
\cref{sec:anhang-mobile} dokumentiert und wurde von der \PraxisInhaberBezeichnung{}
unterzeichnet (siehe Mobile Geräte Selbstverpflichtung).

\textbf{Technische Umsetzung:} Die organisatorischen Richtlinien für mobile Geräte
werden durch das zentrale Mobile Device Management (MDM) technisch durchgesetzt und
überwacht, wie im folgenden Kapitel detailliert beschrieben.

\section{Mobile Device Management}
\label{sec:mdm}

\subsection{Grundsätze und Zielsetzung}

Das Mobile Device Management (MDM) der Praxis erfolgt über \textbf{\MDMTool{}} und gewährleistet
die sichere Verwaltung aller Apple-Geräte entsprechend den Anforderungen der KBV-Richtlinie
nach §390 SGB V. Ziel ist die Balance zwischen Sicherheit, Datenschutz und praktischer
Nutzbarkeit im Praxisalltag.

\textbf{Verwaltete Geräte:} Alle Apple-Geräte der Praxis werden über \MDMTool{} verwaltet. Die vollständige Geräteliste mit technischen Details befindet sich in Anhang~\ref{sec:geraeteliste}.

\subsection{\MDMTool{} Konfiguration}

Die MDM-Konfiguration erfolgt über ein einheitliches \textbf{Blueprint} mit folgenden
Sicherheitsrichtlinien:

\subsubsection{Passcode und Authentifizierung}
\begin{itemize}
  \item \textbf{Code erforderlich:} Auf allen Geräten (iPhone, iPad, Mac)
  \item \textbf{Komplexer Code:} Keine Wiederholungen oder Sequenzen (123, ABC)
  \item \textbf{Mindestlänge:} 8 Zeichen
  \item \textbf{Schonfrist:} Maximal 5 Minuten
  \item \textbf{Fehlversuche:} Maximal 10 vor automatischer Löschung (iOS)
  \item \textbf{Automatische Sperre:} 1 Minute (iOS), 5 Minuten Bildschirmschoner (Mac)
\end{itemize}

\subsubsection{Verschlüsselung und Datenschutz}
\begin{itemize}
  \item \textbf{FileVault:} Vollständige Festplattenverschlüsselung auf Mac-Geräten
  \item \textbf{Verschlüsselte Backups:} iCloud Backup mit erweitertem Datenschutz aktiviert (Ende-zu-Ende-Verschlüsselung)
  \item \textbf{iCloud-Beschränkungen:} Schlüsselbund-Sync und Dokumentensync deaktiviert

\textbf{Hinweis:} Nutzung des lokalen macOS-Schlüsselbunds (kein iCloud-Keychain-Sync).
  \item \textbf{Schreibtisch/Dokumente:} iCloud-Synchronisation deaktiviert (Mac)
\end{itemize}

\subsubsection{Netzwerk und Konnektivität}
\begin{itemize}
  \item \textbf{Praxis-WLAN:} Automatische Verbindung mit WPA3-Verschlüsselung
  \item \textbf{MAC-Randomisierung:} \MACRandomisierung{}
  \item \textbf{AirDrop/USB:} Externe Datenübertragung beschränkt
  \item \textbf{Passwortfreigabe:} Verhindert über AirDrop
\end{itemize}

\subsubsection{Privacy und Tracking}
\begin{itemize}
  \item \textbf{Website-übergreifendes Tracking:} In Safari verhindert
  \item \textbf{Ad-Tracking:} Mit Apple-Werbeplattform eingeschränkt
  \item \textbf{Diagnoseberichte:} Automatisches Senden an Apple deaktiviert
  \item \textbf{Siri:} Deaktiviert auf iOS-Geräten
  \item \textbf{Spotlight Internet-Suche:} Deaktiviert
\end{itemize}

\subsubsection{Sperrbildschirm und Zugriffskontrolle}
\begin{itemize}
  \item \textbf{Benachrichtigungen:} Auf Sperrbildschirm verborgen
  \item \textbf{Kontrollzentrum:} Auf Sperrbildschirm verborgen
  \item \textbf{Apple Watch:} Entsperren und Kopplung deaktiviert
  \item \textbf{Bildschirmzeit:} Aktivierung deaktiviert
\end{itemize}

\subsection{Gerätespezifische Zugriffe und Anwendungen}

\subsubsection{MacBook (\PraxisInhaberBezeichnung{})}
\begin{itemize}
  \item \textbf{\PVSName{} PVS:} Lokale Installation mit Vollzugriff auf Patientendaten
  \item \textbf{Kalender:} Organisatorische Termine über \EmailAnbieter{} oder lokal
  \item \textbf{Backup:} Verschlüsselte iCloud-Backups (MDM-verwaltet)
  \item \textbf{Verschlüsselung:} FileVault mit hinterlegtem Wiederherstellungsschlüssel
  \item \textbf{Updates:} Siehe Abschnitt~\ref{sec:it-sicherheit} - sicherheitskritische Updates in Abstimmung mit PVS-Kompatibilität
\end{itemize}

\subsubsection{iPad (Behandlungsraum)}
\begin{itemize}
  \item \textbf{\PVSName{} PVS App:} Verschlüsselter Zugriff auf Praxissystem
  \item \textbf{Lokale Daten:} \PVSName{} App speichert keine Patientendaten persistent lokal
  \item \textbf{E-Mail/Kalender:} Ausschließlich organisatorische Kommunikation
\end{itemize}

\subsubsection{iPhone (\PraxisInhaberBezeichnung{})}
\begin{itemize}
  \item \textbf{Kein Praxiszugriff:} Ausschließlich organisatorische Nutzung
  \item \textbf{E-Mail/Kalender:} Termine und allgemeine Korrespondenz
\end{itemize}

\subsection{App-Management und Sicherheitsrichtlinien}

\subsubsection{Verbotene Anwendungen}
\begin{itemize}
  \item \textbf{WhatsApp:} Installation und Nutzung untersagt
  \item \textbf{Alternative Marketplace Apps:} Installation deaktiviert
  \item \textbf{Compliance-Überwachung:} Automatische Erkennung verbotener Apps über \MDMTool{}
\end{itemize}

\subsubsection{Erlaubte Anwendungen}
\begin{itemize}
  \item \textbf{\PVSName{} PVS:} Auf MacBook und iPad
  \item \textbf{\EmailAnbieter{}:} E-Mail und Kalender für organisatorische Zwecke
  \item \textbf{Safari:} Mit aktivierten Tracking-Schutzmaßnahmen
  \item \textbf{Standard-iOS/macOS Apps:} Nach Sicherheitskonfiguration
\end{itemize}

\subsection{Incident Response und Notfallmaßnahmen}

\subsubsection{Geräteverlust oder -diebstahl}
\begin{enumerate}
  \item \textbf{Sofortige Meldung:} An \PraxisInhaberBezeichnung{} binnen 2 Stunden
  \item \textbf{Remote Wipe:} Fernlöschung über \MDMTool{} innerhalb von 4 Stunden
  \item \textbf{Geolocation:} Aktivierung zur Geräteortung (falls möglich)
  \item \textbf{Passwort-Änderung:} Alle betroffenen Accounts (E-Mail, \PVSName{})
  \item \Dokumentation Incident-Protokoll gemäß Abschnitt~\ref{sec:notfall}
\end{enumerate}

\subsubsection{Compliance-Verstöße}
\begin{enumerate}
  \item \textbf{Automatische Erkennung:} Über \MDMTool{} Device Compliance Status
  \item \textbf{Benachrichtigung:} Sofortige Meldung an \PraxisInhaberBezeichnung{}
  \item \textbf{Remediation:} Wiederherstellung der Compliance binnen 72 Stunden
  \item \textbf{Eskalation:} Bei wiederholten Verstößen temporäre Gerätesperre
\end{enumerate}

\subsection{Verschlüsselung und Datenübertragung}

\textbf{Keine VPN-Infrastruktur erforderlich}, da:
\begin{itemize}
  \item \textbf{\PVSName{} App:} Verschlüsselte Verbindung (TLS/HTTPS) zum Praxissystem
  \item \textbf{MacBook:} Lokale PVS-Installation, keine Fernzugriffe
  \item \textbf{Kalender:} \EmailAnbieter{} oder lokal (keine Cloud-Synchronisation von Patientendaten)
  \item \textbf{Backup:} Verschlüsselte iCloud-Backups (MDM-verwaltet, AVV vorhanden)
  \item \textbf{Organisatorische Trennung:} Keine externen Zugriffe auf Praxisinfrastruktur
\end{itemize}

\subsection{Monitoring und Wartung}

\subsubsection{Regelmäßige Überprüfungen}
\begin{itemize}
  \item \textbf{Monatlich:} Device Compliance Status in \MDMTool{}
  \item \textbf{Quartalsweise:} Überprüfung der Blueprint-Konfiguration
  \item \textbf{Halbjährlich:} Bewertung der MDM-Richtlinien und Anpassungen
  \item \textbf{Jährlich:} Vollständige Sicherheitsevaluation aller verwalteten Geräte
\end{itemize}

\subsubsection{Dokumentation und Compliance}
\begin{itemize}
  \item \textbf{Geräteliste:} Alle Apple-Geräte der Praxis - siehe vollständige Geräteliste in Anhang~\ref{sec:geraeteliste}
  \item \textbf{AVV \MDMTool{}:} Siehe Anhang~\ref{sec:avv-register}
  \item \textbf{VVT MDM:} Siehe Anhang~\ref{sec:vvt}
  \item \textbf{Incident-Protokolle:} Archivierung für 3 Jahre
\end{itemize}

\textbf{Verantwortlichkeit:} Die \PraxisInhaberBezeichnung{} ist für die Einhaltung der MDM-Richtlinien
und die ordnungsgemäße Konfiguration aller verwalteten Geräte verantwortlich.
IT-Ansprechpartner administriert \MDMTool{} (siehe Rechte-Matrix \cref{sec:rechte-rollenmatrix}).
\MDMTool{} ist für bis zu 3 Geräte kostenfrei.

\section{Wechseldatenträger und Speichermedien}
\label{sec:wechseldatentraeger}

\subsection{Kernmaßnahmen}
Wechseldatenträger werden ausschließlich als verschlüsseltes Volume verwendet. Fremde
Datenträger werden grundsätzlich nicht genutzt. Mitnahme nur durch \PraxisInhaberBezeichnung{} für verschlüsselte
Backup-Medien.

\subsection{Versand von Wechseldatenträgern}
\textbf{Nicht anwendbar:} In dieser Praxis erfolgt kein Versand von Wechseldatenträgern
oder Speichermedien. Datenaustausch erfolgt ausschließlich über verschlüsselte
Online-Übertragung (TLS/S-MIME). Sollte zukünftig ein Versand erforderlich werden,
sind entsprechende Kennzeichnungs- und Versandrichtlinien zu entwickeln.

Details verbindlich in \cref{sec:anhang-wechseldatentraeger}.

\section{Verschwiegenheit und externe Dienstleister}
\label{sec:verschwiegenheit}

\subsection{Grundsätze}
Alle externen Dienstleister, die Zugang zu Praxisdaten haben könnten, werden zur
Verschwiegenheit verpflichtet. Dies umfasst IT-Support, Wartungsdienste und andere
Auftragsverarbeiter.

\subsection{Auftragsverarbeitung}
Mit allen Dienstleistern werden Auftragsverarbeitungsverträge (AVV) nach Art.~28 DSGVO
geschlossen. Diese regeln die Verarbeitung nach Weisung, technische und organisatorische
Maßnahmen sowie Löschungsfristen (siehe \cref{sec:avv-register}).

\subsection{Zugangskontrollen}
Externe Zugriffe erfolgen nur nach vorheriger Terminabsprache und unter Aufsicht.
Remote-Zugriffe sind organisatorisch untersagt. IT-Support erfolgt ausschließlich
vor Ort oder unter direkter Bildschirmaufsicht (siehe SOP \cref{sec:vorlage-remote-support}).

\subsection{Dokumentation}
Alle Verschwiegenheitserklärungen externer Dienstleister sind in
\cref{sec:anhang-verschwiegenheit} dokumentiert und unterzeichnet (siehe
Verschwiegenheitserklärung externe Dienstleister).

\section{Notfall- und Incident-Management}
\label{sec:notfall}

\subsection{Notfallkontakte}

% Tagging: Keine Header-Zeile (Key-Value Tabelle)
\begin{tabular}{p{0.35\textwidth} p{0.55\textwidth}}
  \textbf{\PraxisInhaberBezeichnung{}:} & \PraxisInhaberin \\
  & Praxis: \PraxisTelefon, Privat: \PraxisInhaberinPrivatTelefon \\
  & E-Mail: info@\PraxisDomain \\[0.5em]
  \textbf{IT-Ansprechpartner:} & \ITAnsprechpartner \\
  & Mobil: \ITAnsprechpartnerTelefon \\
  & Adresse: \ITAnsprechpartnerAdresse \\[0.5em]
  \textbf{Externe Dienstleister:} & \PVSName{}, \InternetAnbieter{} (Internet) \\
  \textbf{Behörden:} & \KV{} (IT-Sicherheit), \Kammer{} (Kammer) \\
  & \Datenschutzbehoerde{} (Datenschutz-Aufsichtsbehörde): \DatenschutzbehoerdeAdresse \\
\end{tabular}

\subsection{Verfügbarkeitsziele}
\label{sec:verfuegbarkeit}

\textbf{Kritische Systeme (Praxisbetrieb gefährdet):}
\begin{itemize}
  \item \textbf{\PVSName{} PVS:} Max. 4h Ausfall, Ziel 99,5\% Verfügbarkeit
  \item \textbf{Internet/E-Mail:} Max. 2h Ausfall, Ziel 99,8\% Verfügbarkeit
  \item \textbf{Telefon:} Max. 1h Ausfall, Ziel 99,9\% Verfügbarkeit
\end{itemize}

\textbf{Wichtige Systeme (Betrieb eingeschränkt):}
\begin{itemize}
  \item \textbf{Drucker:} Max. 72h Ausfall, Ziel 95\% Verfügbarkeit
  \item \textbf{Mobile Geräte:} Max. 8h Ausfall, Ziel 99\% Verfügbarkeit
\end{itemize}

\textbf{Unkritische Systeme:}
\begin{itemize}
  \item \textbf{Website:} Max. 72h Ausfall, Ziel 95\% Verfügbarkeit
\end{itemize}

\subsection{Datenschutzvorfall-Management}

Bei Datenschutzverletzungen erfolgen binnen 24 Stunden die Dokumentation des Vorfalls
(Was, Wann, Wie, betroffene Daten), Schadensbegrenzung, Information des
IT-Ansprechpartners und eine Risikobewertung (siehe \cref{sec:vorlage-breach-register}).

Die Meldepflicht an die Aufsichtsbehörde LfD erfolgt binnen 72 Stunden mit Angabe der
Art der Verletzung, Kategorien betroffener Daten, ungefähre Anzahl Betroffener, Folgen
und ergriffene Maßnahmen. Bei hohem Risiko für Rechte und Freiheiten werden Betroffene
unverzüglich in verständlicher Sprache informiert.

\subsection{IT-Security-Incident-Response}

IT-Sicherheitsvorfälle werden nach Kategorien behandelt: Malware/Virus-Befall, unbefugter
Zugriff, Datenverlust/Hardware-Ausfall und Netzwerk-/Internet-Ausfall.

\textbf{Incident-Response-Timeline:}

\textbf{Sofortmaßnahmen (0–15 Min):}
\begin{itemize}
  \item Isolation: Betroffenes Gerät sofort vom Netz trennen (WLAN aus, Kabel ziehen)
  \item Schadensbegrenzung: Weitere Systeme prüfen und ggf. isolieren
  \item Erste Dokumentation: Was, Wann, Wo in Stichpunkten notieren
\end{itemize}

\textbf{Meldung und Analyse (15 Min - 2h):}
\begin{itemize}
  \item IT-Ansprechpartner informieren: Binnen 30 Min nach Abschluss der laufenden Patientenbehandlung (außer nachts)
  \item Incident-Log erstellen: Binnen 1h strukturiert dokumentieren
  \item Erste Ursachenanalyse: Binnen 2h Verdacht eingrenzen
\end{itemize}

\textbf{Wiederherstellung (2h - 24h):}
\begin{itemize}
  \item Bereinigung:
    \begin{itemize}
      \item Malware-Befall: 2–4h (Neuinstallation aus Backup)
      \item Ransomware: 4–8h (vollständige Systemwiederherstellung)
      \item Hardware-Ausfall: 24h (Ersatzbeschaffung)
    \end{itemize}
  \item System-Wiederherstellung: \PVSName{} binnen 4h, andere Systeme binnen 24h
  \item Vollständige Dokumentation: Binnen 24h abgeschlossen
\end{itemize}

\textbf{Nachbereitung (24h - 72h):}
\begin{itemize}
  \item Ursachenanalyse final: Binnen 48h
  \item Lessons Learned: Binnen 72h dokumentiert
  \item Maßnahmen-Update: Binnen 1 Woche umgesetzt
\end{itemize}

\textbf{Eskalation:} Nach 2h ohne Lösung → Externe Hilfe (\PVSName{}). Nach 4h → Praxisbetrieb gefährdet → Notfallmodus.

\subsection{Frühwarnsystem (Honeypots)}
\label{sec:honeypots}

Zur frühzeitigen Erkennung unbefugter Zugriffe sind digitale Köder (Canary Tokens)
im Praxisnetzwerk platziert. Diese inaktiven Dateien simulieren vertrauliche Inhalte
und lösen bei Zugriff automatisch Alarm aus.

\textbf{Implementierung:}
\begin{itemize}
  \item Mehrere Köderdateien verschiedener Typen an strategischen Speicherorten
  \item Dateinamen suggerieren sensiblen Inhalt (z.\,B. Patientendaten, Abrechnungen)
  \item Bei Zugriff: Automatische E-Mail-Benachrichtigung an Praxisinhaberin
  \item Alarminhalt: Zeitstempel, IP-Adresse, Gerätekennung
\end{itemize}

\textbf{Reaktion bei Auslösung:}
\begin{enumerate}
  \item Sofortige Prüfung: Legitimer Zugriff (Fehlalarm) oder Sicherheitsvorfall?
  \item Bei Verdacht: Incident-Response-Prozess gemäß vorherigem Abschnitt einleiten
  \item Dokumentation im Vorfallprotokoll
\end{enumerate}

\textit{Hinweis: Genaue Dateitypen, Namen und Speicherorte werden aus Sicherheitsgründen
nicht dokumentiert und sind nur der Praxisinhaberin bekannt.}

\subsection{Backup-Wiederherstellung}

Die Wiederherstellung erfolgt primär über lokales Time Machine-Backup, sekundär über das
NAS am \BackupStandort{}. Verantwortlich ist \ITAnsprechpartner{} mit Zugriff über
\ITAnsprechpartnerAdresse.

Der Wiederanlauf erfolgt in folgender Reihenfolge: Netzwerk und Internet, Betriebssystem
und Grundkonfiguration, Praxisverwaltungssoftware (\PVSName{}), E-Mail und Kommunikation,
Patientendaten und Dokumentation.

\subsection{IT-Risikobewertung}
\label{sec:risikobewertung}

\begin{center}
\small
\begin{tabular}{p{2.5cm}p{2cm}p{1.8cm}p{1.5cm}p{3cm}}
\toprule
\textbf{Bedrohung} & \textbf{Wahrschein-lichkeit} & \textbf{Schaden} & \textbf{Risiko-Score} & \textbf{Maßnahmen} \\
\midrule
Ransomware & Mittel & Kritisch & HOCH & Backup, Schulung, macOS Security \\
Gerätediebstahl & Niedrig & Hoch & MITTEL & FileVault, Remote Wipe \\
Internetausfall & Hoch & Mittel & MITTEL & Mobilfunk-Backup \\
Datenverlust & Niedrig & Kritisch & MITTEL & 3–2-1 Backup-Regel \\
Phishing & Mittel & Mittel & MITTEL & S/MIME, Schulung \\
\bottomrule
\end{tabular}
\end{center}
\normalsize

\textbf{Risiko-Score:} Niedrig×Niedrig=NIEDRIG, Niedrig×Mittel=NIEDRIG,
Mittel×Mittel=MITTEL, Hoch×Hoch=KRITISCH

\subsection{KI-Dokumentationsausfall}

Bei Ausfall der KI-gestützten Dokumentation erfolgt die Sitzungsdokumentation
direkt digital im PVS. Patientenbehandlung ist nicht beeinträchtigt, da die
manuelle Dokumentation den regulären Behandlungsstandard darstellt.

\subsection{Eskalation}

Die Eskalation erfolgt stufenweise: IT-Ansprechpartner (\ITAnsprechpartner), externe
Dienstleister (\PVSName{}, Provider), Behörden (bei Datenschutzvorfällen), Kammer und sofern
vorhanden Versicherung (bei schweren Vorfällen).

Die Erreichbarkeit ist während der Praxiszeiten nach Absprache mobil gewährleistet,
außerhalb der Zeiten über Privat \PraxisInhaberinPrivatTelefon. IT-Support erfolgt über
\ITAnsprechpartner{} \ITAnsprechpartnerTelefon.

\subsection{IT-Sicherheits-KPIs}
\label{sec:kpis}

\textbf{Monatliche Metriken:}
\begin{itemize}
  \item \textbf{Backup-Erfolgsrate:} Ziel >99\% (Time Machine + NAS)
  \item \textbf{Update-Compliance:} Ziel 100\% binnen 14 Tagen
  \item \textbf{Phishing-Tests:} Quartalsweise Selbsttest
  \item \textbf{Passwort-Hygiene:} \PasswortManager{} Security Score >80
\end{itemize}

\textbf{Jährliche Reviews:}
\begin{itemize}
  \item Penetration-Test (Selbsttest mit Online-Tools, angestrebt)
  \item Compliance-Audit (Eigenprüfung gegen KBV-Anforderungen)
  \item Risikobewertung-Update
  \item Notfallplan-Test (Tabletop-Übung, mindestens jährlich, angestrebt halbjährlich)
\end{itemize}

\textbf{Eskalation bei Unterschreitung:}
Backup <95\% → Sofortige Prüfung, Updates >30 Tage → Risikobewertung

\subsection{Jährliche Tabletop-Übung}

Zur Überprüfung der Notfallprozesse und Schulung der Beteiligten wird
mindestens jährlich, angestrebt halbjährlich, eine 30-minütige Tabletop-Übung durchgeführt. Typische Szenarien:
Ransomware-Angriff, Geräteverlust/Diebstahl, Netzwerkausfall oder Datenschutzvorfall.

\textbf{Ablauf:} Szenario-Präsentation, Rollenverteilung, Durchspielen der Maßnahmen
nach Notfallplan, Identifikation von Lücken und Verbesserungsmaßnahmen, Dokumentation
der Erkenntnisse und Ableitung von Anpassungen.

\textbf{Teilnehmende:} \PraxisInhaberBezeichnung{}, IT-Ansprechpartner, ggf. weitere Mitarbeitende.

\Dokumentation Strukturierte Protokollierung nach Arbeitsvorlage in
\cref{sec:vorlage-tabletop} mit Szenario, Ablauf, Erkenntnissen und Maßnahmen.

\section{Technische und Organisatorische Maßnahmen}
\label{sec:toms}

\textbf{Hinweis zur Darstellung der TOMs nach Art.\ 32 DSGVO:}\\
Gemäß Art.\ 32 DSGVO sind alle Maßnahmen vollständig zu dokumentieren.
Die nachfolgende Übersicht stellt daher die TOMs in rechtlicher Vollständigkeit dar.
Zur Vermeidung von Redundanz verweisen einzelne Maßnahmen auf die entsprechenden Fachkapitel
(z.\,B.\ \ref{sec:netzwerk} Netzwerksicherheit,~\ref{sec:kommunikation} E-Mail-Kommunikation),
in denen die technische Umsetzung detailliert beschrieben ist.
Zusätzlich dient der Netzplan im Anhang zur Visualisierung der Konfiguration.

\subsection{Übersicht}
Die technischen und organisatorischen Maßnahmen nach Art.\ 32 DSGVO gewährleisten\ die
Datensicherheit durch ein mehrstufiges Schutzkonzept.

\subsection{Zugangs- und Zutrittskontrolle}

Die Praxis ist abschließbar mit Schlüsseln nur bei der \PraxisInhaberBezeichnung{}
und dem Reinigungspersonal (mit Verschwiegenheitserklärung). Benutzerkonten
existieren nur für autorisierte Personen mit starken Passwörtern (\PasswortManager{}) und
automatischer Bildschirmsperre. Zwei-Faktor-Authentifizierung wird wo möglich eingesetzt.

\subsection{Zugriffs- und Weitergabekontrolle}

Rollenbasierte Zugriffe im PVS (\PVSName{}) beschränken Berechtigungen auf erforderliche
Daten. Gemeinsam genutzte Accounts existieren nicht. Die systematische Zugriffskontrolle
ist in der Rechte- und Rollenmatrix (\cref{sec:rechte-rollenmatrix}) vollständig
dokumentiert. Bei Datenübertragung werden verschlüsselte Datenträger und sichere
Übertragungswege (HTTPS) verwendet.

\subsection{Eingabe- und Auftragskontrolle}

Dateneingaben erfordern Benutzer-Authentifizierung und werden in \PVSName{} protokolliert.
Änderungshistorien werden bei kritischen Daten geführt. Auftragsverarbeitungsverträge
(AVV) mit Dienstleistern regeln die Verarbeitung nach Weisung, ergänzt durch
Vertraulichkeitsvereinbarungen.

\subsection{Verfügbarkeits- und Trennungskontrolle}

Backup-Konzept und redundante Systeme sichern die Verfügbarkeit (Details siehe
\cref{sec:datensicherung}). Separate Benutzerkonten für verschiedene Aufgaben und
getrennte Netzwerke (Praxis vs. privat)\ gewährleisten\ die Zwecktrennung.

\subsection{Verschlüsselung und Pseudonymisierung}

Festplattenverschlüsselung (FileVault), verschlüsselte Backups und Wechseldatenträger
schützen personenbezogene Daten. E-Mail-Verschlüsselung siehe
\cref{sec:kommunikation}. Statistiken und Berichte werden pseudonymisiert erstellt.

\subsection{Netzwerk- und Systemsicherheit}

Netzwerksicherheit durch Firewall-Konfiguration (Details siehe
\cref{sec:netzwerk}). Systemlogs werden anlassbezogen ausgewertet, zusätzlich
stichprobenartige Sichtung im jährlichen Eigencheck. Automatische Software-Updates und
integrierte macOS-Sicherheitsfeatures (XProtect, Gatekeeper) gewährleisten aktuellen
Schutzstand (Details siehe \cref{sec:endgeraete}).

\subsection{Physische Sicherheit}

\textbf{Zutrittskonzept:} Die Praxis ist abschließbar, Schlüssel werden nur an
autorisiertes Personal ausgegeben. Bei zukünftigen
Einstellungen: Schlüsselvergabe dokumentiert, Rückgabe beim Ausscheiden.
Büro/IT-Räume sind zusätzlich gesichert und nur für Praxisinhaberin und IT-Support
unter Aufsicht zugänglich.

\textbf{Clean-Desk-Policy:} Digitale Arbeitsweise ohne Papierakten. Bildschirme werden
bei Verlassen des Arbeitsplatzes gesperrt (automatisch nach 5 Min). Wechseldatenträger
werden verschlüsselt und in abschließbaren Schränken aufbewahrt. Ausdrucke mit
Patientendaten werden unmittelbar nach Verwendung geschreddert.

\textbf{Besucherregelung:} Patienten haben Zugang zum Wartebereich und Behandlungsräumen,
nicht zu Büro/IT-Räumen. Externe Dienstleister (IT-Support, Wartung) arbeiten nur unter
Aufsicht oder außerhalb der Praxiszeiten. Alle externen Zugriffe werden dokumentiert
(siehe \cref{sec:vorlage-remote-support}).

\textbf{Geräte- und Bildschirmsicherheit:} IT-Geräte sind in abschließbaren Räumen
untergebracht. Mobile Geräte werden bei Nichtgebrauch sicher verwahrt. Bildschirme sind
vor Einsicht Unbefugter geschützt positioniert und mit Sichtschutzfolien ausgestattet.
Kameras sind mit Abdeckungen versehen.

\subsection{Regelmäßige Überprüfung}

Jährliche Überprüfung der TOMs und halbjährliche IT-Sicherheitsprüfungen (angestrebt) 
gewährleisten kontinuierliche Verbesserung. Für Einzelpraxen ist eine jährliche 
umfassende Prüfung ausreichend, ergänzt durch anlassbezogene Kontrollen bei 
Änderungen oder Sicherheitsvorfällen. Die Rechte- und Rollenmatrix wird jährlich auf Aktualität
und Angemessenheit geprüft. Anpassungen erfolgen bei Änderungen der Technik oder
Rechtslage mit vollständiger Dokumentation.

Die strukturierte Überprüfung aller TOMS-Bereiche erfolgt nach den Arbeitsvorlagen
in \cref{sec:vorlage-toms-pruefung} mit dokumentierten Prüfzyklen und Nachweisführung.

IT-Sicherheitsschulungen werden jährlich durch die \PraxisInhaberBezeichnung{} durchgeführt, 
quartalsweise angestrebt, und nach dem Schulungsplan in \cref{sec:vorlage-schulungsplan} dokumentiert. 
Zusätzliche Schulungen erfolgen anlassbezogen bei aktuellen Bedrohungen oder Sicherheitsvorfällen.

\textbf{Hinweis für Einzelpraxen:} Die gesetzliche Mindestanforderung (§390 SGB V) beträgt 
4 Stunden jährlich. Quartalsweise Kurzmodule sind wünschenswert, aber nicht verpflichtend.
Bei zukünftigen Einstellungen sind alle neuen Mitarbeitenden vor Systemzugang zu schulen.

Alle Prüfungen werden systematisch durchgeführt und 10 Jahre aufbewahrt.

\subsection{Betroffenenrechte nach Art. 12--23 DSGVO}

Die Praxis gewährleistet die ordnungsgemäße Bearbeitung aller Betroffenenanfragen
nach DSGVO Art. 15 (Auskunft), Art. 16 (Berichtigung), Art. 17 (Löschung),
Art. 18 (Einschränkung), Art. 20 (Datenübertragbarkeit) und Art. 21 (Widerspruch).

\textbf{Verfahren:} Eingehende Anträge werden registriert, die Identität des
Antragstellers geprüft (Geburtsdatum, Adresse) und binnen einem Monat bearbeitet.
Bei komplexen Anfragen erfolgt eine begründete Fristverlängerung um maximal zwei Monate.
Alle betroffenen Datenbestände (PVS, E-Mail, Papierakten, Backups) werden systematisch
geprüft und die Bearbeitung vollständig dokumentiert.

\textbf{Verantwortlichkeit:} \PraxisInhaberBezeichnung{} als Verantwortliche nach Art. 4 Nr. 7 DSGVO.

\Dokumentation Strukturierte Bearbeitung nach Arbeitsvorlage in
\cref{sec:vorlage-betroffenenrechte} mit Nachweis über Eingang, Identitätsprüfung,
Fristberechnung, Datenbestandsprüfung und Antwortversendung.

\section{Rechte- und Rollenmatrix}
\label{sec:rechte-rollenmatrix}

\subsection{Übersicht}

Die Rechte- und Rollenmatrix definiert systematisch die Zugriffsberechtigung auf alle
IT-Systeme der Praxis. Sie gewährleistet das Prinzip der minimalen Berechtigung und
unterstützt die Nachvollziehbarkeit von Systemzugriffen gemäß §~390 SGB~V.

\subsection{Rollendefinitionen}

\begin{tabular}{>{\raggedright\arraybackslash}p{3cm}>{\raggedright\arraybackslash}p{4cm}>{\raggedright\arraybackslash}p{7cm}}
\toprule
\textbf{Rolle} & \textbf{Beschreibung} & \textbf{Verantwortlichkeiten} \\
\midrule
Praxisinhaberin & \PraxisInhaberBezeichnung{} & Gesamtverantwortung IT-Sicherheit, Patientendaten, Compliance \\
IT-Support & Externer IT-Ansprechpartner & Technische Betreuung, Wartung, Updates (unter Aufsicht) \\
\PVSName{}-Support & PVS-Hersteller (extern) & \PVSName{}-System-Support, \RemoteTool{}-Fernwartung (unter Aufsicht) \\
Therapeut/in & Angestellte Therapeuten & Patientenbehandlung, Dokumentation (zukünftig) \\
Verwaltung & Verwaltungsangestellte & Termine, Abrechnung, Organisation (zukünftig) \\
Vertretung & Praxisvertretung & Notfall-Patientenversorgung (zukünftig) \\
\bottomrule
\end{tabular}

\subsection{Aktuelle Rollenzuordnung}

\begin{tabular}{ll}
\toprule
\textbf{Person} & \textbf{Rolle} \\
\midrule
\PraxisInhaberin & Praxisinhaberin \\
\ITAnsprechpartner{} (extern) & IT-Support \\
\PVSName{} GmbH (extern) & \PVSName{}-Support \\
\textit{(noch nicht besetzt)} & \textit{Therapeut/in (inkl. Vertretung)} \\
\textit{(noch nicht besetzt)} & \textit{Verwaltung (zukünftig)} \\
\bottomrule
\end{tabular}

\subsection{Systemberechtigungen nach Rollen}

\subsubsection{Praxisverwaltungssystem (\PVSName{})}

\begin{tabular}{p{4cm}p{2.5cm}p{2.5cm}p{2.5cm}p{2.5cm}}
\toprule
\textbf{Funktion} & \textbf{Praxisinhaberin} & \textbf{IT-Support} & \textbf{Therapeut/in*} & \textbf{Verwaltung*} \\
\midrule
Admin-Account & \cmark & \xmark & \xmark & \xmark \\
Standard-Benutzer & \cmark & \xmark & \cmark & \cmark \\
Patientendaten & Vollzugriff & \xmark & Eigene Patienten & Lesezugriff \\
Termine & Vollzugriff & \xmark & Eigene Termine & Vollzugriff \\
Abrechnung & Vollzugriff & \xmark & \xmark & Vollzugriff \\
Systemeinstellungen & \cmark & \xmark & \xmark & \xmark \\
\bottomrule
\end{tabular}

\textit{* = Zukünftige Rollen, noch nicht besetzt}

\textbf{Hinweis:} \PVSName{}-Support hat Admin-Zugriff auf alle PVS-Funktionen, jedoch nur unter Aufsicht der Praxisinhaberin via \RemoteTool{}-Session. \RemoteTool{} wird nur bei Bedarf gestartet und ist standardmäßig nicht aktiv. Alle Remote-Support-Sitzungen werden nach der SOP in \cref{sec:vorlage-remote-support} durchgeführt und dokumentiert.

\subsubsection{Sicherheit und Passwörter}

\begin{tabular}{p{4cm}p{2.5cm}p{2.5cm}p{2.5cm}p{2.5cm}}
\toprule
\textbf{System} & \textbf{Praxisinhaberin} & \textbf{IT-Support} & \textbf{Therapeut/in} & \textbf{Verwaltung} \\
\midrule
\PasswortManager{} Praxis-Vault & Vollzugriff & Emergency & \xmark & \xmark \\
\PasswortManager{} Shared Vault & \cmark & \cmark & \xmark & \xmark \\
TLS-Zertifikate & \cmark & \cmark & \xmark & \xmark \\
\bottomrule
\end{tabular}

\subsubsection*{Netzwerk-Infrastruktur}

\begin{tabular}{p{4cm}p{2.5cm}p{2.5cm}p{2.5cm}p{2.5cm}}
\toprule
\textbf{Komponente} & \textbf{Praxisinhaberin} & \textbf{IT-Support} & \textbf{Therapeut/in} & \textbf{Verwaltung} \\
\midrule
\RouterKurz{} Admin & \cmark & \cmark & \xmark & \xmark \\
WLAN-Verwaltung & \cmark & \cmark & \xmark & \xmark \\
Gastnetz-Aktivierung & \cmark & \cmark & \xmark & \xmark \\
Netzwerk-Monitoring & \cmark & \cmark & \xmark & \xmark \\
\bottomrule
\end{tabular}

\tabellenheader{Hardware und Geräte}
\begin{tabular}{p{4cm}p{2.5cm}p{2.5cm}p{2.5cm}p{2.5cm}}
\toprule
\textbf{Gerät} & \textbf{Praxisinhaberin} & \textbf{IT-Support} & \textbf{Therapeut/in} & \textbf{Verwaltung} \\
\midrule
MacBook (Benutzer) & Persönlicher User & Aufsicht & Persönlicher User & Persönlicher User \\
MacBook (Admin) & \texttt{\AdminUser{}}-Account & Aufsicht & \xmark & \xmark \\
Drucker (Nutzung) & \cmark & \cmark & \cmark & \cmark \\
Drucker (Admin) & \cmark & \cmark & \xmark & \xmark \\
TI-Terminal & \cmark & Updates & \xmark & \xmark \\
Mobile Geräte & \xmark & \xmark & \xmark & \xmark \\
\bottomrule
\end{tabular}

\tabellenheader{TI-Infrastruktur}
\begin{tabular}{p{4cm}p{2.5cm}p{2.5cm}p{2.5cm}p{2.5cm}}
\toprule
\textbf{Komponente} & \textbf{Praxisinhaberin} & \textbf{IT-Support} & \textbf{Therapeut/in} & \textbf{Verwaltung} \\
\midrule
SMC-B Karte & Geteilte Praxiskarte & \xmark & Geteilte Praxiskarte & \xmark \\
KIM/ePA/eAU & \cmark & \xmark & \cmark & \xmark \\
TI-Gateway & Monitoring & Updates & \xmark & \xmark \\
\bottomrule
\end{tabular}

\tabellenheader{Backup und Wiederherstellung}
\begin{tabular}{p{4cm}p{2.5cm}p{2.5cm}p{2.5cm}p{2.5cm}}
\toprule
\textbf{System} & \textbf{Praxisinhaberin} & \textbf{IT-Support} & \textbf{Therapeut/in} & \textbf{Verwaltung} \\
\midrule
NAS-Backup & Vollzugriff & \xmark & \xmark & \xmark \\
USB-Backup & Vollzugriff & \xmark & \xmark & \xmark \\
Restore-Durchführung & \cmark & Unterstützung & \xmark & \xmark \\
\bottomrule
\end{tabular}

\tabellenheader{Web-Services und Cloud}
\begin{tabular}{p{4cm}p{2.5cm}p{2.5cm}p{2.5cm}p{2.5cm}}
\toprule
\textbf{Service} & \textbf{Praxisinhaberin} & \textbf{IT-Support} & \textbf{Therapeut/in} & \textbf{Verwaltung} \\
\midrule
Website (\HostingAnbieter{}/\IaCTool{}) & \xmark & Vollzugriff & \xmark & \xmark \\
\HostingAnbieter{}/\CodeRepo{} & \xmark & Vollzugriff & \xmark & \xmark \\
E-Mail-Postfach & Persönlich + info@ & \xmark & \xmark & \xmark \\
iCloud-Zugriff & \cmark & \xmark & \xmark & \xmark \\
\MDMTool{} MDM & Vollzugriff & Vollzugriff & \xmark & \xmark \\
\KIToolName{} KI-Dokumentation & \cmark & \xmark & \cmark & \xmark \\
\bottomrule
\end{tabular}

\subsection{Berechtigungsprinzipien}

\textbf{Minimale Berechtigung:} Jede Rolle erhält nur die für ihre Aufgaben erforderlichen
Mindestrechte.

\textbf{Funktionstrennung:} Administrative und operative Tätigkeiten sind klar getrennt.

\textbf{Datenschutz:} Therapeuten haben nur Zugriff auf ihre eigenen Patientendaten.

\textbf{Vier-Augen-Prinzip:} Kritische Systemänderungen erfolgen in Abstimmung zwischen
Praxisinhaberin und IT-Support.

\textbf{Aufsichtspflicht:} Externe IT-Support-Tätigkeiten erfolgen grundsätzlich unter
Aufsicht der Praxisinhaberin.

\subsection{Änderungsmanagement}

\textbf{Neue Mitarbeitende:} Erhalten standardmäßig die Mindestrechte ihrer Rolle.
Erweiterte Berechtigungen werden individuell geprüft und genehmigt.

\textbf{Rollenwechsel:} Bei Funktionsänderungen werden Berechtigungen entsprechend
angepasst und alte Zugänge deaktiviert.

\textbf{Ausscheiden:} Alle Zugänge werden unverzüglich deaktiviert (spätestens am nächsten Werktag), 
geteilte Passwörter geändert,
Hardware zurückgegeben.

\textbf{Regelmäßige Überprüfung:} Die Rechte-Matrix wird jährlich im Rahmen der
TOMs-Überprüfung auf Aktualität und Angemessenheit geprüft.

\Dokumentation Alle Änderungen werden im Changelog der IT-Sicherheitsdokumentation
erfasst.

\clearpage
\begin{landscape}
  \enlargethispage{2cm}

  % Alles zusammen etwas nach oben ziehen (Bild + Überschrift + Text)
  \vspace*{-2.5cm}   % vorher -2.0cm → jetzt 0.5cm höher

  % Phantom Section für TOC-Eintrag mit Nummer
  \phantomsection
  \refstepcounter{section}
  \addcontentsline{toc}{section}{\protect\numberline{\thesection}Anhang – Netzplan (Praxisnetz)}
  \label{sec:anhang-netzplan}

  % Bild + Überschrift im TikZ-Rahmen
  \begin{center}
    \begin{tikzpicture}[alt={Überschrift-Overlay für Netzplan-Diagramm}]
      % Bild-Knoten: seitenfüllend, linksbündig
      \node[inner sep=0, anchor=north west] (img) at (0,0) {%
        \includegraphics[
          width=\paperwidth,     % praktisch ganze Seite im Querformat
          height=0.84\textheight,
          keepaspectratio,
          alt={Netzplan der Praxis mit Router, Arbeitsplätzen und Netzwerk-Segmentierung. Zeigt die Trennung zwischen Praxisnetz und Standortnetz über einen dedizierten Router.}
        ]{assets/netzplan.png}%
      };

      % Überschrift oben links auf dem Bild mit Nummer
      \node[
        anchor=north west,
        xshift=-35mm,   % näher an die linke Kante (0mm = direkt am Bildrand)
        yshift=-4mm   % leicht ins Bild nach unten
      ] at (img.north west) {%
        \ifaccessible
          % Accessible: Keine TikZ-Überschrift
        \else
          % Standard: TikZ-Überschrift
          \usekomafont{sectioning}\Large \thesection{} Anhang – Netzplan (Praxisnetz)%
        \fi
      };
    \end{tikzpicture}
  \end{center}

  % Textblock unter dem Bild 0.6cm näher ans Bild heranziehen
  \vspace*{-0.6cm}   % vorher -0.5cm → jetzt 0.6cm näher am Bild

  % Info-Boxen nebeneinander
  \ifaccessible
    % Accessible: Einfache vertikale Struktur ohne minipage
    \textbf{\large \RouterKurz{} Konfiguration}\\
    \begin{small}
      \textbf{Modell:} \RouterModell{} (\RouterName)\\
      \textbf{LAN:} \PraxisSubnetz{}, Router: .1, DHCP: .20–.250\\
      \textbf{WLAN:} \WLANName{} (5GHz), WPA2/3, 23–07h aus\\
      \textbf{Gäste-WLAN:} \WLANGastName, WPA2/3, Mo–Fr 8–18h, Gastgeräte isoliert\\
      \textbf{Sicherheit:} Keine Portfreigaben, UPnP aus, Remote-Admin aus\\
      \textbf{Updates:} Firmware Auto-Update aktiviert
    \end{small}

    \vspace{0.5cm}
    \textbf{\large Netzwerk-Segmentierung}\\
    \begin{small}
      \textbf{Trennung:} \RouterModell{} isoliert Praxisnetz von \StandortTyp{}\\
      \textbf{Verkabelung:} \NetzwerkUplink{}\\
      \textbf{Netz:} Separates Subnetz \PraxisSubnetz{} mit eigenem DHCP\\
      \textbf{Firewall:} NAT/Firewall-Schutz gegen externe Zugriffe\\
      \textbf{Kontrolle:} Vollständige Verwaltungshoheit über Praxisgeräte\\
      \textbf{Physische Sicherheit:} \RouterKurz{} in abgeschlossenem Praxisraum
    \end{small}
  \else
    % Standard: minipage nebeneinander
    \begin{minipage}[t]{0.5\textwidth}
      \textbf{\large \RouterKurz{} Konfiguration}\\
      \begin{small}
        \textbf{Modell:} \RouterModell{} (\RouterName)\\
        \textbf{LAN:} \PraxisSubnetz{}, Router: .1, DHCP: .20–.250\\
        \textbf{WLAN:} \WLANName{} (5GHz), WPA2/3, 23–07h aus\\
        \textbf{Gäste-WLAN:} \WLANGastName, WPA2/3, Mo–Fr 8–18h, Gastgeräte isoliert\\
        \textbf{Sicherheit:} Keine Portfreigaben, UPnP aus, Remote-Admin aus\\
        \textbf{Updates:} Firmware Auto-Update aktiviert
      \end{small}
    \end{minipage}
    \hfill
    \begin{minipage}[t]{0.49\textwidth}
      \textbf{\large Netzwerk-Segmentierung}\\
      \begin{small}
        \textbf{Trennung:} \RouterModell{} isoliert Praxisnetz von \StandortTyp{}\\
        \textbf{Verkabelung:} \NetzwerkUplink{}\\
        \textbf{Netz:} Separates Subnetz \PraxisSubnetz{} mit eigenem DHCP\\
        \textbf{Firewall:} NAT/Firewall-Schutz gegen externe Zugriffe\\
        \textbf{Kontrolle:} Vollständige Verwaltungshoheit über Praxisgeräte\\
        \textbf{Physische Sicherheit:} \RouterKurz{} in abgeschlossenem Praxisraum
      \end{small}
    \end{minipage}
  \fi

\end{landscape}

\clearpage
\section*{} % Keine sichtbare Überschrift
\stepcounter{section} % Zähler erhöhen für korrekte Nummerierung
\addcontentsline{toc}{section}{\protect\numberline{\thesection}Anhang – IT-Notfall-Checkkarte}
\label{sec:anhang-notfallkarte}

\ifaccessible
  % Accessible: Einfache Struktur ohne minipage/tikzpicture
  \textbf{\Large IT-Notfall-Checkkarte}\\
  \textbf{Praxis für Psychotherapie \PraxisInhaberin}

  \subsection*{STUFE 1: Sofortmaßnahmen}
  \begin{enumerate}
    \item Ruhe bewahren -- Keine Panik
    \item System isolieren -- Netzwerk trennen
    \item Nichts löschen -- Spuren erhalten
    \item IT-Ansprechpartner kontaktieren
  \end{enumerate}

  \subsection*{STUFE 2: Dokumentation}
  Was, Wann, Wie notieren\\
  → Siehe Incident-Management (\cref{sec:notfall})

  \subsection*{STUFE 3: DSGVO-Meldung (72h)}
  Nur bei Risiko für Betroffene:\\
  Patientendaten unbefugt zugänglich/verloren

  Aufsichtsbehörde LfD\\
  \DatenschutzbehoerdeAdresse\\
  \BoxEmpty{} Art der Verletzung beschreiben\\
  \BoxEmpty{} Betroffene Daten/Personen angeben\\
  \BoxEmpty{} Bei hohem Risiko: Betroffene informieren

  \subsection*{Notfallkontakte}
  SOFORT ANRUFEN:\\
  IT: \ITAnsprechpartner\\
  Mobil: \ITAnsprechpartnerTelefon\\
  E-Mail: \href{mailto:\ITAnsprechpartnerMail}{\ITAnsprechpartnerMail}

  Praxis: \PraxisInhaberin\\
  Tel: \PraxisTelefon

  \subsection*{Weitere Kontakte}
  \PVSName{} Support: \PVSSupportTel\\
  SMC-B/eHBA: Herstellerportal

  Diese Karte ausdrucken und griffbereit halten!
\else
  % Standard: Original mit minipage/tikzpicture
  \begin{center}
    \textbf{\Large IT-Notfall-Checkkarte}\\
    \textbf{Praxis für Psychotherapie \PraxisInhaberin}
  \end{center}

  % Linke Spalte: Stufen-Ablauf
  \begin{minipage}[t]{0.48\textwidth}

    \subsection*{STUFE 1: Sofortmaßnahmen}
    \begin{enumerate}
      \item \textbf{Ruhe bewahren} – Keine Panik
      \item \textbf{System isolieren} – Netzwerk trennen
      \item \textbf{Nichts löschen} – Spuren erhalten
      \item \textbf{IT-Ansprechpartner kontaktieren}
    \end{enumerate}

    \subsection*{STUFE 2: Dokumentation}
    \textbf{Was, Wann, Wie notieren}\\
    → Siehe Incident-Management (\cref{sec:notfall})

    \subsection*{STUFE 3: DSGVO-Meldung (72h)}
    \textbf{Nur bei Risiko für Betroffene:}\\
    Patientendaten unbefugt zugänglich/verloren\\[0.2cm]
    \textbf{Aufsichtsbehörde LfD}\\
    \DatenschutzbehoerdeAdresse\\
    \BoxEmpty{} Art der Verletzung beschreiben\\
    \BoxEmpty{} Betroffene Daten/Personen angeben\\
    \BoxEmpty{} Bei hohem Risiko: Betroffene informieren

  \end{minipage}
  \hfill
  % Rechte Spalte: Kontakte & Support
  \begin{minipage}[t]{0.48\textwidth}

    \subsection*{Notfallkontakte}
    \begin{tikzpicture}
      \node[rectangle, draw=EmergencyBoxBorder, fill=EmergencyBoxFill, rounded corners=3pt,
      text width=6cm, align=left, inner sep=0.4cm] {
        \textbf{SOFORT ANRUFEN:}\\[0.1cm]
        \textbf{IT:} \ITAnsprechpartner\\
        \textbf{Mobil:} \ITAnsprechpartnerTelefon\\
        \textbf{E-Mail:} \href{mailto:\ITAnsprechpartnerMail}{\ITAnsprechpartnerMail}\\[0.2cm]
        \textbf{Praxis:} \PraxisInhaberin\\
        \textbf{Tel:} \PraxisTelefon\\[0.2cm]
      };
    \end{tikzpicture}

    \subsection*{Weitere Kontakte}
    \textbf{\PVSName{} Support:} \PVSSupport\\
    \textbf{SMC-B/eHBA:} Herstellerportal

  \end{minipage}

  \begin{center}
  \begin{tikzpicture}
    \node[rectangle, draw=EmergencyBoxBorder, fill=EmergencyBoxFill, rounded corners=5pt,
    text width=8cm, align=center, inner sep=0.5cm] {
      \textbf{Diese Karte ausdrucken und griffbereit halten!}
    };
  \end{tikzpicture}
  \end{center}
\fi


\clearpage
\section{Anhang – Selbstverpflichtung Mobile Geräte}
\label{sec:anhang-mobile}

\textbf{Geltungsbereich:} Diese Selbstverpflichtung gilt für den Umgang mit allen mobilen
Geräten (iPhone, iPad), die zur Verarbeitung von Praxisdaten eingesetzt werden.

\subsection*{Verpflichtungen}

\textbf{Erstkonfiguration und Gerätesicherheit:}
\begin{itemize}
  \item Ich führe vor Erstnutzung alle verfügbaren Sicherheitsupdates durch
  \item Ich nutze komplexe Entsperrcodes (min. 8-stellig) und biometrische Authentifizierung
  \item Ich aktiviere die automatische Gerätesperre nach max. 5 Minuten Inaktivität
  \item Ich melde mich nach Beendigung der Arbeit aktiv von allen Apps und Diensten ab
  \item Ich sperre das Gerät bei jeder Unterbrechung der Nutzung manuell
  \item Ich aktiviere automatische Betriebssystem- und App-Updates
  \item Ich schütze alle SIM-Karten durch eine PIN und verwahre Super-PIN/PUK sicher
  \item Ich verwende keine Jailbreak- oder Rooting-Verfahren
  \item Ich halte Siri deaktiviert (Datenschutz/Abhörschutz)
\end{itemize}

\textbf{App-Management und Datenübertragung:}
\begin{itemize}
  \item Ich installiere Apps ausschließlich aus dem offiziellen App Store
  \item Ich übertrage Patientendaten nur TLS-verschlüsselt (HTTPS/S-MIME)
  \item Erlaubte Datenübertragung: E-Mails, Kalendertermine (anonymisiert), Kontaktdaten
  \item Ich speichere keine Gesundheitsdaten in nicht-konformen Cloud-Diensten
  \item Ich trenne dienstliche und private Nutzung strikt
\end{itemize}

\textbf{Notfallverfahren und Entsorgung:}
\begin{itemize}
  \item Bei Verlust melde ich den Vorfall unverzüglich und sperre SIM-Karte (\MobilfunkAnbieter{}:
    \tel{\SIMSperrnummer{}})
  \item Ich aktiviere sofort Fernsperre und Fernlöschung
  \item Vor Entsorgung führe ich sicheren Werksreset mit Datenlöschung durch
  \item Ich dokumentiere alle Vorfälle vollständig
\end{itemize}

\praxissignatur

\BestandteilJaehrlich


\clearpage
\section{Anhang – Selbstverpflichtung Wechseldatenträger}
\label{sec:anhang-wechseldatentraeger}

\textbf{Geltungsbereich:} Diese Selbstverpflichtung gilt für den Umgang mit allen
Wechseldatenträgern (USB-Sticks, externe Festplatten, SD-Karten), die zur Verarbeitung von
Praxisdaten eingesetzt werden.

\subsection*{Verpflichtungen}
\textbf{Erstkonfiguration und Verschlüsselung:}
\begin{itemize}
  \item Ich prüfe neue Datenträger auf Malware und lösche vorhandene Daten vor Erstnutzung
  \item Ich verwende ausschließlich APFS-Verschlüsselung (BSI-TR-02102 konform) mit
    starken Passwörtern
  \item Ich speichere nur notwendige Daten und lösche sie nach Zweckfortfall unverzüglich
\end{itemize}

\textbf{Erlaubte Datenmitnahme:}
\begin{itemize}
  \item Verschlüsselte Backup-Daten für Offsite-Lagerung
  \item Leere, verschlüsselte USB-Sticks für Notfälle
  \item Keine Patientendaten außer in verschlüsselten Backup-Containern
  \item Mitnahme wird dokumentiert (Zweck, Datum, Rückgabe)
\end{itemize}

\textbf{Sicherheit und Entsorgung:}
\begin{itemize}
  \item Ich verwende grundsätzlich keine fremden USB-Sticks oder Wechseldatenträger
  \item Verluste melde ich unverzüglich und dokumentiere sie
  \item Zentrale Entsorgung mit sicherer Löschung, defekte Datenträger werden physisch vernichtet
\end{itemize}

\praxissignatur

\BestandteilJaehrlich


\clearpage
\section{Anhang – Selbstverpflichtung Sichere Datenträgerentsorgung}

\textbf{Geltungsbereich:} Diese Selbstverpflichtung gilt für alle IT-Geräte der Praxis,
die Patientendaten oder andere sensible Informationen enthalten oder enthalten haben.

\subsection*{Verpflichtung zur sicheren Löschung}

Ich verpflichte mich, vor jeder Entsorgung, jedem Verkauf oder jeder Weitergabe von
IT-Geräten (Computer, Smartphones, Tablets, Festplatten, USB-Sticks, etc.) eine sichere
Löschung aller darauf gespeicherten Daten durchzuführen.

\subsection*{Löschverfahren}

\textbf{SSDs und moderne Speicher:}
\begin{itemize}
  \item Verwendung der Herstellersoftware für Secure Erase/Crypto Erase
  \item Bei Apple-Geräten: \enquote{Alle Inhalte \& Einstellungen löschen} \\mit
    FileVault-Verschlüsselung
\end{itemize}

\textbf{Herkömmliche Festplatten:}
\begin{itemize}
  \item Mindestens dreifaches Überschreiben mit Zufallsdaten
  \item Alternativ: Mechanische Zerstörung der Datenträger
\end{itemize}

\textbf{Bei Defekten oder Unsicherheit:}
\begin{itemize}
  \item Physische Zerstörung der Datenträger
  \item Professionelle Datenvernichtung durch zertifizierten Dienstleister
\end{itemize}
\subsection*{Dokumentation}

Jede Entsorgung wird dokumentiert mit:
\begin{itemize}
  \item Geräteart und Seriennummer
  \item Datum der Löschung/Entsorgung
  \item Verwendete Löschmethode
  \item Bestätigung der vollständigen Löschung
\end{itemize}

\praxissignatur

\BestandteilJaehrlich


\clearpage
\section{Anhang – Vereinbarung über die Vertraulichkeit (extern)}
\label{sec:anhang-verschwiegenheit}

\subsection*{Vereinbarung über die Vertraulichkeit}

\textbf{1.} Zwischen

\ifaccessible
  % Accessible: Einfache Struktur ohne minipage
  \textbf{Auftraggeberin:}\\
  Praxis für Psychotherapie\\
  \PraxisInhaberin\\
  \PraxisAdresse\\
  \PraxisOrt

  \vspace{0.5cm}
  \textbf{Auftragnehmer:}\\
  \rule{5cm}{0.4pt}\\[0.2cm]
  \rule{5cm}{0.4pt}\\[0.2cm]
  \rule{5cm}{0.4pt}
\else
  % Standard: minipage-Struktur
  \begin{minipage}[t]{0.45\textwidth}
    \textbf{Auftraggeberin:}\\
    Praxis für Psychotherapie\\
    \PraxisInhaberin\\
    \PraxisAdresse\\
    \PraxisOrt
  \end{minipage}
  \hfill
  \begin{minipage}[t]{0.45\textwidth}
    \textbf{Auftragnehmer:}\\
    \rule{5cm}{0.4pt}\\[0.2cm]
    \rule{5cm}{0.4pt}\\[0.2cm]
    \rule{5cm}{0.4pt}
  \end{minipage}
\fi

wird folgende Vertraulichkeitsvereinbarung geschlossen:

\textbf{2.} Dem Auftragnehmer werden zum Zwecke der IT-Dienstleistungen vertrauliche
Informationen der Auftraggeberin bekannt.

\textbf{3.} Vertrauliche Informationen im Sinne dieser Vereinbarung sind sämtliche
Informationen (ob schriftlich, elektronisch, mündlich, digital verkörpert oder in anderer
Form), die von der Auftraggeberin an den Auftragnehmer zum vorgenannten Zweck offenbart
werden. Als vertrauliche Informationen gelten insbesondere Patientendaten,
Geschäftsgeheimnisse, geschäftliche Beziehungen, Geschäftsstrategien, Finanzplanung,
Personalangelegenheiten sowie das Bestehen dieser Vereinbarung und ihr Inhalt.

\textbf{4.} Keine vertrauliche Informationen sind solche Informationen, die der
Öffentlichkeit vor der Mitteilung oder Übergabe durch die Auftraggeberin bekannt oder
allgemein zugänglich waren oder dies zu einem späteren Zeitpunkt ohne Verstoß gegen eine
Geheimhaltungspflicht werden; die dem Auftragnehmer bereits vor der Offenlegung durch die
Auftraggeberin und ohne Verstoß gegen eine Geheimhaltungspflicht nachweislich bekannt
waren; die von dem Auftragnehmer ohne Nutzung oder Bezugnahme auf vertrauliche
Informationen selber gewonnen wurden; oder die der Auftragnehmer von einem berechtigten
Dritten ohne Verstoß gegen eine Geheimhaltungspflicht übergeben oder zugänglich gemacht werden.

\textbf{5.} Der Auftragnehmer verpflichtet sich, die vertraulichen Informationen streng
vertraulich zu behandeln und nur im Zusammenhang mit dem vorgenannten Zweck zu verwenden;
die vertraulichen Informationen nur gegenüber solchen Vertretern offen zu legen, die auf
die Kenntnis dieser Informationen angewiesen sind, vorausgesetzt, dass der Auftragnehmer
sicherstellt, dass seine Vertreter diese Vereinbarung einhalten, als wären sie selbst
durch diese Vereinbarung gebunden; die vertraulichen Informationen ebenfalls durch
angemessene Geheimhaltungsmaßnahmen gegen den unbefugten Zugriff durch Dritte zu sichern
und bei der Verarbeitung der vertraulichen Informationen die gesetzlichen und
vertraglichen Vorschriften zum Datenschutz einzuhalten. Dies beinhaltet auch dem
aktuellen Stand der Technik angepasste technische Sicherheitsmaßnahmen (Art.~32 DS-GVO)
und die Verpflichtung der Mitarbeiter auf die Vertraulichkeit und die Beachtung des
Datenschutzes (Art.~28 Abs.~3 lit.~b DS-GVO).

\textbf{6.} Sofern der Empfänger aufgrund geltender Rechtsvorschriften gerichtlicher oder
behördlicher Anordnungen verpflichtet ist, teilweise oder sämtliche vertraulichen
Informationen offenzulegen, die Auftraggeberin hierüber unverzüglich schriftlich zu
informieren und alle zumutbaren Anstrengungen zu unternehmen, um den Umfang der
Offenlegung auf ein Minimum zu beschränken und der Auftraggeberin erforderlichenfalls
jede zumutbare Unterstützung zukommen zu lassen.

\textbf{7.} Auf Aufforderung der Auftraggeberin sowie ohne Aufforderung spätestens nach
Erreichung oben genannten Zwecks ist der Auftragnehmer verpflichtet, sämtliche
vertraulichen Informationen einschließlich der Kopien hiervon innerhalb von zehn (10)
Arbeitstagen nach Zugang der Aufforderung bzw.\ nach Beendigung des Zwecks zurückzugeben
oder zu vernichten, sofern nicht mit der Auftraggeberin vereinbarte oder gesetzliche
Aufbewahrungspflichten dem entgegenstehen.

\textbf{8.} Auf Verlangen der Auftraggeberin hat der Auftragnehmer schriftlich zu
versichern, dass er sämtliche vertrauliche Informationen nach den Maßgaben der
vorstehenden Ziffer und den Weisungen der Auftraggeberin vollständig und unwiderruflich
gelöscht hat.

\textbf{9.} Verletzt der Auftragnehmer oder Mitarbeiter des Auftragnehmers oder sonstige
Personen, für die der Auftragnehmer gemäß §§~31, 278, 831 BGB einzustehen hat, die sich
aus dieser Vereinbarung ergebenden Pflichten, verpflichtet sich der Auftragnehmer für
jeden Fall der Zuwiderhandlung unter Ausschluss der Einrede des Fortsetzungszusammenhangs
eine Vertragsstrafe in Höhe von bis zu 25.000~EUR an die Auftraggeberin zu zahlen. Die
Höhe der Vertragsstrafe wird von der Auftraggeberin nach billigem Ermessen festgelegt.
Die Höhe der Vertragsstrafe ist vom zuständigen Gericht überprüfbar. Die sonstigen
Ansprüche, insbesondere etwaige Schadensersatzansprüche, auf die jedoch die
Vertragsstrafe angerechnet wird, bleiben unberührt.

\textbf{10.} Die Bestimmungen dieser Vereinbarung unterliegen in ihrer Durchführung und
Auslegung deutschem Recht unter Ausschluss des internationalen Privatrechts.
Ausschließlicher Gerichtsstand für Streitigkeiten aus oder im Zusammenhang mit der
Vereinbarung ist \Gerichtsstand.

\textbf{11.} Der Auftragnehmer hat ein Exemplar dieser Vereinbarung erhalten.

\vspace{1cm}

\ifaccessible
  % Accessible: Einfache Struktur ohne minipage
  \noindent
  Ort, Datum: \FormField[4cm] \hfill \PraxisInhaberin (\PraxisInhaberBezeichnung{}): \FormField[4cm]

  \vspace{2cm}
  \noindent
  Ort, Datum: \FormField[4cm] \hfill Auftragnehmer: \FormField[4cm]
\else
  % Standard: minipage-Struktur
  \begin{minipage}[t]{0.45\textwidth}
    \centering
    \rule{4cm}{0.3pt}\\[0.3cm]
    {\small Ort, Datum}
  \end{minipage}
  \hfill
  \begin{minipage}[t]{0.45\textwidth}
    \centering
    \rule{4cm}{0.3pt}\\[0.3cm]
    {\small \PraxisInhaberin \\
    (\PraxisInhaberBezeichnung{})}
  \end{minipage}

  \vspace{2cm}

  \begin{minipage}[t]{0.45\textwidth}
    \centering
    \rule{4cm}{0.3pt}\\[0.3cm]
    {\small Ort, Datum}
  \end{minipage}
  \hfill
  \begin{minipage}[t]{0.45\textwidth}
    \centering
    \rule{4cm}{0.3pt}\\[0.3cm]
    {\small Auftragnehmer}
  \end{minipage}
\fi

\vspace{0.5cm}

\textbf{Hinweis:} Diese Vereinbarung ist Bestandteil der IT-Sicherheitsdokumentation nach
§~390 SGB~V.


\clearpage
\section{Anhang – KBV-Compliance-Mapping}
\label{sec:kbv-compliance}

\subsection*{Zweck und Aufbau}

Dieses Kapitel dokumentiert die Umsetzung und den Erfüllungsstatus der Anforderungen der
IT-Sicherheitsrichtlinie nach §390 SGB V. Es dient als Nachweis gegenüber KBV-Prüfern
und ermöglicht die schnelle Lokalisierung relevanter Dokumentationsstellen.

\textbf{Aufbau:} Jede KBV-Anforderung wird mit Status und Verweis auf das entsprechende
Kapitel dieser Dokumentation verknüpft. Dadurch ist bei Prüfungen sofort ersichtlich,
wo die Umsetzung dokumentiert ist.

\subsection*{Übersicht Anlage 1 (IT-Sicherheitsrichtlinie §390 SGB V)}

\textit{Basis: KBV-Richtlinie v1.1 (gültig ab Oktober 2025)}

\small
\begin{longtable}{>{\raggedright\arraybackslash}p{0.7cm}
    >{\raggedright\arraybackslash}p{6.8cm} >{\centering\arraybackslash}p{1.2cm}
  >{\raggedright\arraybackslash}p{4.3cm}}
  \toprule
  \textbf{Nr.} & \textbf{Anforderung} & \Label{Status} &
  \textbf{Referenz/Kapitel} \\
  \midrule  \ifaccessible\else\endfirsthead
  \toprule
  \textbf{Nr.} & \textbf{Anforderung} & \Label{Status} &
  \textbf{Referenz/Kapitel} \\
  \midrule  \endhead\fi

  1 & Geregelte Einarbeitung neuer Mitarbeitender & + &~\ref{sec:personal} Personal
  (dokumentiert für zukünftige Einstellungen) \\
  2 & Geregelte Verfahrensweise beim Weggang von Mitarbeitenden & + &~\ref{sec:personal}
  Personal (dokumentiert für zukünftige Einstellungen) \\
  3 & Festlegung von Regelungen für den Einsatz von Fremdpersonal & +
  &~\ref{sec:personal} Personal \\
  4 & Vertraulichkeitsvereinbarungen für den Einsatz von Fremdpersonal & +
  &~\ref{sec:anhang-verschwiegenheit} Verschwiegenheit extern \\
  5 & Aufgaben und Zuständigkeiten von Mitarbeitenden & + &~\ref{sec:personal} Personal \\
  6 & Qualifikation des Personals & + &~\ref{sec:personal} Personal \\
  7 & Überprüfung der Vertrauenswürdigkeit von Mitarbeitenden & + &~\ref{sec:personal}
  Personal (Einstellungsverfahren dokumentiert) \\
  8 & Sensibilisierung der Praxisinhaberin für Informationssicherheit & +
  &~\ref{sec:sensibilisierung} Sensibilisierung \\
  9 & Einweisung des Personals in den sicheren Umgang mit IT & +
  &~\ref{sec:sensibilisierung} Sensibilisierung \\
  10 & Durchführung von Sensibilisierungen und Schulungen zur Informationssicherheit & +
  &~\ref{sec:sensibilisierung} Sensibilisierung und Schulung \\
  11 & Absicherung der Netzübergangspunkte & + &~\ref{sec:netzwerk} Netzwerksicherheit \\
  12 & Dokumentation des Netzes & + &~\ref{sec:anhang-netzplan} Netzplan \\
  13 & Grundlegende Authentisierung für den Netzmanagement-Zugriff & +
  &~\ref{sec:netzwerk} Netzwerksicherheit \\
  14 & Installation von Updates & + &~\ref{sec:patch} Patch-Management \\
  15 & Verantwortlichkeit für Updates & + &~\ref{sec:patch} Patch-Management \\
  16 & Identifizierung ausbleibender Updates & + &~\ref{sec:patch} Patch-Management \\
  17 & Ausmusterung oder Separierung bei ausbleibenden Updates & + &~\ref{sec:patch}
  Patch-Management \\
  18 & Verhinderung der unautorisierten Nutzung von Rechner-Mikrofonen und Kameras & +
  &~\ref{sec:endgeraete} Endgeräte-Sicherheit \\
  19 & Abmelden nach Aufgabenerfüllung & + &~\ref{sec:endgeraete} Endgeräte-Sicherheit \\
  20 & Einsatz von Viren-Schutzprogrammen & + &~\ref{sec:endgeraete} Endgeräte-Sicherheit \\
  21 & Regelmäßige Datensicherung & + &~\ref{sec:datensicherung} Datensicherung \\
  22 & Schutz der Datensicherung & + &~\ref{sec:datensicherung} Datensicherung \\
  23 & Art der Datensicherung & + &~\ref{sec:datensicherung} Datensicherung \\
  24 & Verantwortliche der Datensicherung & + &~\ref{sec:datensicherung} Datensicherung \\
  25 & Test der Datensicherung & + &~\ref{sec:vorlage-restore-tests} Restore-Tests \\
  26 & Der Zugriff auf Geräte und Software muss abgesichert werden & +
  &~\ref{sec:endgeraete} Endgeräte-Sicherheit \\
  27 & Konfiguration von Synchronisationsmechanismen (Windows) & -- &~\ref{sec:windows}
  Windows-Endgeräte \\
  28 & Datei- und Freigabeberechtigungen (Windows) & -- &~\ref{sec:windows} Windows-Endgeräte \\
  29 & Datensparsamkeit (Windows) & -- &~\ref{sec:windows} Windows-Endgeräte \\
  30 & Verwendung der SIM-Karten-PIN & + &~\ref{sec:mobile-geraete} Mobile Geräte \\
  31 & Sichere Grundkonfiguration für mobile Geräte & + &~\ref{sec:mobile-geraete} Mobile Geräte \\
  32 & Verwendung eines Zugriffschutzes & + &~\ref{sec:mobile-geraete} Mobile Geräte \\
  33 & Datenschutz-Einstellungen & + &~\ref{sec:mobile-geraete} Mobile Geräte \\
  34 & Sperrmaßnahmen bei Verlust eines Mobiltelefons & + &~\ref{sec:mobile-geraete}
  Mobile Geräte \\
  35 & Nutzung der Sicherheitsmechanismen von Mobiltelefonen & +
  &~\ref{sec:mobile-geraete} Mobile Geräte \\
  36 & Schutz vor Schadsoftware (Wechseldatenträger) & + &~\ref{sec:wechseldatentraeger}
  Wechseldatenträger \\
  37 & Angemessene Kennzeichnung der Datenträger beim Versand & n.a.
  &~\ref{sec:wechseldatentraeger} Kein Versand, nur Online-Übertragung \\
  38 & Sichere Versandart und Verpackung & n.a. &~\ref{sec:wechseldatentraeger} Wechseldatenträger \\
  39 & Sicheres Löschen der Datenträger vor und nach der Verwendung & +
  &~\ref{sec:wechseldatentraeger} Wechseldatenträger \\
  40 & Sichere Konfiguration der E-Mail-Clients & + &~\ref{sec:kommunikation} E-Mail-Sicherheit \\
  41 & Umgang mit Spam durch Benutzende & + &~\ref{sec:kommunikation} E-Mail-Sicherheit \\
  42 & Sichere Apps nutzen & + &~\ref{sec:mobile-geraete} Mobile Apps \\
  43 & Sichere Speicherung lokaler App-Daten & + &~\ref{sec:mobile-geraete} Mobile Apps \\
  44 & Verhinderung von Datenabfluss & + &~\ref{sec:mobile-geraete} Mobile Apps \\
  45 & Authentisierung bei Webanwendungen (Anbieter) & n.a. &~\ref{sec:kommunikation}
  Website-Sicherheit (statische Seite) \\
  46 & Schutz vertraulicher Daten (Anbieter) & n.a. &~\ref{sec:kommunikation} Website-Sicherheit \\
  47 & Einsatz von Web Application Firewalls (Anbieter) & n.a. &~\ref{sec:kommunikation}
  Website-Sicherheit \\
  48 & Schutz vor unerlaubter automatisierter Nutzung (Anbieter) & n.a.
  &~\ref{sec:kommunikation} Website-Sicherheit \\
  49 & Kryptografische Sicherung vertraulicher Daten (Anwender) & + &~\ref{sec:internet}
  Internet/Cloud \\
  50 & Sicherheit von Cloud-Dienstleistern & + &~\ref{sec:internet} Internet/Cloud \\
  \bottomrule
\end{longtable}

\subsection*{Übersicht Anlage 2 (Zusätzliche Anforderungen für mittlere Praxen) -
freiwillig (nicht verpflichtend für Praxen $\leq$5)}

\small
\begin{longtable}{>{\raggedright\arraybackslash}p{0.7cm}
    >{\raggedright\arraybackslash}p{6.8cm} >{\centering\arraybackslash}p{1.2cm}
  >{\raggedright\arraybackslash}p{4.3cm}}
  \toprule
  \textbf{Nr.} & \textbf{Anforderung} & \Label{Status} &
  \textbf{Referenz/Kapitel} \\
  \midrule  \ifaccessible\else\endfirsthead
  \toprule
  \textbf{Nr.} & \textbf{Anforderung} & \Label{Status} &
  \textbf{Referenz/Kapitel} \\
  \midrule  \endhead\fi

  1 & Alarmierung und Logging: Wichtige Ereignisse auf Netzkomponenten automatisch an zentrales Management-System übermitteln & ± &~\ref{sec:netzwerk} \RouterKurz{}-Logging, macOS-Systemlogs, kein zentrales SIEM (Einzelpraxis-angemessen) \\
  2 & Nutzung verschlüsselter Verbindungen (TLS): Kryptografische Algorithmen nach Stand der Technik verwenden & + &~\ref{sec:kommunikation} TLS 1.2+, HTTPS, S/MIME dokumentiert \\
  3 & Restriktive Rechtevergabe: Rechte nach Need-to-know-Prinzip, regelmäßige Überprüfung & + &~\ref{sec:rechte-rollenmatrix} Rollenmatrix, Einzelpraxis-Berechtigungen dokumentiert \\
  4 & Sichere zentrale Authentisierung in Windows-Netzen (Kerberos/SSO): Ausschließlich Kerberos für SSO & -- & Nicht anwendbar (macOS-Umgebung, kein Windows-Netz) \\
  5 & Richtlinie für mobile Geräte: Verbindliche Richtlinie für Mitarbeitende zur Benutzung mobiler Geräte & + &~\ref{sec:anhang-mobile} Mobile-Geräte-Richtlinie mit Unterschrift \\
  6 & Verwendung von Sprachassistenten: Nur bei zwingender Notwendigkeit, sonst deaktivieren & + &~\ref{sec:mobile-geraete} Siri deaktiviert, Hey Siri deaktiviert \\
  7 & Sicherheitsrichtlinien für Mobiltelefon-Nutzung: Nutzungs- und Sicherheitsrichtlinie erstellen & + &~\ref{sec:mobile-geraete} Mobiltelefon-Richtlinie dokumentiert \\
  8 & Sichere Datenübertragung über Mobiltelefone: Regelung welche Daten übertragen werden dürfen, Verschlüsselung & + &~\ref{sec:mobile-geraete} Nur organisatorische Daten, S/MIME-Verschlüsselung \\
  9 & Regelung zur Mitnahme von Wechseldatenträgern: Schriftliche Regeln für Transport außer Haus & + &~\ref{sec:wechseldatentraeger} Wechseldatenträger-Richtlinie mit Unterschrift \\
  10 & Minimierung und Kontrolle von App-Berechtigungen: App-Berechtigungen auf notwendiges Minimum einschränken & + &~\ref{sec:mdm} App-Berechtigungen restriktiv konfiguriert \\
  \bottomrule
\end{longtable}

\subsection*{Übersicht Anlage 5 (TI-Anforderungen)}

\small
\begin{longtable}{>{\raggedright\arraybackslash}p{0.7cm}
    >{\raggedright\arraybackslash}p{6.8cm} >{\centering\arraybackslash}p{1.2cm}
  >{\raggedright\arraybackslash}p{4.3cm}}
  \toprule
  \textbf{Nr.} & \textbf{Anforderung} & \Label{Status} &
  \textbf{Referenz/Kapitel} \\
  \midrule
  \ifaccessible\else\endfirsthead
  \toprule
  \textbf{Nr.} & \textbf{Anforderung} & \Label{Status} &
  \textbf{Referenz/Kapitel} \\
  \midrule
  \endhead\fi

  1 & Planung und Durchführung der Installation nach gematik-Vorgaben & + &~\ref{sec:ti}
  TI-Komponenten, \PVSName{}-Gateway nach Herstellervorgaben \\
  2 & Betrieb der TI-Komponenten & + &~\ref{sec:ti} TI-Betrieb dokumentiert \\
  3 & Schutz vor unberechtigtem physischem Zugriff & + &~\ref{sec:ti} SMC-B sicher
  aufbewahrt, Terminal am Praxisstandort \\
  4 & Internet-Verbindung parallel zur TI-Anbindung absichern & + &~\ref{sec:netzwerk}
  \RouterKurz{} Firewall, Netzwerksicherheit \\
  5 & Verbindung zu gehostetes TI-Gateway absichern (VPN) & + &~\ref{sec:ti}
  \PVSName{}-Gateway mit sicherer Verbindung \\
  6 & Beachtung der Vorgaben des TI-Gateway-Anbieters & + &~\ref{sec:ti} \PVSName{}-Vorgaben befolgt \\
  7 & Geschützte Kommunikation mit TI-Gateway (Zertifikate) & + &~\ref{sec:ti}
  Authentisierung über \PVSName{}-System \\
  8 & Zeitnahes Installieren verfügbarer Aktualisierungen & + &~\ref{sec:ti} Updates
  zeitnah eingespielt \\
  9 & Sicheres Aufbewahren von Administrationsdaten & + &~\ref{sec:ti}
  Administrationsdaten sicher aufbewahrt \\
  \bottomrule
\end{longtable}

\subsection*{Compliance-Status}

\Legende + = Erfüllt, ± = Teilweise erfüllt, × = Nicht erfüllt, n.a. = Nicht anwendbar

\textbf{Anlage 1:} 41 von 41 anwendbaren Anforderungen erfüllt (\textbf{100\% Compliance})
\begin{itemize}
  \item 41 Anforderungen: + Vollständig erfüllt
  \item 9 Anforderungen: n.a. Nicht anwendbar (Kein Versand: Nr. 37,38; Windows: Nr.
    27–29; Webdienst-Anbieter: Nr. 45–48)
\end{itemize}

\textbf{Anlage 2:} 7 von 8 anwendbaren Anforderungen vollständig erfüllt (\textbf{88\%
Compliance}), 1 teilweise erfüllt (\textbf{12\%})
\begin{itemize}
  \item 7 Anforderungen: + Vollständig erfüllt (Nr. 2,3,5,6,7,8,9,10)
  \item 1 Anforderung: ± Teilweise erfüllt (Nr. 1 - Logging vorhanden, kein zentrales SIEM)
  \item 1 Anforderung: -- Nicht anwendbar (Nr. 4 - macOS-Umgebung, kein Windows-Netz)
\end{itemize}

\textit{Hinweis: Für Praxen mit weniger als 6 Personen sind gemäß KBV die Anforderungen
  aus Anlage 1 und Anlage 5 umzusetzen (A.IV.1); Anlage 2 ist nicht verpflichtend, kann
aber freiwillig dokumentiert werden.}

\textbf{Anlage 5 (TI):} 9 von 9 Anforderungen erfüllt (\textbf{100\% Compliance})
\begin{itemize}
  \item 9 Anforderungen: + Vollständig erfüllt (Installation, Betrieb, Schutz, Updates,
    Administrationsdaten)
\end{itemize}

\textbf{Gesamtstatus:} Alle verpflichtenden Anforderungen gemäß KBV-Richtlinie v1.1
erfüllt (Anlage 1: 41 + Anlage 5: 9 = 50). Zusätzlich wurden Anlage-2-Anforderungen
(freiwillig) 7/8 vollständig und 1/8 teilweise umgesetzt. (Rechtsgrundlage: A.IV.1; neue
Pflichten ab 01.10.2025).

\textbf{KI-Software Compliance:} \KIToolName{} erfüllt die KBV-Anforderungen für KI-Software
gemäß KBV-Leitfaden \enquote{Praxiswissen KI} (S. 5): C5-Zertifizierung durch das BSI liegt vor.
Nachweis: \url{\KIToolSicherheitURL}

\normalsize
\vspace{1cm}
% \textit{Stand: \today}


\clearpage
\begin{landscape}
  \section{Anhang – Vollständige Geräteliste}
  \label{sec:geraeteliste}
  \setlength{\extrarowheight}{0.0ex}
  \scriptsize
  \begin{adjustbox}{max width=\linewidth}
    \csvreader[
      tabular=|P{1.6cm}|P{2.8cm}|P{3.0cm}|P{1.5cm}|P{1.6cm}|P{1.2cm}|P{1.3cm}|P{1.3cm}|P{1.4cm}|P{1.0cm}|P{3.5cm}|,
      table head=\hline
      \rowcolor{gray!20}\textbf{Kate\-gorie} &
      \Label{Name} &
      \textbf{Seriennummer} &
      \textbf{Verschlüs\-selung} &
      \textbf{Stand\-ort} &
      \textbf{Eigen\-tum} &
      \textbf{Nutzung} &
      \textbf{Backup} &
      \textbf{Verant\-wortlich} &
      \textbf{EOL} &
      \textbf{Bemerkungen}\\\hline,
      late after line=\\\hline,
      separator=comma
    ]{assets/geraeteliste.csv}{%
      Kategorie=\Kategorie,
      Name=\Name,
      Seriennummer=\Seriennummer,
      Verschluesselung=\Verschluesselung,
      Standort=\Standort,
      Eigentum=\Eigentum,
      Nutzung=\Nutzung,
      Backup=\Backup,
      Verantwortlich=\Verantwortlich,
      EOL=\EOL,
      Bemerkungen=\Bemerkungen
    }{%
      \Kategorie & \Name & \Seriennummer & \Verschluesselung & \Standort & \Eigentum & \Nutzung & \Backup & \Verantwortlich & \EOL & \Bemerkungen
    }
  \end{adjustbox}
\end{landscape}

\clearpage
\section{Anhang – Auftragsverarbeitungsverträge (AVV-Register)}
\label{sec:avv-register}

\subsection*{Auftragsverarbeitungsverträge (AVV)}

\textbf{Stand:} \today

\vspace{1em}

\begin{longtable}{>{\raggedright\arraybackslash}p{3.8cm} >{\raggedright\arraybackslash}p{4.2cm} >{\raggedright\arraybackslash}p{3.2cm} >{\raggedright\arraybackslash}p{2.3cm}}
  \toprule
  \textbf{Dienstleister} & \textbf{Verarbeitungs\-zweck} & \Label{Status} &
  \Label{Datum} \\
  \midrule

  \PVSHersteller{} & PVS/Behandlung & \textcolor{StatusGreen}{vorhanden} & 01.09.2025 \\

  \HostingAnbieter{} & Website-Hosting & \textcolor{StatusGreen}{vorhanden, C5-Testat geprüft \DocumentDate} & 01.09.2025 \\

  \EmailAnbieterFirma{} & E-Mail (organisatorisch, keine Gesundheitsdaten) &
  \textcolor{StatusGreen}{vorhanden} & 30.11.2025 \\

  Apple Inc. (Business Manager) & MDM/Backup (verschlüsselte iCloud-Backups, keine Gesundheitsdaten) &
  \textcolor{StatusGreen}{vorhanden} & 01.09.2025 \\

  \MDMTool{} Software LLC (\MDMTool{}) & Mobile Device Management & \textcolor{StatusGreen}{vorhanden} & 01.09.2025 \\

  \PatKomAnbieter{} & Sichere~Patienten-Kommunikation & \textcolor{StatusGreen}{vorhanden} & 01.09.2025 \\

  \KIToolAnbieter{} & KI-gestützte Sitzungsdokumentation & \textcolor{StatusGreen}{vorhanden, C5-Testat geprüft \DocumentDate} & 01.09.2025 \\

  \bottomrule
\end{longtable}

\vspace{1em}

\Legende
\begin{itemize}
  \item \textcolor{StatusGreen}{vorhanden} = AVV vorhanden und aktuell
  \item \textcolor{StatusOrange}{geplant} = AVV geplant/in Bearbeitung
  \item \textcolor{StatusRed}{zu prüfen} = Status zu prüfen
  \item \textcolor{StatusGray}{nicht erforderlich} = Kein AVV erforderlich/möglich
  \item \textcolor{StatusGray}{nicht verfügbar} = AVV derzeit nicht verfügbar
\end{itemize}

\vspace{1em}

\textbf{Apple Inc. – Detaillierter Status und Monitoring:}

\textbf{Aktuelle Situation:} Apple Business Manager Customer Data Processing Addendum
ist verfügbar und wurde aktiviert. Apple Business Manager wird ausschließlich für
Mobile Device Management (MDM) und verschlüsselte iCloud-Backups genutzt.
Gesundheitsdaten werden grundsätzlich nicht über Apple-Dienste verarbeitet.

\textbf{Wichtig:} E-Mail-Kommunikation erfolgt ausschließlich über \EmailAnbieter{}
(siehe Kap.~\ref{sec:kommunikation}). Kalender wird über \EmailAnbieter{} oder lokal verwaltet.

\textbf{Halbjährliches Monitoring:}
\begin{itemize}
  \item \textbf{AVV-Compliance} - Regelmäßige Prüfung der Vertragsbedingungen
  \item \textbf{Strikte Datentrennung} - Keine Gesundheitsdaten in Apple-Diensten
  \item \textbf{Bewertung} bei neuen Apple Business-Optionen für Gesundheitswesen
  \item \textbf{Nächste Prüfungen:} März 2026, September 2026
\end{itemize}

\textbf{Rechtliche Einordnung:} Nutzung erfolgt auf Basis berechtigter Interessen
(Art. 6 Abs. 1 lit. f DSGVO) für organisatorische Zwecke ohne Gesundheitsdatenbezug.
Gesundheitskommunikation erfolgt ausschließlich über \PatKomTool{} (mit AVV).

\subsection*{Hinweise zur Pflege}

\textbf{Regelmäßige Prüfung:} Dieses Register wird quartalsweise auf Aktualität geprüft
und bei Änderungen der Dienstleister oder Verträge aktualisiert. \textbf{AVV-Verträge
werden jährlich überprüft} (siehe TOMs, Maßnahme zur Auftragsverarbeitung).

\Dokumentation Die vollständigen AVV-Verträge werden digital archiviert
(\PasswortManager{}/sicherer Ordner) und sind bei Prüfungen elektronisch oder als Ausdruck
vorzulegen.

\textbf{Verantwortlichkeit:} Die \PraxisInhaberBezeichnung{} ist für die Vollständigkeit und
Aktualität der AVVs verantwortlich.


\clearpage
\section{Anhang – Verzeichnis von Verarbeitungstätigkeiten (VVT)}
\label{sec:vvt}

Das Verzeichnis von Verarbeitungstätigkeiten gemäß Art.\ 30 DSGVO dokumentiert alle
Datenverarbeitungen der Praxis systematisch. Jede Verarbeitungstätigkeit wird mit
Zweck, betroffenen Personen, Datenkategorien, Empfängern und Löschfristen erfasst.

\textbf{Hinweis:} Dieses VVT wird jährlich im Rahmen der IT-Sicherheitsevaluation
überprüft und bei Bedarf aktualisiert. Änderungen werden im Changelog dokumentiert.

\textbf{Rechtliche Grundlagen:}
\begin{itemize}
  \item Art.\ 6 Abs.\ 1 lit.\ b DSGVO (Vertragserfüllung)
  \item Art.\ 6 Abs.\ 1 lit.\ c DSGVO (rechtliche Verpflichtung)
  \item Art.\ 9 Abs.\ 2 lit.\ h DSGVO (Gesundheitsvorsorge)
\end{itemize}

\textbf{Datenschutzbeauftragte:} \Datenschutzbeauftragte{} (nicht
erforderlich gem. Art. 37 DSGVO: Einzelpraxis unter 20 Mitarbeitenden,
keine systematische Überwachung, Gesundheitsdatenverarbeitung nicht umfangreich)

\begin{samepage}
\VVTBox
  {blue!20}
  {Verarbeitungstätigkeit 1: Patientenbehandlung und Praxisverwaltung}
  {01.09.2025}
  {01.09.2025}
  {Behandlungsdokumentation, Terminplanung, Abrechnung mit Krankenkassen, Kommunikation mit anderen Leistungserbringern über KIM}
  {Patienten, gesetzliche Vertreter}
  {Stammdaten (Name, Adresse, Geburtsdatum), Gesundheitsdaten (Diagnosen, Behandlungsverläufe), Abrechnungsdaten, Kontaktdaten}
  {\textbf{Intern:} \PraxisInhaberBezeichnung{} \newline \textbf{Extern:} Krankenkassen, \KVLang, Ärztekammer \Bundesland, andere Ärzte/Psychotherapeuten (bei Überweisung), Finanzamt (rechtliche Verpflichtung), Steuerberater (Auftragsverarbeiter/Beratung), \PVSName{} GmbH (Auftragsverarbeiter)}
  {\VVTFooter{Nein}{10 Jahre nach Behandlungsende (§\,630f BGB)}}
\end{samepage}

\begin{samepage}
\VVTBox
  {green!20}
  {Verarbeitungstätigkeit 2: Praxiskommunikation und Terminorganisation}
  {01.09.2025}
  {01.09.2025}
  {E-Mail-Kommunikation mit Patienten, Terminplanung, organisatorische Absprachen. \textbf{Wichtig:} Keine Übermittlung von Gesundheitsdaten per E-Mail; sensible Inhalte ausschließlich über \PatKomTool{}.}
  {Patienten, Interessenten, Geschäftspartner}
  {E-Mail-Adressen, Termininfos mit Patientenkürzel, organisatorische Kommunikationsinhalte (KEINE Diagnosen, Symptome, Befunde). Bei Erhalt medizinischer Inhalte: Löschung + Hinweis an Patienten.}
  {\textbf{Intern:} \PraxisInhaberBezeichnung{} \newline \textbf{Extern:} \EmailAnbieter{} (E-Mail-Hosting, Deutschland), E-Mail-Empfänger}
  {\VVTFooter{Nein (\EmailAnbieter{}, Deutschland)}{Organisatorische Kommunikation max. 1 Jahr}}
\end{samepage}

\begin{samepage}
\VVTBox
  {orange!20}
  {Verarbeitungstätigkeit 3: Praxis-Website und Online-Präsenz}
  {01.09.2025}
  {01.09.2025}
  {Praxisdarstellung, Patienteninformation, Kontaktmöglichkeit}
  {Website-Besucher, Interessenten}
  {IP-Adressen, Browsertyp, Betriebssystem, Referrer-URL, Zugriffszeitpunkt (Server-Logs)}
  {\textbf{Intern:} \PraxisInhaberBezeichnung{} \newline \textbf{Extern:} \HostingAnbieter{} – Auftragsverarbeiter, Region \HostingRegion{}}
  {\VVTFooter{Ja (\CDNService{} global, \DNSService{} global - Website-Traffic und Besucherdaten werden weltweit verarbeitet, jedoch keine Gesundheitsdaten). Hosting in \HostingRegion{}. \HostingAnbieter{} verfügt über BSI-C5-Testat und Angemessenheitsbeschluss USA}{Server-Logs werden nach 30 Tagen automatisch gelöscht}}
\end{samepage}

\begin{samepage}
\VVTBox
  {gray!20}
  {Verarbeitungstätigkeit 4: Datensicherung und Backup}
  {01.09.2025}
  {01.09.2025}
  {Datensicherheit, Wiederherstellung bei Datenverlust, Geschäftskontinuität. Backups verschlüsselt, Zugriff nur durch \PraxisInhaberBezeichnung{}.}
  {Patienten, gesetzliche Vertreter (indirekt durch Backup der Behandlungsdaten)}
  {Vollständige Kopie aller Praxisdaten (Patientendaten, E-Mails, Dokumente)}
  {\textbf{Intern:} \PraxisInhaberBezeichnung{} \newline \textbf{Extern:} Keine (lokale Speicherung auf externen Festplatten und NAS am \BackupStandort{})}
  {\VVTFooter{Nein}{Automatische Ausdünnung durch Time Machine, Löschung bei Platzmangel. Gelöschte Daten verbleiben verschlüsselt bis zur Überschreibung.}}
\end{samepage}

\begin{samepage}
\VVTBox
  {purple!20}
  {Verarbeitungstätigkeit 5: Sichere Patientenkommunikation über \PatKomTool{}}
  {01.09.2025}
  {01.09.2025}
  {Verschlüsselte Übertragung sensibler Inhalte (Diagnosen, Befunde, Therapieinhalte)}
  {Patienten, gesetzliche Vertreter}
  {Behandlungsdaten, Diagnosen, Befunde, Therapieinhalte, Kontaktdaten}
  {\textbf{Intern:} \PraxisInhaberBezeichnung{} \newline \textbf{Extern:} \PatKomTool{} GmbH (Auftragsverarbeiter), Patienten (Empfänger)}
  {\VVTFooter{Nein}{Automatische Löschung, entsprechend der Aufbewahrungspflicht (10 Jahre)}}
\end{samepage}

\begin{samepage}
\VVTBox
  {cyan!20}
  {Verarbeitungstätigkeit 6: KI-gestützte Sitzungsdokumentation (\KIToolName{})}
  {01.09.2025}
  {01.09.2025}
  {Automatisierte Erstellung von Sitzungsnotizen und psychologischen Berichten mittels KI-Technologie. Temporäre Audioaufzeichnung und Transkription zur Dokumentationserstellung.}
  {Patienten (mit expliziter Einwilligung nach Art. 9 Abs. 2 lit. a DSGVO)}
  {Audioaufzeichnungen (temporär), Transkripte (temporär), Sitzungsinhalte, generierte Protokolle und Berichte}
  {\textbf{Intern:} \PraxisInhaberBezeichnung{}, Therapeut/in \newline \textbf{Extern:} \KIToolAnbieter{} (Auftragsverarbeiter, C5-zertifiziert), Microsoft Azure/Amazon Bedrock/Google Cloud (Sub-Auftragsverarbeiter für LLM-Services, alle C5-zertifiziert)}
  {\VVTFooter{Ja (EU-Verarbeitung durch C5-zertifizierte Cloud-Anbieter: Microsoft Azure EU, Amazon Bedrock EU, Google Cloud EU). Keine Datenspeicherung bei \KIToolName{} - automatische Löschung nach Verarbeitung}{Generierte Protokolle: 10 Jahre (wie Behandlungsdokumentation). Audio/Transkripte: Sofortige Löschung nach Protokollerstellung}}
\end{samepage}

\begin{samepage}
\VVTBox
  {red!20}
  {Verarbeitungstätigkeit 7: Mobile Device Management (\MDMTool{})}
  {01.09.2025}
  {01.09.2025}
  {Verwaltung und Sicherheit mobiler Geräte (iPhone/iPad), Konfiguration, App-Verteilung, Compliance-Überwachung, Remote-Management}
  {\PraxisInhaberBezeichnung{} (indirekt durch Geräteverwaltung)}
  {Geräte-IDs, Gerätename, Betriebssystemversion, App-Inventar, Konfigurationsstatus, Standortdaten (nur bei Verlust), Netzwerkinformationen}
  {\textbf{Intern:} \PraxisInhaberBezeichnung{} \newline \textbf{Extern:} \MDMTool{} Software LLC (Auftragsverarbeiter, USA mit EU-U.S. Data Privacy Framework)}
  {\VVTFooter{Ja (USA) – \MDMTool{} Software LLC ist EU-U.S. Data Privacy Framework zertifiziert und verfügt über DSGVO-konformen Auftragsverarbeitungsvertrag}{Gerätedaten werden nach Geräte-Deregistrierung gelöscht, spätestens nach 3 Jahren}}
\end{samepage}

\begin{samepage}
\VVTBox
  {yellow!20}
  {Verarbeitungstätigkeit 8: Praxistelefonie und Patientenkommunikation}
  {01.09.2025}
  {01.09.2025}
  {Terminvereinbarung, Patientenberatung, organisatorische Absprachen}
  {Patienten, gesetzliche Vertreter, Interessenten}
  {Telefonnummern, Verbindungsdaten, Gesprächsinhalte (temporär)}
  {\textbf{Intern:} \PraxisInhaberBezeichnung{} \newline \textbf{Extern:} \MobilfunkAnbieter{} (TK-Anbieter), Patienten}
  {\VVTFooter{Nein}{Verbindungsdaten nach gesetzlichen Vorgaben, keine Aufzeichnung von Gesprächen}}
\end{samepage}

\subsection*{Hinweise zur Pflege}

\textbf{Regelmäßige Prüfung:} Dieses VVT wird jährlich auf Vollständigkeit und Aktualität
geprüft und bei Änderungen der Verarbeitungstätigkeiten entsprechend aktualisiert.

\Dokumentation Das VVT ist bei Prüfungen durch Aufsichtsbehörden oder im Rahmen
von Compliance-Checks vorzulegen.

\textbf{Verantwortlichkeit:} Die \PraxisInhaberBezeichnung{} ist für die Vollständigkeit und
Aktualität des VVT verantwortlich.


\clearpage
\section{Changelog und Versionsverlauf}
\label{sec:changelog}

\subsection*{Versionshistorie}

\small
\begin{longtable}{p{3cm} >{\raggedright\arraybackslash}p{10cm}}
  \toprule
  \textbf{Version} & \textbf{Änderungen} \\
  \midrule
  \ifaccessible\else\endhead\fi

  2025.09.01 & Initial Release der vollständigen IT-Sicherheitsdokumentation nach §390 SGB V mit Sicherheitsmaßnahmen und Richtlinien \\

  & \\[1cm]

  & \\[1cm]

  & \\[1cm]

  & \\[1cm]

  & \\[1cm]

  \bottomrule
\end{longtable}
\normalsize


\clearpage
\section{Glossar}

\begin{description}
  \item[3–2-1 Regel] 3 Kopien, 2 Medien, 1 extern
  \item[AVV] Auftragsverarbeitungsvertrag nach Art.~28 DSGVO
  \item[\IaCTool{}] \IaCToolLang{}
  \item[BSI] Bundesamt für Sicherheit in der Informationstechnik
  \item[CI/CD] Continuous Integration/Continuous Deployment
  \item[CIA-Auswirkung] Bewertung der Auswirkungen auf Vertraulichkeit, Integrität und Verfügbarkeit
  \item[CIS Benchmark] Center for Internet Security -- Sicherheitsstandards für Systemkonfiguration
  \item[DKIM] DomainKeys Identified Mail -- E-Mail-Authentifizierungsverfahren
  \item[KIM] Kommunikation im Medizinwesen -- Sicherer Nachrichtendienst der TI für Leistungserbringer
  \item[DMARC] Domain-based Message Authentication, Reporting \& Conformance
  \item[DNS] Domain Name System -- Namensauflösung im Internet
  \item[DSGVO] Datenschutz-Grundverordnung
  \item[eHBA] Elektronischer Heilberufsausweis zur Authentifizierung in der Telematikinfrastruktur
  \item[EOL] End of Life -- Datum des Support-Endes durch Hersteller  \item[FileVault] macOS-Festplattenverschlüsselung
  \item[\RouterKurz{}] Router-System (z.B. AVM FRITZ!Box, Telekom Speedport)
  \item[GitHub Actions] Automatisierte CI/CD-Pipeline von GitHub
  \item[Infrastructure as Code] Verwaltung von IT-Infrastruktur durch Code
  \item[KBV] Kassenärztliche Bundesvereinigung
  \item[KI] Künstliche Intelligenz zur automatisierten Erstellung von Sitzungsnotizen und Berichten
  \item[\KV] \KVLang{} (Beispiel für \Bundesland)
  \item[NAS] Network Attached Storage (Netzwerkspeicher)
  \item[LfD] Landesbeauftragte für den Datenschutz (siehe \texttt{\textbackslash Datenschutzbehoerde})
  \item[MDM] Mobile Device Management -- Zentrale Verwaltung mobiler Geräte
  \item[PKN] \KammerLang{} (Beispiel für \Bundesland)
  \item[PVS] Praxisverwaltungssystem
  \item[SGB~V] Sozialgesetzbuch Fünftes Buch
  \item[SHA256] Kryptografische Hash-Funktion zur Integritätsprüfung
  \item[SIEM] Security Information and Event Management -- Zentrale Sicherheitsereignis-Analyse
  \item[SMC-B] Security Module Card Typ B (Praxisausweis)
  \item[SPF] Sender Policy Framework -- E-Mail-Authentifizierungsverfahren
  \item[S/MIME] Secure/Multipurpose Internet Mail Extensions
  \item[TI] Telematikinfrastruktur
  \item[TI-Gateway] VPN-basierte Alternative zum TI-Konnektor ohne lokale Hardware
  \item[Time Machine] macOS-Backup-System
  \item[TLS] Transport Layer Security
  \item[TOM] Technische und Organisatorische Maßnahmen
  \item[UPnP] Universal Plug and Play
  \item[VPN] Virtual Private Network
  \item[VVT] Verzeichnis von Verarbeitungstätigkeiten nach Art.~30 DSGVO
  \item[WireGuard] Modernes VPN-Protokoll
  \item[WPA2/WPA3] Wi-Fi Protected Access (WLAN-Verschlüsselung)
  \item[XProtect] macOS-Malware-Schutz
\end{description}


% Label für letzte Seite des Hauptdokuments (vor Arbeitsvorlagen)
\label{LastPage}

% TRENNSEITE
\clearpage
\ifaccessible
  % Accessible: Manuell twoside=false simulieren
  \let\cleardoublepage\clearpage
\else
  % Standard: KOMA twoside=false
  \KOMAoptions{twoside=false}
\fi
\thispagestyle{empty}
\begin{center}
  \vspace*{1.2cm}

  \praxislogo[2.5cm]

  {\LARGE\bfseries Arbeitsvorlagen}\\[2mm]
  {\large zur IT-Sicherheitsdokumentation}\\[8mm]
  {\Large \PraxisName}

  \vspace{1.0cm}

  \ifaccessible
    % Accessible: Einfache Struktur statt tikzpicture
    Regelmäßig zu füllende Arbeitsblätter\\
    für kontinuierliche Compliance-Dokumentation

    \textbf{Inhalt der Arbeitsvorlagen:}
    \begin{itemize}
      \item Jährliche IT-Sicherheitsevaluation
      \item Backup-Wiederherstellungstests (halbjährlich)
      \item TOMS-Überprüfungen (jährlich und quartalsweise)
      \item IT-Sicherheits-Schulungsplan
      \item Breach-Register (DSGVO Art. 33)
      \item SOP Remote-Support (unter Aufsicht)
      \item Betroffenenrechte (DSGVO Art. 12--23)
      \item Tabletop-Übung (Notfallsimulation)
      \item Mitarbeiter On-/Offboarding-Checklisten
    \end{itemize}
  \else
    % Standard: TikZ-Box
    \begin{tikzpicture}
      \node[rectangle, draw=InfoBoxBorder, fill=InfoBoxFill, rounded corners=5pt,
      text width=10cm, align=left, inner sep=0.6cm] {
        \begin{center}
          Regelmäßig zu füllende Arbeitsblätter\\
          für kontinuierliche Compliance-Dokumentation\\[0.3cm]
        \end{center}

        \vspace{0.3cm}

        \begin{center}
        \textbf{Inhalt der Arbeitsvorlagen:}
        \end{center}
        \begin{itemize}
          \item Jährliche IT-Sicherheitsevaluation
          \item Backup-Wiederherstellungstests (halbjährlich)
          \item TOMS-Überprüfungen (jährlich und quartalsweise)
          \item IT-Sicherheits-Schulungsplan
          \item Breach-Register (DSGVO Art. 33)
          \item SOP Remote-Support (unter Aufsicht)
          \item Betroffenenrechte (DSGVO Art. 12--23)
          \item Tabletop-Übung (Notfallsimulation)
          \item Mitarbeiter On-/Offboarding-Checklisten
        \end{itemize}
      };
    \end{tikzpicture}
  \fi

  \vspace{0.6cm}

  \vfill

  % Versions-Info am unteren Rand (wie Hauptdokument)
  \ifaccessible
    % Accessible: Einfacher Text statt tikzpicture
    \textbf{Stand:} \DocumentDate\\
    \textbf{Separate Mappe} \quad | \quad \textbf{Klassifizierung:} \DokumentKlassifizierung{}
  \else
    % Standard: TikZ-Box
    \begin{tikzpicture}
      \node[rectangle, draw=VersionBoxBorder, fill=VersionBoxFill, rounded corners=3pt,
      text width=12cm, align=center, inner sep=0.8cm] {
        \textbf{Stand:} \DocumentDate\\[0.2cm]
        \textbf{Separate Mappe} \quad | \quad \textbf{Klassifizierung:} \DokumentKlassifizierung{}
      };
    \end{tikzpicture}
  \fi

  % \vspace{1cm}
\end{center}

% ARBEITSVORLAGEN (neue Seitenzählung ab 1)
\clearpage
\maindocfalse
\newgeometry{left=2.5cm,right=2.5cm,top=2.5cm,bottom=2.5cm}
\setcounter{page}{1}
\renewcommand{\thepage}{V-\arabic{page}}

\cfoot{\thepage{} \quad | \quad Version \Version\TemplateAttribution}

\addcontentsline{toc}{section}{Arbeitsvorlagen (separate Mappe)}

%\section*{} % Keine sichtbare Überschrift
%\stepcounter{section} % Zähler erhöhen für korrekte Nummerierung
%\addcontentsline{toc}{section}{\protect\numberline{\thesection}Anhang – Jährliche
% IT-Sicherheitsevaluation}
\section{Vorlage – Jährliche IT-Sicherheitsevaluation}

\label{sec:vorlage-eigenpruefung}

%\subsection*{Jährliche IT-Sicherheitsevaluation (§390 SGB V)}
\textbf{Praxis für Psychotherapie \PraxisInhaberin}\\
%\textbf{Version der IT-Sicherheitsdokumentation:} \Version\\
\textbf{Zeitraum der Überprüfung:} Januar \rule{1.5cm}{0.3pt} -- Dezember \rule{1.5cm}{0.3pt}

%\Hinweis Die nachfolgende Tabelle ist als Template vorgesehen und wird im
%laufenden Betrieb handschriftlich geführt. Spätestens zum Zeitpunkt einer Prüfung liegen
%ausgefüllte Blätter vor.

\noindent\Hinweis Die jährliche IT-Sicherheitsevaluation dient dem Nachweis der
Aktualität gemäß \S390 SGB~V/KBV.
Pro Prüffeld Ja/Nein ankreuzen und ggf.\ kommentieren und Maßnahmen einleiten.

% Verwende einzelne Checkbox für cmark/xmark System
\newcommand{\CheckboxSize}{1.0ex}
\newcommand{\checkbox}{\fbox{\rule{0pt}{\CheckboxSize}\rule{\CheckboxSize}{0pt}}}

{%
\renewcommand{\arraystretch}{1.4}
\setlength{\tabcolsep}{6pt}
\footnotesize

\begin{longtable}{p{0.55\textwidth} p{0.16\textwidth} p{0.19\textwidth}}
  \toprule
  \textbf{Prüffeld} & \textbf{Erfüllt?} & \textbf{Bemerkungen} \\
  \midrule
  \ifaccessible\else\endhead\fi

  \textbf{Alle Geräte und Software erfasst?} (Geräteliste vollständig und aktuell) &
  \checkbox & \\ \midrule

  \textbf{End-of-Life-Geräte} (Router/Clients erhalten noch Sicherheitsupdates oder
  Austausch bis \NextReview{} geplant) & \checkbox & \\ \midrule

  \textbf{Netzwerkschutz} (\RouterKurz{} aktuell, keine Portfreigaben, WPA2/3 aktiv) &
  \checkbox & \\ \midrule

  \textbf{Änderungen der Netzwerkkonfiguration} (Alle Änderungen dokumentiert) &
  \checkbox & \\ \midrule

  \textbf{Endgeräte} (File\-Vault/Passcode aktiv, Auto-Update an) & \checkbox & \\ \midrule

  \textbf{Backups} (lokal + Offsite funktionsfähig, \textit{Restore-Test durchgeführt}) &
  \checkbox & \\ \midrule

  \textbf{TI-Komponenten} (Gateway/TI-Terminal aktuell, SMC-B sicher verstaut) & \checkbox & \\ \midrule

  \textbf{E-Mail-Sicherheit} (TLS/S/MIME, SPF/DKIM/DMARC, MTA-STS, CAA, Spam-Handling) & \checkbox & \\ \midrule

  \textbf{Website-Sicherheit} (HTTPS, HSTS, CSP, X-Content-Type-Options, Referrer-Policy,
  30-Tage-Logs) & \checkbox & \\ \midrule

  \textbf{Updates von Richtlinien oder Verfahren} (Alle Änderungen dokumentiert und
  umgesetzt) & \checkbox & \\ \midrule

  \textbf{S/MIME-Zertifikat} (Erneuerung alle 2 Jahre durchgeführt oder geplant) &
  \checkbox & \\ \midrule

  \textbf{Restore-Tests} (Mindestens jährlich, angestrebt halbjährlich Januar/Juli) &
  \checkbox & \\ \midrule

  \textbf{IT-Sicherheitsrichtlinie auf Aktualität geprüft?} (Richtlinien entsprechen
  aktuellen Anforderungen) & \checkbox & \\ \midrule

  \textbf{Passwort-Hygiene} (Starke Passwörter $\geq$\PasswortMindestlaenge{} Zeichen, 2FA aktiv,
  Security Score $\geq$\PasswortSecurityScore{}) & \checkbox & \\ \midrule

  \textbf{Ggf. vorhandene Sicherheitsvorfälle dokumentiert?} (Keine Vorfälle oder
  vollständig dokumentiert) & \checkbox & \\ \midrule

  \textbf{Ggf. vorhandene Entsorgung von Datenträgern dokumentiert?} (Keine Entsorgung
  oder sichere Löschung dokumentiert) & \checkbox & \\ \midrule

  \textbf{Schulung/Sensibilisierung} (Strukturierte Selbstschulung dokumentiert) &
  \checkbox & \\ \midrule

  \Label{Dokumentation} (Changelog gepflegt, Geräteliste aktuell) & \checkbox & \\
  \bottomrule

\end{longtable}
}% Ende der footnotesize-Gruppe für Tabelle

{\footnotesize
\Legende \fbox{\cmark} = Erfüllt, \fbox{\xmark} = Nicht erfüllt
}% Ende der footnotesize-Gruppe für Legende

\noindent\textbf{Ergebnis:} Die jährliche IT-Sicherheitsevaluation wurde am
\rule{3.8cm}{0.3pt} durchgeführt.\quad
Alle Anforderungen sind \BoxEmpty~erfüllt \quad/\quad Anpassungen \BoxEmpty~erforderlich.

\praxissignatur

% \vspace{1.5em}
% \noindent\textbf{Verwendung:}
% \begin{itemize}
%   \item Seite ausdrucken, ankreuzen, unterschreiben → im Ordner abheften
%   \item Für das Folgejahr Jahr in der Überschrift anpassen und neu generieren
%   \item Bei \enquote{Nein}-Antworten: Maßnahmen einleiten und dokumentieren
% \end{itemize}


\clearpage
\section{Vorlage – Protokoll der Backup-Wiederherstellungstests}
\label{sec:vorlage-restore-tests}

Restore-Tests werden mindestens einmal jährlich sowie anlassbezogen mit repräsentativen 
Testdateien durchgeführt und dokumentiert. Halbjährliche Tests (Januar/Juli) sind für 
Einzelpraxen wünschenswert, aber nicht verpflichtend,
um die vollständige Wiederherstellbarkeit aller Backup-Daten zu gewährleisten:

\textbf{Hinweis:} Die nachfolgende Tabelle ist als Template vorgesehen und wird im
laufenden Betrieb handschriftlich geführt. Spätestens zum Zeitpunkt einer Prüfung liegen
ausgefüllte Blätter vor.

% Stellschrauben - verwende gleiche Checkboxen wie TOMS
\providecommand{\qsize}{1.05ex}
\renewcommand{\qsize}{1.05ex}
\providecommand{\q}{\fbox{\rule{0pt}{\qsize}\rule{\qsize}{0pt}}}
\renewcommand{\q}{\fbox{\rule{0pt}{\qsize}\rule{\qsize}{0pt}}}
\renewcommand{\arraystretch}{1.0}% Zeilenhöhe wie TOMS
\setlength{\tabcolsep}{6pt}       % Zell-Innenabstand

% Sexy Tabelle mit booktabs (ohne vertikale Linien)
\begin{center}
  {\small
  \begin{tabular}{p{0.15\textwidth} p{0.15\textwidth} p{0.12\textwidth} p{0.42\textwidth}}
    \toprule
    \Label{Datum} & \textbf{Time Machine (Lokal)} & \textbf{NAS (Offsite)} & \textbf{Bemerkungen} \\
    \midrule
    \rule{0pt}{2.6ex} & \q & \q & \\
    \midrule
    \rule{0pt}{2.6ex} & \q & \q & \\
    \midrule
    \rule{0pt}{2.6ex} & \q & \q & \\
    \midrule
    \rule{0pt}{2.6ex} & \q & \q & \\
    \midrule
    \rule{0pt}{2.6ex} & \q & \q & \\
    \midrule
    \rule{0pt}{2.6ex} & \q & \q & \\
    \midrule
    \rule{0pt}{2.6ex} & \q & \q & \\
    \midrule
    \rule{0pt}{2.6ex} & \q & \q & \\
    \bottomrule
  \end{tabular}
  } % Ende der small-Gruppe
\end{center}

\Legende \fbox{\cmark} = Erfolgreich bestanden, \fbox{\xmark} = Fehler aufgetreten

\textbf{Anleitung zum Ausfüllen:}
\begin{itemize}
  \item \textbf{Datum:} Testdatum eintragen (TT.MM.JJJJ).
  \item \textbf{Bemerkungen:} Probleme oder besondere Vorkommnisse kurz notieren.
  \item \textbf{Gerät:} MacBook Air (fest vorgegeben).
  \item \textbf{Häufigkeit:} Alle 6 Monate (empfohlen: Januar \& Juli).
\end{itemize}


\clearpage
\section{Vorlage – TOMS-Überprüfungen}\label{sec:vorlage-toms-pruefung}

% Nur unteren Rand kleiner für mehr Platz
%\newgeometry{bottom=1cm}

\Hinweis Template zum handschriftlichen Ausfüllen. TOMS-Überprüfungen gemäß
Art.\ 32 DSGVO jährlich durchführen und 10 Jahre Aufbewahrung.

\textbf{Einzelpraxis-Hinweis:} Viele Prüfpunkte können jährlich statt quartalsweise 
geprüft werden. Kritische Punkte (Backups, Updates) sollten häufiger kontrolliert werden.

%% Präambel - definiere nur wenn nicht schon vorhanden
\providecommand{\qsize}{1.05ex}
\renewcommand{\qsize}{1.05ex}
\providecommand{\q}{\fbox{\rule{0pt}{\qsize}\rule{\qsize}{0pt}}}
\renewcommand{\q}{\fbox{\rule{0pt}{\qsize}\rule{\qsize}{0pt}}}
\providecommand{\QRow}{\q\kern0.25em\q\kern0.25em\q\kern0.25em\q}
\renewcommand{\QRow}{\q\kern0.25em\q\kern0.25em\q\kern0.25em\q}
\providecommand{\JRow}{\q}
\renewcommand{\JRow}{\q}

{%
\renewcommand{\arraystretch}{1.0}
\setlength{\tabcolsep}{3pt}
\footnotesize

\begin{longtable}{p{0.48\textwidth} p{0.05\textwidth} p{0.10\textwidth} p{0.25\textwidth}}
  \toprule
  \textbf{DSGVO-Maßnahme} & \textbf{Turnus} & \textbf{Erfüllt?} & \textbf{Bemerkungen} \\
  \midrule
  \ifaccessible\else\endhead\fi

  \AccessibleSectionHeader{Zugangs-/Zutrittskontrolle}
  \midrule
  Praxisräume abschließbar, Schlüssel nur Inhaberin & J & \JRow & \\
  Starke Authentifizierung (\PasswortManager{}, 2FA) & J & \JRow & \\
  \PasswortManager{} Sicherheits-Dashboard geprüft (Warnungen behoben) & Q & \QRow & \\
  Mobile Geräte mit Passcode/Touch ID gesichert & Q & \QRow & \\
  Automatische Bildschirmsperre aktiv & Q & \QRow & \\
  \midrule

  \AccessibleSectionHeader{Zugriffs-/Weitergabekontrolle}
  \midrule
  Rollenbasierte Zugriffe im PVS (\PVSName{}) & Q & \QRow & \\
  Keine gemeinsam genutzten Accounts & Q & \QRow & \\
  Verschlüsselte Datenübertragung (HTTPS) & Q & \QRow & \\
  \midrule

  \AccessibleSectionHeader{Eingabe-/Auftragskontrolle}
  \midrule
  \PVSName{}-Zugriffsprotokolle aktiviert & Q & \QRow & \\
  Auftragsverarbeitungsverträge (AVV) aktuell & J & \JRow & \\
  \midrule

  \AccessibleSectionHeader{Verfügbarkeits-/Trennungskontrolle}
  \midrule
  Backup-Konzept funktionsfähig & Q & \QRow & \\
  Separate Benutzerkonten für verschiedene Aufgaben & J & \JRow & \\
  \midrule

  \AccessibleSectionHeader{Verschlüsselung/Pseudonymisierung}
  \midrule
  Festplattenverschlüsselung (FileVault) aktiv & Q & \QRow & \\
  E-Mail-Verschlüsselung (S/MIME) funktionsfähig & Q & \QRow & \\
  S/MIME-Zertifikat gültig (>30 Tage) & Q & \QRow & \\
  S/MIME-Backup verfügbar & Q & \QRow & \\
  DMARC-Reports geprüft (postmaster@) & Q & \QRow & \\
  Verschlüsselte Backups und Wechseldatenträger & Q & \QRow & \\
  \midrule

  \AccessibleSectionHeader{Netzwerk-/Systemsicherheit}
  \midrule
  Firewall-Konfiguration (\RouterKurz{}) & Q & \QRow & \\
  Software-Updates und Patches aktuell & Q & \QRow & \\
  macOS-Firewall aktiviert & Q & \QRow & \\
  Gatekeeper und XProtect aktiv & Q & \QRow & \\
  System Integrity Protection (SIP) aktiviert & Q & \QRow & \\
  Secure Boot \enquote{Full Security} konfiguriert & Q & \QRow & \\
  USB-Zubehör bei Sperre blockiert & Q & \QRow & \\
  Sharing-Dienste deaktiviert & Q & \QRow & \\
  FileVault Recovery-Key sicher hinterlegt & Q & \QRow & \\
  \midrule

  \AccessibleSectionHeader{Physische Sicherheit}
  \midrule
  IT-Geräte in abschließbaren Räumen & J & \JRow & \\
  Bildschirme vor Einsicht geschützt, Sichtschutzfolien & J & \JRow & \\
  \midrule

  \AccessibleSectionHeader{Mobile Device Management (MDM)}
  \midrule
  Alle Geräte MDM-compliant (\MDMTool{}) & Q & \QRow & \\
  Blueprint-Konfiguration aktuell & Q & \QRow & \\
  Keine verbotenen Apps installiert & Q & \QRow & \\
  Remote Wipe verfügbar und konfiguriert & Q & \QRow & \\
  \midrule

  \AccessibleSectionHeader{Organisatorische Maßnahmen}
  \midrule
  Mitarbeiterschulungen durchgeführt & J & \JRow & \\
  Incident-Response-Verfahren aktuell & J & \JRow & \\
  Eigenprüfungsprotokoll (§390 SGB V) erstellt & J & \JRow & \\
  VVT (Verzeichnis Verarbeitungstätigkeiten) aktuell & J & \JRow & \\
  \bottomrule

\end{longtable}
}% Ende der footnotesize-Gruppe für Tabelle

{\footnotesize
\Legende \fbox{\cmark} = Erfüllt, \fbox{\xmark} = Nicht erfüllt, \fbox{$\bullet$} = Nicht zutreffend • J=Jährlich, Q=Quartalsweise (optional)
}% Ende der footnotesize-Gruppe für Legende

\noindent\textbf{Ergebnis:} TOMS-Überprüfung am \rule{2.5cm}{0.3pt} durchgeführt.
Alle Anforderungen \BoxEmpty~erfüllt / Anpassungen \BoxEmpty~erforderlich.

\praxissignatur

%\textbf{Verwendung:} Ausdrucken → ankreuzen → unterschreiben → abheften •
% \Legende J=Jährlich, Q=Quartalsweise (4 Boxen = Q1 Q2 Q3 Q4)

%\textbf{Hinweise:} Aufbewahrung 10 Jahre • Backup-Tests:~\ref{sec:vorlage-restore-tests}
% • IT-Evaluation:~\ref{sec:vorlage-eigenpruefung}

% Geometry für folgende Seiten zurücksetzen
\restoregeometry


\clearpage
\section{Vorlage – IT-Sicherheits-Schulungsplan}
\label{sec:vorlage-schulungsplan}

\Hinweis Template zum handschriftlichen Ausfüllen. IT-Sicherheitsschulungen
gemäß Art. 32 DSGVO jährlich durchführen (quartalsweise angestrebt) und 10 Jahre Aufbewahrung.

\subsection*{Jahresübersicht \FormField[2cm] \hfill Verantwortlich: \PraxisInhaberBezeichnung{}}

\ifaccessible
% Accessible: Einfache Textdarstellung ohne farbige Boxen
\subsubsection*{Q1 (Januar) -- Passwort-Sicherheit}
Dauer: 20 Min | Lernziel: Sichere Passwörter \& 2FA einrichten\\
Inhalt: \PasswortManager{}-Audit, \PVSName{}-Passwort prüfen, 2FA aktivieren\\
Checkliste: \q~\PasswortManager{}-Audit \q~\PVSName{}-Passwort \q~2FA aktiviert\\
Test: Mindestlänge sicherer Passwörter? \FormField[2cm]\\
Durchgeführt am: \FormField[3cm] Unterschrift: \FormField[4cm]

\subsubsection*{Q2 (April) -- Phishing-Erkennung}
Dauer: 25 Min | Lernziel: Phishing-Mails erkennen \& richtig reagieren\\
Inhalt: 5 Merkmale lernen, verdächtige E-Mail identifizieren, S/MIME-Zertifikat prüfen\\
Checkliste: \q~5 Merkmale \q~Verdächtige Mail identifiziert \q~S/MIME geprüft\\
Test: 3 Phishing-Merkmale nennen: \FormField[5cm]\\
Durchgeführt am: \FormField[3cm] Unterschrift: \FormField[4cm]

\subsubsection*{Q3 (Juli) -- Abmelden \& Bildschirmsperre}
Dauer: 15 Min | Lernziel: Arbeitsplatz sicher verlassen\\
Inhalt: Tastenkombinationen üben, automatische Sperre testen, Unterschiede verstehen\\
Checkliste: \q~Tastenkombinationen geübt \q~Auto-Sperre getestet \q~Unterschied verstanden\\
Test: Wann aktiv abmelden? \FormField[5cm]\\
Durchgeführt am: \FormField[3cm] Unterschrift: \FormField[4cm]

\subsubsection*{Q4 (Oktober) -- DSGVO \& Incident Response}
Dauer: 25 Min | Lernziel: Datenschutzvorfälle erkennen \& melden\\
Inhalt: 72h-Meldepflicht verstehen, LfD-Meldeformular ausfüllen, AVV-Register prüfen\\
Checkliste: \q~72h-Regel verstanden \q~Meldeformular ausgefüllt \q~AVV-Register geprüft\\
Test: Zuständige Behörde in \Bundesland? \FormField[4cm]\\
Durchgeführt am: \FormField[3cm] Unterschrift: \FormField[4cm]

\subsubsection*{Jahresbewertung}
Alle Module durchgeführt: \q~Ja \q~Nein (Begründung: \FormField[4cm])\\
Nachschulungen erforderlich: \q~Keine \q~Anzahl: \FormField[1cm]\\
Besondere Vorfälle: \q~Keine \q~Beschreibung: \FormField[4cm]\\
Sicherheitsniveau: \q~Hoch \q~Mittel \q~Verbesserung nötig\\
Datum: \FormField[3cm] Unterschrift \PraxisInhaberBezeichnung{}: \FormField[4cm]

\else
% Standard: Farbige Boxen
% Q1 Card - Blau
\noindent\fcolorbox{blue!50}{blue!20}{\parbox{\dimexpr\textwidth-2\fboxsep-2\fboxrule}{%
\textbf{Q1 (Januar) - Passwort-Sicherheit}\\[0.2cm]
\textbf{Dauer:} 20 Min \textbf{|} \textbf{Lernziel:} Sichere Passwörter \& 2FA einrichten\\
\textbf{Inhalt:} \PasswortManager{}-Audit, \PVSName{}-Passwort prüfen, 2FA aktivieren (Apple-ID)\\
\textbf{Checkliste:} \q~\PasswortManager{}-Audit \q~\PVSName{}-Passwort \q~2FA aktiviert\\
\textbf{Test:} Mindestlänge sicherer Passwörter? \FormField[2cm]\\
\textbf{Durchgeführt am:} \FormField[3cm] \textbf{Unterschrift:} \FormField[4cm]
}}

\vspace{0.3cm}

% Q2 Card - Grün
\noindent\fcolorbox{green!50}{green!20}{\parbox{\dimexpr\textwidth-2\fboxsep-2\fboxrule}{%
\textbf{Q2 (April) - Phishing-Erkennung}\\[0.2cm]
\textbf{Dauer:} 25 Min \textbf{|} \textbf{Lernziel:} Phishing-Mails erkennen \& richtig reagieren\\
\textbf{Inhalt:} 5 Merkmale lernen, verdächtige E-Mail identifizieren, S/MIME-Zertifikat prüfen\\
\textbf{Checkliste:} \q~5 Merkmale \q~Verdächtige Mail identifiziert \q~S/MIME geprüft\\
\textbf{Test:} 3 Phishing-Merkmale nennen: \FormField[5cm]\\
\textbf{Durchgeführt am:} \FormField[3cm] \textbf{Unterschrift:} \FormField[4cm]
}}

\vspace{0.3cm}

% Q3 Card - Orange
\noindent\fcolorbox{orange!50}{orange!20}{\parbox{\dimexpr\textwidth-2\fboxsep-2\fboxrule}{%
\textbf{Q3 (Juli) - Abmelden \& Bildschirmsperre}\\[0.2cm]
\textbf{Dauer:} 15 Min \textbf{|} \textbf{Lernziel:} Arbeitsplatz sicher verlassen\\
\textbf{Inhalt:} Tastenkombinationen üben, automatische Sperre testen, Unterschiede verstehen\\
\textbf{Checkliste:} \q~Tastenkombinationen geübt \q~Auto-Sperre getestet \q~Unterschied verstanden\\
\textbf{Test:} Wann aktiv abmelden? \FormField[5cm]\\
\textbf{Durchgeführt am:} \FormField[3cm] \textbf{Unterschrift:} \FormField[4cm]
}}

\vspace{0.3cm}

% Q4 Card - Rot
\noindent\fcolorbox{red!50}{red!20}{\parbox{\dimexpr\textwidth-2\fboxsep-2\fboxrule}{%
\textbf{Q4 (Oktober) - DSGVO \& Incident Response}\\[0.2cm]
\textbf{Dauer:} 25 Min \textbf{|} \textbf{Lernziel:} Datenschutzvorfälle erkennen \& melden\\
\textbf{Inhalt:} 72h-Meldepflicht verstehen, LfD-Meldeformular ausfüllen, AVV-Register prüfen\\
\textbf{Checkliste:} \q~72h-Regel verstanden \q~Meldeformular ausgefüllt \q~AVV-Register geprüft\\
\textbf{Test:} Zuständige Behörde in \Bundesland? \FormField[4cm]\\
\textbf{Durchgeführt am:} \FormField[3cm] \textbf{Unterschrift:} \FormField[4cm]
}}

\vspace{0.3cm}

\noindent\fcolorbox{gray!50}{gray!20}{\parbox{\dimexpr\textwidth-2\fboxsep-2\fboxrule}{%
\textbf{Jahresbewertung}\\[0.2cm]
\textbf{Alle Module durchgeführt:} \q~Ja \q~Nein (Begründung: \FormField[4cm])\\
\textbf{Nachschulungen erforderlich:} \q~Keine \q~Anzahl: \FormField[1cm]\\
\textbf{Besondere Vorfälle:} \q~Keine \q~Beschreibung: \FormField[4cm]\\
\textbf{Sicherheitsniveau:} \q~Hoch \q~Mittel \q~Verbesserung nötig\\[0.3cm]
\textbf{Datum:} \FormField[3cm] \textbf{Unterschrift \PraxisInhaberBezeichnung{}:} \FormField[4cm]
}}
\fi

\newpage

\subsection*{Grundlagen \& Verantwortlichkeiten}

\textbf{Rechtliche Grundlage:} TOMs Art. 32 DSGVO -- Schulung der Mitarbeitenden\\
\textbf{Durchführung:} \PraxisInhaberBezeichnung{} (Selbstschulung + zukünftige Mitarbeitende)\\
\textbf{Teilnahmepflicht:} \PraxisInhaberBezeichnung{} (Selbstschulung), bei Neueinstellungen alle Mitarbeitenden vor Systemzugang\\
\Dokumentation Teilnahmeblatt + Unterschrift + \enquote{Verstanden}-Checkbox\\
\textbf{Nachschulung:} Bei Verständnisproblemen binnen 2 Wochen\\
\textbf{Review:} Jährlich + bei aktuellen Bedrohungen

\subsection*{Detaillierte Schulungsinhalte}

\ifaccessible\else\begin{samepage}\fi
\ifaccessible
  \EinfacheSchulungsbox
    {Q1: Passwort-Sicherheit}
    {20 Min | Praxisbüro, PC mit Internetzugang | Max. 3 Teilnehmer}
    {Sichere Passwörter gemäß BSI-Empfehlungen (min. 12--16 Zeichen, Einzigartigkeit, Passwortmanager) erstellen und für alle relevanten Dienste (\PVSName{}, Apple-ID, Microsoft 365) Zwei-Faktor-Authentifizierung einrichten}
    {DSGVO Art. 32 (Integrität \& Vertraulichkeit)}
    {BSI \enquote{Sichere Passwörter} (bsi.bund.de) + eigene \PasswortManager{}-Doku}
    {\q~\PasswortManager{}-Audit durchgeführt \q~\PVSName{}: min. \PasswortMindestlaenge{} Zeichen, Bildschirmsperre 5min geprüft \q~2FA bei Apple-ID, Microsoft 365, \PVSName{} aktiviert}
    {\enquote{Wie lang muss ein sicheres Passwort mindestens sein?} Antwort: \FormField[3cm]}
\else
  % Standard: Tabelle
  \ifaccessible\begin{tabular}{p{3.5cm} p{11.0cm}}\else\begin{longtable}{p{3.5cm} p{11.0cm}}\fi
  \toprule
  \AccessibleTableHeader{Q1: Passwort-Sicherheit}{}\\
  \midrule
  \textbf{Dauer \& Ort} & 20 Min | Praxisbüro, PC mit Internetzugang | Max. 3 Teilnehmer \\
  \textbf{Lernziel} & Sichere Passwörter gemäß BSI-Empfehlungen (min. 12--16 Zeichen, Einzigartigkeit, Passwortmanager) erstellen und für alle relevanten Dienste (\PVSName{}, Apple-ID, Microsoft 365) Zwei-Faktor-Authentifizierung einrichten \\
  \textbf{Rechtsbezug} & DSGVO Art. 32 (Integrität \& Vertraulichkeit) \\
  \textbf{Quellen} & BSI \enquote{Sichere Passwörter} (bsi.bund.de) + eigene \PasswortManager{}-Doku \\
  \textbf{Checkliste} & \q~\PasswortManager{}-Audit durchgeführt \q~\PVSName{}: min. \PasswortMindestlaenge{} Zeichen, Bildschirmsperre 5min geprüft \q~2FA bei Apple-ID, Microsoft 365, \PVSName{} aktiviert \\
  \textbf{Mini-Test} & \enquote{Wie lang muss ein sicheres Passwort mindestens sein?} Antwort: \FormField[3cm] \\
  \Label{Dokumentation} & Teilnahmeblatt \\
  \bottomrule
  \ifaccessible\end{tabular}\else\end{longtable}\fi
\fi
\ifaccessible\else\end{samepage}\fi

\ifaccessible\else\begin{samepage}\fi
\ifaccessible\begin{tabular}{p{3.5cm} p{11.0cm}}\else\begin{longtable}{p{3.5cm} p{11.0cm}}\fi

  \toprule
  \AccessibleTableHeader{Q2: Phishing-Erkennung}{}\\
  \midrule

  \textbf{Dauer \& Ort} & 25 Min | Praxisbüro, PC mit E-Mail-Zugang | Max. 3 Teilnehmer \\
  \textbf{Lernziel} & Phishing-Mails anhand von mindestens fünf typischen Merkmalen (z.\,B. Absender, Links, Dringlichkeit, Sprache, Anhänge) zuverlässig erkennen und im Verdachtsfall korrekt reagieren (Nicht-Öffnen, Melden, Löschen) \\
  \textbf{Rechtsbezug} & BSI-Empfehlungen \enquote{Phishing erkennen} + TOMs Awareness-Maßnahmen \\
  \textbf{Quellen} & BSI Newsletter + \KV-Material + Top-3-Bedrohungen: Fake-KV-Mails, Fake-\PVSName{}-Updates, Fake-Patientenanfragen \\
  \textbf{Checkliste} & \q~5 Erkennungsmerkmale Phishing gelernt \q~Verdächtige E-Mail im Postfach identifiziert \q~S/MIME-Zertifikat-Gültigkeit geprüft \\
  \textbf{Mini-Test} & \enquote{Woran erkenne ich eine Phishing-Mail?} (3 Merkmale nennen) Antwort: \FormField[8cm] \\
  \Label{Dokumentation} & Teilnahmeblatt \\
  \bottomrule
\ifaccessible\end{tabular}\else\end{longtable}\fi
\ifaccessible\else\end{samepage}\fi

\ifaccessible\else\begin{samepage}\fi
\ifaccessible\begin{tabular}{p{3.5cm} p{11.0cm}}\else\begin{longtable}{p{3.5cm} p{11.0cm}}\fi

  \toprule
  \AccessibleTableHeader{Q3: Abmelden und Bildschirmsperre}{}\\
  \midrule

  \textbf{Dauer \& Ort} & 15 Min | Arbeitsplatz, alle verwendeten Geräte | Max. 3 Teilnehmer \\
  \textbf{Lernziel} & Nach jeder Arbeitsunterbrechung aktiv abmelden oder Bildschirm sperren und Tastenkombinationen für alle verwendeten Geräte kennen (macOS: Control+Command+Q, iOS: Power-Taste) \\
  \textbf{Rechtsbezug} & KBV Anlage 1 Nr. 19 -- Abmelden nach Aufgabenerfüllung \\
  \textbf{Quellen} & BSI \enquote{Arbeitsplatz absichern} + eigene Gerätedokumentation \\
  \textbf{Checkliste} & \q~Abmelde-Tastenkombinationen an allen Geräten geübt \q~Automatische Bildschirmsperre (5 Min) aktiviert und getestet \q~Unterschied aktives Abmelden vs. Bildschirmsperre verstanden \\
  \textbf{Mini-Test} & \enquote{Wann müssen Sie sich aktiv abmelden?} (2 Situationen nennen) Antwort: \FormField[8cm] \\
  \Label{Dokumentation} & Teilnahmeblatt \\
  \bottomrule
\ifaccessible\end{tabular}\else\end{longtable}\fi
\ifaccessible\else\end{samepage}\fi

\ifaccessible\else\begin{samepage}\fi
\ifaccessible\begin{tabular}{p{3.5cm} p{11.0cm}}\else\begin{longtable}{p{3.5cm} p{11.0cm}}\fi

  \toprule
  \AccessibleTableHeader{Q4: DSGVO \& Incident Response}{}\\
  \midrule

  \textbf{Dauer \& Ort} & 25 Min | Praxisbüro, PC mit Internetzugang | Max. 3 Teilnehmer \\
  \textbf{Lernziel} & Datenschutzvorfälle erkennen, innerhalb von 72 Stunden an die zuständige Aufsichtsbehörde melden und das AVV-Register sowie das VVT auf Aktualität prüfen \\
  \textbf{Rechtsbezug} & DSGVO Art. 33 (Meldung von Verletzungen), Art. 30 (VVT) \\
  \textbf{Quellen} & \Datenschutzbehoerde{} + eigene Incident-Dokumentation \\
  \textbf{Checkliste} & \q~72h-Meldepflicht-Kriterien verstanden \q~LfD-Meldeformular ausgefüllt (Testszenario) \q~AVV-Register auf Vollständigkeit geprüft \\
  \textbf{Mini-Test} & \enquote{Welche Behörde ist für DSGVO-Meldungen in \Bundesland{} zuständig?} Antwort: \FormField[6cm] \\
  \Label{Dokumentation} & Teilnahmeblatt \\
  \bottomrule
\ifaccessible\end{tabular}\else\end{longtable}\fi
\ifaccessible\else\end{samepage}\fi

\subsection*{Zusätzliche Informationen}

\textbf{Quellen regelmäßiger Sicherheitsinformationen:}
\begin{itemize}
  \item BSI Newsletter \enquote{Einfach • Cybersicher} (monatlich, praxisnah)
  \item BSI Technische Sicherheitshinweise (RSS-Feed, fortgeschritten)
  \item \KVLang{} + \KammerLang{} (Rundmails)
\end{itemize}

\textbf{Anlassbezogene Schulungen:}
\begin{itemize}
\item \textbf{Bei Neueinstellung:} Alle 4 Module in ersten 4 Wochen
\item \textbf{Aktuelle Bedrohungen:} BSI-Newsletter → zeitnahe Kurz-Schulung (innerhalb 1 Woche)
\item \textbf{Nach Vorfällen:} Incident-Analyse + verstärkte Schulung betroffener Bereiche
\end{itemize}

\textbf{Dokumentation \& Aufbewahrung:}
Diese Schulungsdokumentation ist Bestandteil der IT-Sicherheitsdokumentation nach § 390 SGB V und wird 10 Jahre aufbewahrt.


\clearpage
\section{Vorlage – Breach-Register (DSGVO Art. 33)}
\label{sec:vorlage-breach-register}

\Hinweis Template zur Dokumentation aller Datenschutzvorfälle mit 72h-Bewertung,
Meldeentscheidung und Maßnahmen. Nachweise werden in der Arbeitsmappe abgelegt. 10 Jahre Aufbewahrung.

%% Präambel - definiere nur wenn nicht schon vorhanden
\providecommand{\qsize}{1.05ex}
\renewcommand{\qsize}{1.05ex}
\providecommand{\q}{\fbox{\rule{0pt}{\qsize}\rule{\qsize}{0pt}}}
\renewcommand{\q}{\fbox{\rule{0pt}{\qsize}\rule{\qsize}{0pt}}}

{%
\renewcommand{\arraystretch}{1.1}
\setlength{\tabcolsep}{4pt}
\footnotesize

\begin{longtable}{p{0.28\textwidth} p{0.68\textwidth}}
\toprule
\AccessibleTableHeader{Breach-Protokoll}{}
\midrule
\textbf{ID / Ticket} & \\[0.2cm]
\textbf{T0 (Entdeckung)} & Datum \FormField[2.5cm] \quad Uhrzeit \FormField[2cm] \\[0.2cm]
\textbf{Meldende Person} & Name \FormField[4cm] Kontakt \FormField[3cm] \\[0.2cm]
\textbf{Kurzbeschreibung} & Was ist passiert? Welche Systeme/Prozesse betroffen? \\
& \FormField[8cm] \\
& \FormField[8cm] \\[0.2cm]
\textbf{Datenarten / Umfang} & Kategorien (z.\,B. Gesundheitsdaten, Kontakt), \# betroffene Personen \\
& \FormField[8cm] \\[0.2cm]
\textbf{CIA-Auswirkung} & \q~Vertraulichkeit \q~Integrität \q~Verfügbarkeit \\
& Bewertung: \q~niedrig \q~mittel \q~hoch \\[0.2cm]
\textbf{Sofortmaßnahmen} & Eindämmung, Trennung vom Netz, Passwortrücksetzungen \\
& \FormField[8cm] \\
& \FormField[8cm] \\[0.2cm]
\textbf{72h-Timer} & Start: T0 \quad Deadline: T0 + 72h \\
& Zwischenstatus: \FormField[4cm] \\[0.2cm]
\textbf{Risikobewertung} & \q~gering \q~mittel \q~hoch \\
& Begründung: \FormField[5cm] \\[0.2cm]
\textbf{Meldeentscheidung} & \textbf{LfD melden?} \q~Ja \q~Nein \\
& Begründung: \FormField[5cm] \\
& \textbf{Betroffene informieren?} \q~Ja \q~Nein \\[0.2cm]
\textbf{Meldungen} & LfD: Datum \FormField[2.5cm] Uhrzeit \FormField[2cm] \\
& Aktenzeichen: \FormField[4cm] \\
& Betroffene: Datum \FormField[2.5cm] Art: \q~Brief \q~E-Mail \\[0.2cm]
\textbf{Abschluss-} & Patches, Key-Rotation, Härtung, Wiederherstellung \\
\textbf{maßnahmen} & \FormField[8cm] \\
& \FormField[8cm] \\[0.2cm]
\textbf{Erkenntnisse} & Prozess-/Technik-Anpassungen, Schulungsbedarf \\
& \FormField[8cm] \\
& \FormField[8cm] \\[0.2cm]
\textbf{Verantwortlich} & Name: \FormField[4cm] Funktion: \FormField[2.5cm] \\
\textbf{Freigabe} & Datum: \FormField[2.5cm] Unterschrift: \FormField[3.5cm] \\
\bottomrule
\end{longtable}

}% Ende der footnotesize-Gruppe

\textbf{Anleitung}

\textbf{Sofortmaßnahmen bei Verdacht:}
\begin{itemize}
  \item[\q] Vorfall dokumentieren (Zeitpunkt, Umstände, Beweise sichern)
  \item[\q] Schadensbegrenzung einleiten (System isolieren, Zugriffe sperren)
  \item[\q] IT-Ansprechpartner informieren (siehe IT-Notfallkarte)
  \item[\q] 72h-Timer starten und Risikobewertung durchführen
\end{itemize}

\textbf{Meldepflicht \Datenschutzbehoerde:}
\begin{itemize}
  \item \textbf{Immer melden:} Wenn Risiko für Rechte und Freiheiten der Betroffenen
  \item \textbf{Kontakt:} \DatenschutzbehoerdeAdresse
  \item \textbf{Frist:} 72 Stunden nach Kenntniserlangung
  \item \textbf{Formular:} Online-Meldeformular auf \DatenschutzbehoerdeURL{}
\end{itemize}

\BestandteilZehnJahre

\Legende \fbox{\cmark} = Erledigt, \fbox{\xmark} = Fehler aufgetreten, \q = Noch nicht durchgeführt


\clearpage
\section{Vorlage – SOP Remote-Support (unter Aufsicht)}
\label{sec:vorlage-remote-support}

\Hinweis Template zur sicheren Durchführung von Fernwartung durch externe
Dienstleister. 10 Jahre Aufbewahrung.

%% Präambel - definiere nur wenn nicht schon vorhanden
\providecommand{\qsize}{1.05ex}
\renewcommand{\qsize}{1.05ex}
\providecommand{\q}{\fbox{\rule{0pt}{\qsize}\rule{\qsize}{0pt}}}
\renewcommand{\q}{\fbox{\rule{0pt}{\qsize}\rule{\qsize}{0pt}}}

\textbf{Zweck \& Geltung:} Diese SOP regelt Fernwartung durch externe Dienstleister für Praxis-IT (z.\,B. PVS,
Netzwerk, Endgeräte). Ziel ist eine nachvollziehbare, sichere Durchführung unter
Aufsicht der Praxisinhaberin bei minimalen Rechten und lückenloser Protokollierung.

\textbf{Voraussetzungen:}
\begin{itemize}
  \item Zugelassen sind nur benannte Dienstleister lt. AVV/Vertrag; Identität wird vor jeder Sitzung geprüft.
  \item Kein dauerhafter Fernzugang. Sitzungen sind \emph{terminiert}, \emph{zeitlich begrenzt} und \emph{einmalig freizugeben}.
  \item Nutzung eines \emph{temporären, eingeschränkten} Kontos; \emph{kein} Einsatz persönlicher Konten der Dienstleister.
  \item Bildschirmfreigabe \emph{unter Aufsicht}; Dateiübertragungen nur nach expliziter Zustimmung.
  \item Aktive Schutzfunktionen (AV, Firewall, FileVault) bleiben eingeschaltet; Protokollierung ist aktiviert.
\end{itemize}

\textbf{Ablauf:}

\textbf{Vorher:}
\begin{itemize}
  \item[\q] Ticket/Anlass definieren (Fehlerbild, Ziel, betroffene Systeme)
  \item[\q] Temporäres Konto anlegen, \emph{Least-Privilege} (nur benötigte Rechte)
  \item[\q] Zeitfenster vereinbaren; Nutzer informiert; Backups aktuell
\end{itemize}

\textbf{Während:}
\begin{itemize}
  \item[\q] Sitzung unter Aufsicht der Praxisinhaberin/IT-Ansprechpartner:in
  \item[\q] Änderungsliste mitschreiben (Komponenten, Befehle, Versionen)
  \item[\q] Keine Persistenz hinterlassen (keine dauerhaften Dienste/Tunnel)
\end{itemize}

\textbf{Danach:}
\begin{itemize}
  \item[\q] Sitzung beenden; temporäre Konten/Token \emph{sofort} sperren/löschen
  \item[\q] Funktionstest (Smoke-Test); ggf. Rollback aus Backup
  \item[\q] Ticket schließen mit Freigabe durch Praxisinhaberin
\end{itemize}

\textbf{Abbruchkriterien:} Unklare Identität, ungeplante Dateiübertragungen oder Deaktivierung von Schutzmechanismen.

{%
\renewcommand{\arraystretch}{1.0}
\setlength{\tabcolsep}{3pt}
\footnotesize

\begin{longtable}{p{0.30\textwidth} p{0.66\textwidth}}
\toprule
\AccessibleTableHeader{Remote-Support Protokoll}{}
\midrule
\textbf{Dienstleister / Techniker} & \FormField[7cm] \\[0.15cm]
\textbf{Ticket / Anlass / Ziel} & \FormField[7cm] \\
& \FormField[7cm] \\[0.15cm]
\textbf{Datum / Zeit / Dauer} & \FormField[7cm] \\[0.15cm]
\textbf{Betroffene Systeme} & (Host, OS, Inventar-ID) \FormField[5cm] \\[0.15cm]
\textbf{Zugänge} & (temporäres Konto, Rechte, 2FA) \FormField[5cm] \\[0.15cm]
\textbf{Durchgeführte Änderungen} & (Schritte, Versionen, Konfigs) \\
& \FormField[7cm] \\
& \FormField[7cm] \\[0.15cm]
\textbf{Dateiübertragungen} & (Quelle/Ziel, Zweck) \FormField[5cm] \\[0.15cm]
\textbf{Ergebnis / Tests} & (Smoke-Test ok/Fehler) \FormField[5cm] \\[0.15cm]
\textbf{Nacharbeiten} & (Passwortrotation, Cleanup) \FormField[5cm] \\[0.15cm]
\textbf{Freigabe Praxisinhaberin} & Name: \FormField[3cm] Datum: \FormField[2cm] \\
& Unterschrift: \FormField[4cm] \\
\bottomrule
\end{longtable}

}% Ende der footnotesize-Gruppe

\BestandteilZehnJahre

\Legende \fbox{\cmark} = Erledigt, \fbox{\xmark} = Fehler aufgetreten, \q = Noch nicht durchgeführt


\clearpage
\section{Vorlage – Betroffenenrechte (Art. 12--23 DSGVO)}
\label{sec:vorlage-betroffenenrechte}

\Hinweis Template zur strukturierten Bearbeitung von Betroffenenanfragen
nach DSGVO. Frist: 1 Monat, max. +2 Monate bei Komplexität. 10 Jahre Aufbewahrung.

%% Präambel - definiere nur wenn nicht schon vorhanden
\providecommand{\qsize}{1.05ex}
\renewcommand{\qsize}{1.05ex}
\providecommand{\q}{\fbox{\rule{0pt}{\qsize}\rule{\qsize}{0pt}}}
\renewcommand{\q}{\fbox{\rule{0pt}{\qsize}\rule{\qsize}{0pt}}}

{%
\renewcommand{\arraystretch}{1.0}
\setlength{\tabcolsep}{3pt}
\footnotesize

\begin{longtable}{p{0.32\textwidth} p{0.64\textwidth}}
\toprule
\AccessibleTableHeader{Betroffenenanfrage-Protokoll}{}
\midrule
\textbf{Eingangsdatum} & \FormField[3cm] \quad \textbf{Kanal:} \q~Brief \q~E-Mail \q~Persönlich \\[0.15cm]
\textbf{Antragsteller/in} & Name: \FormField[5cm] \\
& Adresse: \FormField[5cm] \\[0.15cm]
\textbf{Identitätsprüfung} & \q~Geburtsdatum \q~Adresse \q~Weitere: \FormField[3cm] \\
& \q~Identität bestätigt \q~Nachfrage erforderlich \\[0.15cm]
\textbf{Art der Anfrage} & \q~Auskunft (Art. 15) \q~Berichtigung (Art. 16) \\
& \q~Löschung (Art. 17) \q~Einschränkung (Art. 18) \\
& \q~Datenübertragbarkeit (Art. 20) \q~Widerspruch (Art. 21) \\[0.15cm]
\textbf{Beschreibung} & Konkrete Anfrage/gewünschte Maßnahme: \\
& \FormField[6cm] \\
& \FormField[6cm] \\[0.15cm]
\textbf{Fristberechnung} & Eingang: \FormField[2.5cm] Antwortfrist: \FormField[2.5cm] \\
& \q~Fristverlängerung (+2 Mon.) Neue Frist: \FormField[2.5cm] \\[0.15cm]
\textbf{Datenbestandsprüfung} & \q~PVS/\PVSName{} \q~E-Mail-Archive \q~Papierakten \q~Backups \\
& \q~Weitere Systeme: \FormField[4cm] \\[0.15cm]
\textbf{Ergebnis} & \q~Vollständig erfüllt \q~Teilweise erfüllt \\
& \q~Begründet abgelehnt \q~Nicht möglich \\
& Begründung: \FormField[5cm] \\[0.15cm]
\textbf{Antwort versendet} & Datum: \FormField[2.5cm] \q~Brief \q~E-Mail \\
& \q~Datenauskunft \q~Bestätigung \q~Ablehnung \q~Rechtsverweis \\[0.15cm]
\textbf{Nachweise/Anlagen} & \q~Kopie Anfrage \q~Kopie Antwort \q~Datenauskunft \\
& \q~Identitätsprüfung Ablageort: \FormField[4cm] \\[0.15cm]
\textbf{Bearbeitung} & Verantwortlich: \FormField[3cm] Datum: \FormField[2cm] \\
& Unterschrift: \FormField[4cm] \\
\bottomrule
\end{longtable}

}% Ende der footnotesize-Gruppe

\textbf{Rechtliche Grundlagen:}

\textbf{Bearbeitungsfristen:}
\begin{itemize}
  \item \textbf{Regelfall:} 1 Monat • \textbf{Komplexe Fälle:} +2 Monate • \textbf{Identitätszweifel:} Frist nach Klärung
\end{itemize}

\textbf{Häufige Anfragen:}
\begin{itemize}
  \item \textbf{Art. 15:} Auskunft • \textbf{Art. 16:} Berichtigung • \textbf{Art. 17:} Löschung • \textbf{Art. 21:} Widerspruch
\end{itemize}

\textbf{Ablehnungsgründe:}
\begin{itemize}
  \item Aufbewahrungspflicht (§ 630f BGB - 10 Jahre) • Berechtigte Interessen • Identität unklar
\end{itemize}

\BestandteilZehnJahre

\Legende \fbox{\cmark} = Erledigt, \fbox{\xmark} = Fehler aufgetreten, \q = Noch nicht durchgeführt


\clearpage
\section{Vorlage – Tabletop-Übung (Notfallsimulation)}
\label{sec:vorlage-tabletop}

\Hinweis Template für jährliche Notfall-Tabletop-Übungen zur Überprüfung
der Business-Continuity-Prozesse. 10 Jahre Aufbewahrung.

%% Präambel - definiere nur wenn nicht schon vorhanden
\providecommand{\qsize}{1.05ex}
\renewcommand{\qsize}{1.05ex}
\providecommand{\q}{\fbox{\rule{0pt}{\qsize}\rule{\qsize}{0pt}}}
\renewcommand{\q}{\fbox{\rule{0pt}{\qsize}\rule{\qsize}{0pt}}}

{%
\renewcommand{\arraystretch}{1.0}
\setlength{\tabcolsep}{3pt}
\footnotesize

\begin{longtable}{p{0.30\textwidth} p{0.66\textwidth}}
\toprule
\AccessibleTableHeader{Tabletop-Übung Protokoll}{}
\midrule
\textbf{Datum / Uhrzeit} & \FormField[4cm] \quad Dauer: \FormField[2cm] Min \\[0.15cm]
\textbf{Teilnehmende} & \FormField[6cm] \\
& \FormField[6cm] \\[0.15cm]
\textbf{Szenario} & \q~Ransomware \q~Geräteverlust \q~Netzwerkausfall \q~Datenschutzvorfall \\
& \q~Sonstiges: \FormField[4cm] \\[0.15cm]
\textbf{Szenario-Details} & Beschreibung der simulierten Situation: \\
& \FormField[6cm] \\
& \FormField[6cm] \\[0.15cm]
\textbf{Rollenverteilung} & Incident Commander: \FormField[3cm] \\
& IT-Ansprechpartner: \FormField[3cm] \\
& Weitere Rollen: \FormField[3cm] \\[0.15cm]
\textbf{Durchgeführte Maßnahmen} & \q~Sofortmaßnahmen \q~Eskalationswege \q~Kommunikation \\
& \q~Wiederherstellung \q~Dokumentation \\[0.15cm]
\textbf{Erkenntnisse} & Was lief gut? \FormField[5cm] \\
& Was war unklar/problematisch? \FormField[4cm] \\[0.15cm]
\textbf{Identifizierte Lücken} & Prozess-Lücken: \FormField[4cm] \\
& Schulungsbedarf: \FormField[4cm] \\
& Technische Verbesserungen: \FormField[4cm] \\[0.15cm]
\textbf{Maßnahmen} & 1. \FormField[4cm] bis \FormField[1.5cm] \\
& 2. \FormField[4cm] bis \FormField[1.5cm] \\
& 3. \FormField[4cm] bis \FormField[1.5cm] \\[0.15cm]
\textbf{Nächste Übung} & Geplant für: \FormField[3cm] Fokus: \FormField[3cm] \\[0.15cm]
\textbf{Freigabe} & Übungsleiter: \FormField[3cm] Datum: \FormField[2cm] \\
& Unterschrift: \FormField[4cm] \\
\bottomrule
\end{longtable}

}% Ende der footnotesize-Gruppe

\textbf{Typische Szenarien:}

\textbf{Ransomware-Angriff:}
\begin{itemize}
  \item Verschlüsselung der Praxis-IT • Fokus: Isolation, Backup-Wiederherstellung, Meldung
\end{itemize}

\textbf{Geräteverlust/Diebstahl:}
\begin{itemize}
  \item MacBook mit Patientendaten gestohlen • Fokus: Meldung, Betroffeneninformation
\end{itemize}

\textbf{Netzwerkausfall:}
\begin{itemize}
  \item Internet/Telefon ausgefallen • Fokus: Alternative Kommunikation, Wiederanlauf
\end{itemize}

\BestandteilZehnJahre

\Legende \fbox{\cmark} = Erfolgreich durchgeführt, \fbox{\xmark} = Verbesserungsbedarf, \q = Noch nicht bearbeitet


\clearpage
\section{Vorlage – Einarbeitung neuer Mitarbeitende}
\label{sec:vorlage-einarbeitung}

\textbf{Geltungsbereich:} Diese Checkliste gilt für die systematische Einarbeitung aller
neuen Mitarbeitenden vor Aufnahme der Tätigkeit.

\subsection*{Einarbeitung-Checkliste}

\textbf{Vor dem ersten Arbeitstag:}
\begin{itemize}
  \item[\q] Arbeitszeugnis und Referenzen geprüft
  \item[\q] Verschwiegenheitserklärung unterschrieben (Anhang)
  \item[\q] IT-Sicherheitsschulung durchgeführt (Schulungsplan)
  \item[\q] Benutzerkonten eingerichtet (\PVSName{}, E-Mail, etc.)
  \item[\q] Passwort-Manager-Zugang erstellt (\PasswortManager{})
  \item[\q] Mobile-Device-Richtlinie erklärt und unterschrieben
  \item[\q] Arbeitsplatz und Sicherheitsmaßnahmen gezeigt
  \item[\q] Notfallkontakte und -verfahren erklärt
\end{itemize}

\textbf{Zusätzliche Einarbeitung:}
\begin{itemize}
  \item[\q] IT-Sicherheitsdokumentation ausgehändigt und erklärt
  \item[\q] DSGVO-Schulung durchgeführt (TOMs Art. 32)
  \item[\q] Praxisspezifische Arbeitsabläufe erklärt
  \item[\q] Backup- und Notfallverfahren demonstriert
  \item[\q] Erste Woche: Tägliche Rücksprache über Sicherheitsfragen
\end{itemize}

\textbf{Verantwortlich:} \PraxisInhaberBezeichnung{} | \textbf{Frist:} Vor erstem Arbeitstag

\vspace{1cm}

\textbf{Bestätigung \PraxisInhaberBezeichnung{}:}\\
Ich bestätige, dass die Einarbeitung vollständig nach obiger Checkliste durchgeführt wurde.

\praxissignatur

\textbf{Bestätigung Fachkraft:}\\
Ich bestätige, dass ich die Einarbeitung erhalten und alle Sicherheitsrichtlinien verstanden habe.

\vspace{1.5cm}
\noindent\makebox[0.4\textwidth]{\hrulefill} \hfill \makebox[0.4\textwidth]{\hrulefill}\\
\makebox[0.4\textwidth][c]{Datum} \hfill \makebox[0.4\textwidth][c]{Unterschrift Fachkraft}

\BestandteilZehnJahre


\clearpage
\section{Vorlage – Offboarding Mitarbeitende}
\label{sec:vorlage-offboarding}

\textbf{Geltungsbereich:} Diese Checkliste gilt für das strukturierte Offboarding aller
ausscheidenden Mitarbeitenden am letzten Arbeitstag.

\subsection*{Offboarding-Checkliste}

\textbf{Rückgabe und Deaktivierung:}
\begin{itemize}
  \item[\q] Alle Unterlagen und Dokumente zurückgegeben
  \item[\q] Schlüssel und Zugangsberechtigungen eingezogen
  \item[\q] IT-Geräte (Laptop, Handy, etc.) zurückgegeben
  \item[\q] Benutzerkonten deaktiviert (\PVSName{}, E-Mail, \PasswortManager{})
  \item[\q] Passwörter geändert (geteilte Accounts)
  \item[\q] Datenträger-Rückgabe dokumentiert
  \item[\q] Persönliche Daten von Arbeitsgeräten entfernt
  \item[\q] Zugriff auf Cloud-Dienste entzogen
\end{itemize}

\textbf{Abschlussgespräch:}
\begin{itemize}
  \item[\q] Verschwiegenheitsverpflichtung nochmals betont
  \item[\q] Fortdauernde Datenschutzpflichten erklärt
  \item[\q] Exit-Gespräch mit Sicherheitshinweisen geführt
  \item[\q] Kontaktverbot zu Patientinnen und Patienten erklärt
  \item[\q] Rückfragen zu Sicherheitsverfahren beantwortet
\end{itemize}

\textbf{Verantwortlich:} \PraxisInhaberBezeichnung{} | \textbf{Frist:} Letzter Arbeitstag

\vspace{0.5cm}

\textbf{Bestätigung \PraxisInhaberBezeichnung{}:}\\
Ich bestätige, dass das Offboarding vollständig nach obiger Checkliste durchgeführt wurde.

\praxissignatur

\textbf{Bestätigung Fachkraft:}\\
Ich bestätige die vollständige Rückgabe aller Unterlagen und Geräte sowie das Verständnis der fortdauernden Verschwiegenheitspflichten.

\vspace{0.5cm}
\noindent\makebox[0.4\textwidth]{\hrulefill} \hfill \makebox[0.4\textwidth]{\hrulefill}\\
\makebox[0.4\textwidth][c]{Datum} \hfill \makebox[0.4\textwidth][c]{Unterschrift Fachkraft}

\BestandteilZehnJahre


% Geometry zurücksetzen (falls später noch Inhalte folgen)
\restoregeometry

\end{document}
